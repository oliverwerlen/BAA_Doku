% !TEX root = PREN2_Dokumentation.tex
\section{Evaluation und Validation}\label{Evaluation}
\subsection{Ziel der Arbeit}
Durch die progressive Web App «JTI Pick-Up Station» ist es dem Kunden möglich, seine Ware bequem im Onlineshop zu bestellen und direkt und ohne Wartezeit an der gewünschten Pick-Up Station abzuholen. Durch den Prototyp sollen die Funktionalität und Zweckmässigkeit dieses für die Firma neuen Absatzkanals aufgezeigt werden. Im besten Fall findet die Applikation nicht nur in der Schweiz Verwendung, sondern wird von JTI auch in anderen Märkten weltweit eingesetzt. Das Hauptaugenmerk der Arbeit liegt auf der Implementierung eines Prototyps mit den folgenden Schwerpunkten: Bestellung, Kauf, Nutzererfassung, Suche nach Pick-Up Stations und der Abholung an der Station. Bei der Nutzererfassung muss das Alter des Nutzers verifiziert werden. Die Lösung soll so weit als möglich in die Projektpartner-Systeme integriert werden. Hinzu kommt die Recherche von artverwandten Technologien und das Requirements Engineering. \\\\

\subsection{Validierung der Requirements}
Das gesamte Funktionsmuster wurde in verschiedene Requirements aufgeteilt. Die genaue Beschreibung der Requirements ist in \ref{RSR} zu finden. Durch die Validierung wird die Funktionsfähigkeit des Funktionsmusters bewiesen. 
\subsubsection{Funktionale Requirements}
\paragraph{PWA-Checklist}
Es werden die Kriterien von Google genutzt. [\cite{pwaChecklist}]
\subparagraph{Core Progressive Web App checklist}
\begin{itemize}
	\item [{$\times$}] Starts fast, stays fast
	\item [{$\checkmark$}] Works in any browser
	\item [{$\checkmark$}] Responsive to any screen size
	\item [{$\checkmark$}] Provides Custom offline page
	\item [{$\checkmark$}] Is Installable
\end{itemize}
\subparagraph{Optimal Progressive Web App checklist}
Es wurde bei der Implementierung sogar noch weiter gegangen, als dass es dieses Requirement vorschreibt. So wurden die meisten Punkte von der Optimal checklist ebenfalls abgeschlossen. 
\begin{itemize}
	\item [{$\checkmark$}] Provides an offline experience
	\item [{$\checkmark$}] Is fully accessible
	\item [{$\times$}] Can be discovered through search
	\item [{$\checkmark$}] Works with any input type
	\item [{$\checkmark$}] Provides context permission requests
	\item [{$\checkmark$}] Follows best practice for healthy code
\end{itemize}

\paragraph{Abstimmung auf physische Pick-Up Station}
Während dem gesamten Projekt wurde der Kontakt mit der Elektrotechnik gesucht und von Beginn an die Schnittstelle definiert. Die Kommunikation von der Informatik zur Elektrotechnik bei der Abholung einer Bestellung funktioniert. Auch kann in der entgegengesetzten Richtung eine Verbindung hergestellt werden, entsprechend wird das Inventar erstellt. 

\paragraph{Registrierung und Login}
Es ist dem Nutzer möglich, sich zu registrieren und ein Login durchzuführen. Die Umsetzung dieses Mechanismus wurde durch \ac{JWT} umgesetzt. \\
Damit der Nutzer sich nach der Registrierung anmelden kann, ist es zwingen nötig, die Altersverifikation erfolgreich abzuschliessen. Erst dann ist ein Login möglich. \\

\paragraph{Altersverifikation}
Die Altersverifikation war einer der zentralen Punkte bei diesem Projekt. Die Implementierung dient als Referenz für andere Projekte des Auftraggebers, bei denen eine Altersverifikation zum Einsatz kommt. \\
Die Umsetzung der ID-Verifikation verlieft Problemlos und war sehr intuitiv. Um die Altersverifikation durchzuführen, war ein erneuter API-Call nötig. Dass die Daten nicht direkt abgefragt werden könne ist eine Schwäche der Jumio-API. Die Altersverifikation wurde mit verschiedenen Testcases getestet, um die Zuverlässigkeit gewährleisten zu können. Es wurden alle bestanden. 

\paragraph{Bestellung}
Bei der Umsetzung eines Webshops ist es unumgänglich, dass auch Produkte bestellt werden können. Dem Kunden stehen im Shop verschiedenste Produkte zur Auswahl, abhängig von der ausgewählten Pick-Up Station, mit der entsprechenden Verfügbarkeit. In der installieren Version der \ac{PWA} werden die Packshots initial nicht korrekt angezeigt. \\Vom Auftraggeber wurde der Bezahldienst saferpay von Six Payment vorgegeben. Die Kommunikation mit der API war dabei kein Problem, jedoch führte \ac{CORS} zu Problemen. Es wurde eine Lösung gefunden, die Bezahlung funktioniert einwandfrei. Da es sich um einen Prototypen handelt, wird ein Testsystem verwendet. Die genauen Unterschiede werden in Kapitel \ref{testvslive} beschrieben. \\

\paragraph{Pick-Up Station}
Damit ein Nutzer eine Bestellung platzieren kann, muss er zwingend eine Station ausgewählt haben. Dem Nutzer werden auf der Karte alle Stationen angezeigt. Die Koordinaten, der Name und die Beschreibung werden initial vom Administrator gesetzt. Auf der Station besteht keine Möglichkeit, diese Daten einzugeben. \\Zu einer Station gehört ein Inventar von Produkten. Bei jeder Bestellung wird es angepasst. Dieses Vorgehen ist im finite State Machine \ref{img: stateMachine} dargestellt.\\
Die Anzeige von Pick-Up Station ist voll funktionsfähig, es sind keine Fehler bekannt. 

\paragraph{Aktueller Warenbestand}
Dem Administrator wird eine Möglichkeit angeboten, kritische Warenbestände zu bemerken. In einer Tabelle sieht er alle Inventuren, welche weniger als fünf Produkte umfassen. \\
Das Senden von Benachrichtigungen in diesem Fall wurde nur konzeptuell umgesetzt. Daher ist dieses Requirement nicht erfüllt, es wäre hier Nacharbeit zu leisten. 

\paragraph{Hinzufügen und Bearbeiten von Artikeln}
Für das Hinzufügen von Artikel ist keine graphische Oberfläche umgesetzt worden. Selbiges gilt für die Bearbeitung von Produkten.

\paragraph{Abholung}
Zur Abholung muss der Nutzer einen QR-Code auf der Station mit der \gls{PWA} einlesen. Die Verfügbaren Bestellungen werden in seinem Profil angezeigt, von hier aus kann zum QR-Code Scanner gewechselt werden. \\
Bei einer erfolgreichen Abholung kommen die Daten korrekt bei der Station an. Das Senden von den Daten über die \ac{UART}-Schnittstelle ist möglich. Die Daten werden korrekt an die Elektrotechnik via \ac{UART} übergeben und weiter verarbeitet. 

\subsubsection{Nicht funktionale Anforderungen}
\paragraph{Mit 5 Klicks bestellen}
Wird davon ausgegangen, dass der Nutzer bereits angemeldet ist. Zudem wird der Bezahlvorgang nicht dazugerechnet. \\
Die Anzahl Klicks kann, abhängig vom Anwendungsfall, sehr stark variieren. Im Durchschnitt benötigt der Nutzer zwischen 5-7 Klicks, um eine Bestellung platzieren zu können. Das Requirement darf als erreicht angesehen werden. 

\paragraph{Internationaler Einsatz}
Das UI wurde auf Deutsch gestaltet. Um International eingesetzt werden zu können, wäre es nötig, verschiedene Sprachen anzubieten.\\ Die Features haben keine Abhängigkeit zum Standort. Bei der Zeit wird im gesamten System die Zeitzone Europe/Zurich verwendet. Hier wäre es wünschenswert, dass bei der Anzeige im \ac{UI} die aktuelle Zeitzone des Benutzers genutzt wird. \\
In der momentanen Version ist die Applikation nur im deutschsprachigen Raum sinnvoll nutzbar. 

\paragraph{Kommunikation via HTTPS}
Die Kommunikation im gesamten System wird via \ac{HTTPS} durchgeführt. Die Verbindung zwischen Backend und Pick-Up Station ist ebenfalls durch \gls{zerotier} verschlüsselt. 

\paragraph{Authentifizierungsmechanismus}
\ac{JWT} gilt als de-facto Standard beim Sichern von API's. Die Tokens werden zusätzlich mit einem Private Key signiert, um sicherzustellen, dass es sich um einen Token von dieser API handelt. Ein Blacklisting von Token ist nicht umgesetzt worden. Der Nutzer wird bei längerer Inaktivität automatisch abgemeldet, der Token aus dem \gls{SessionStorage} entfernt. Die Gültigkeit des Tokens ist mit einer Stunde bewusst kurz gehalten. \\

\paragraph{Erweiterbarkeit}
Es wurden durchgängig die empfohlenen Vorgehensweisen der Frameworks genutzt. Durch die Dokumentation mittels \ac{UML}-Diagrammen von bestimmten Teilbereichen wird die Einarbeitungszeit verringert und das Verständnis erhöht. Im Projekt wurden die gängigsten Clean-Code Ansätze verwendet. 

\paragraph{CI/CD Pipeline}
Die beiden Hauptbestandteile besitzen eine CI/CD Pipeline. Die Pipeline übernimmt das Erstellen und die Verteilung des Docker Containers auf die produktive Umgebung. Dabei ist es sehr einfach möglich, ein Deployment in eine andere Umgebung durchzuführen. \\

Bei der Deployment-Stage tritt zuweilen das Problem auf, dass diese nicht erfolgreich durchläuft. Der Fehler ist sehr selten und nicht nachvollziehbar. Bei einem Retry von dieser Stage wird der Container erfolgreich auf die produktive Umgebung verteilt. 

\subsubsection{Übersicht}
\begin{itemize}
	\item [{$\checkmark$}] PWA-Checklist
	\item [{$\checkmark$}] Abstimmung auf physische Pick-Up Station
	\item [{$\checkmark$}] Registrierung und Login
	\item [{$\checkmark$}] Altersverifikation
	\item [{$\checkmark$}] Bestellung
	\item [{$\checkmark$}] Pick-Up Station
	\item [{$\times$}] Aktueller Warenbestand
	\item [{$\times$}] Hinzufügen und bearbeiten von Artikeln
	\item [{$\checkmark$}] Abholung\\\\
	\item [{$-$}] Mit 5 Klicks bestellen
	\item [{$\times$}] Internationaler Einsatz
	\item [{$\checkmark$}] Kommunikation via HTTPS
	\item [{$\checkmark$}] Authentifizierungsmechanismus
	\item [{$\checkmark$}] Erweiterbarkeit
	\item [{$\checkmark$}] CI/CD Pipeline
\end{itemize}
\subsection{Fazit}
Der Funktionalität und Zweckmässigkeit des neuen Absatzkanals kann durch diesen Prototypen von Informatikseite vollends bewiesen werden. Alle wesentlichen Punkte konnten umgesetzt und getestet werden. Die erforderlichen Anforderung konnten durch eine \gls{PWA} umgesetzt werden, es sind keine Einschränkungen vorhanden. Das Funktionsmuster ermöglicht eine vollständige Integration in die Projektpartner-Systeme. \\
Zusätzlich konnte der Einsatz vom Altersverifikationsanbieter Jumio für \ac{JTI} demonstriert und allfällige Schwächen identifiziert werden. Dieses Wissen kann für weitere Anwendungen genutzt werden. \\

\newpage