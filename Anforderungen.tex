% !TEX root = PREN2_Dokumentation.tex
\section{Anforderungen}\label{Anforderungen}
\subsection{Einführung}

\subsection{Einzelanforderungen}
\subsubsection{Funktionale Anforderungen}
\begin{table}[H]
    \setlength\extrarowheight{2pt} % for a bit of visual "breathing space"
    \begin{tabularx}{\textwidth}{|l|X|l|}
        \hline
        \textbf{ID} &  \textbf{Anforderung} & \textbf{Muss/Kann}  \\
        \hline
        F.1 & Die Web-Applikation muss die Punkte in der von Google aufgestellten Core Progressive Web App checklist \cite{pwaChecklist} erfüllen& Muss \\
        \hline
        F.2 & Die Web-Applikation ist auf eine phyisische Pick-Up Station abgestimmt. & Muss \\
        \hline
        F.3 & Die Web-Applikation ist international einsetzbar & Kann \\
        \hline

    \end{tabularx}
    \caption{ \label{tbl: Funktionsanforderungen}Anforderungen, Quelle: Autoren}
\end{table}
\subsubsection{Anforderungen an den Kaufprozess}
\begin{table}[H]
    \setlength\extrarowheight{2pt} % for a bit of visual "breathing space"
    \begin{tabularx}{\textwidth}{|l|X|l|}
    	\hline
        F.1 & Die Web-Applikation bietet die Möglichkeit, Kunden zu Erfassen & Muss \\
		\hline
		F.2 & Die Web-Applikation bietet die Möglichkeit, durch die Anbindung an eine 3rd Party eine Altersverifikation durchzuführen  & Muss \\
		\hline
		F.3 & Die Web-Applikation ermöglicht es, verschiedene Produkte zu bestellen & Kann \\
		\hline
		F.4 & Die Web-Applikation ermöglicht die Anbindung an einen bereits bekannten Bezahldienst, um eine sichere Bezahlung zu garantieren & Muss \\
		\hline
		F.5 & Die Web-Applikation ermöglicht es, verschiedene Produkte zu bestellen & Kann \\
		\hline
		F.6 & Dem Kunden kann eine Pick-Up Station zur Abholung auswählen, um die für Ihn nächstgelegene Station auswählen zu können  & Kann \\
		\hline
    \end{tabularx}
    \caption{ \label{tbl: Anforderungen an die Benutzbarkeit}Anforderungen, Quelle: Autoren}
\end{table}
\subsubsection{Anforderungen an die Leistungsfähigkeit}
\begin{table}[H]
    \setlength\extrarowheight{2pt} % for a bit of visual "breathing space"
    \begin{tabularx}{\textwidth}{|l|X|l|}
        \hline
        \textbf{ID} &  \textbf{Anforderung} & \textbf{Muss/Kann}  \\
        \hline
        L.1 & Das System muss jederzeit von mindestens 100 bis 120 Benutzern gleichzeitig verwendet werden können. & Muss \\
        \hline
 \end{tabularx}
    \caption{ \label{tbl: Anforderungen an die Leistungsfaehigkeit}Anforderungen, Quelle: Autoren}
\end{table}
\subsubsection{Nicht funktionale Anforderungen}
\begin{table}[H]
    \setlength\extrarowheight{2pt} % for a bit of visual "breathing space"
    \begin{tabularx}{\textwidth}{|l|X|l|}
        \hline
        \textbf{ID} &  \textbf{Anforderung} & \textbf{Muss/Kann}  \\
        \hline
        Q.1 & Als Benutzer muss ich mich bei vorhandener Internetverbindung jederzeit im System einloggen können. & Muss \\
        \hline
    \end{tabularx}
    \caption{ \label{tbl: Anforderungen an die Qualitaet}Anforderungen, Quelle: Autoren}
\end{table}
\newpage
