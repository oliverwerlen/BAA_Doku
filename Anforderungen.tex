% !TEX root = PREN2_Dokumentation.tex
\section{Anforderungen}\label{Anforderungen}
\subsection{Einführung}

\subsection{Einzelanforderungen}
\subsubsection{Funktionen}
\begin{table}[H]
    \setlength\extrarowheight{2pt} % for a bit of visual "breathing space"
    \begin{tabularx}{\textwidth}{|l|X|l|}
        \hline
        \textbf{ID} &  \textbf{Anforderung} & \textbf{Muss/Kann}  \\
        \hline
        F.1 & Als Benutzer kann ich gewünschte Übungen markieren um diese selektiv nochmals absolvieren zu können. & Muss \\
        \hline
        F.2 & Als Lehrer kann ich eine Prüfung mit selektierten Fragen erstellen. & Muss \\
        \hline
    \end{tabularx}
    \caption{ \label{tbl: Funktionsanforderungen}Anforderungen, Quelle: Autoren}
\end{table}
\subsubsection{Anforderungen an die Benutzbarkeit}
\begin{table}[H]
    \setlength\extrarowheight{2pt} % for a bit of visual "breathing space"
    \begin{tabularx}{\textwidth}{|l|X|l|}
        \hline
        \textbf{ID} &  \textbf{Anforderung} & \textbf{Muss/Kann}  \\
        \hline
        B.1 & Die Applikation soll ein modernes, innovatives und zeitgemässes User Interface bieten. & Ver \\
        \hline
    \end{tabularx}
    \caption{ \label{tbl: Anforderungen an die Benutzbarkeit}Anforderungen, Quelle: Autoren}
\end{table}
\subsubsection{Anforderungen an die Leistungsfähigkeit}
\begin{table}[H]
    \setlength\extrarowheight{2pt} % for a bit of visual "breathing space"
    \begin{tabularx}{\textwidth}{|l|X|l|}
        \hline
        \textbf{ID} &  \textbf{Anforderung} & \textbf{Muss/Kann}  \\
        \hline
        L.1 & Das System muss jederzeit von mindestens 100 bis 120 Benutzern gleichzeitig verwendet werden können. & Muss \\
        \hline
 \end{tabularx}
    \caption{ \label{tbl: Anforderungen an die Leistungsfaehigkeit}Anforderungen, Quelle: Autoren}
\end{table}
\subsubsection{Weitere Qualitätsmerkmale}
\begin{table}[H]
    \setlength\extrarowheight{2pt} % for a bit of visual "breathing space"
    \begin{tabularx}{\textwidth}{|l|X|l|}
        \hline
        \textbf{ID} &  \textbf{Anforderung} & \textbf{Muss/Kann}  \\
        \hline
        Q.1 & Als Benutzer muss ich mich bei vorhandener Internetverbindung jederzeit im System einloggen können. & Muss \\
        \hline
    \end{tabularx}
    \caption{ \label{tbl: Anforderungen an die Qualitaet}Anforderungen, Quelle: Autoren}
\end{table}
\subsubsection{Wartungs- und Supportinformationen}
\begin{table}[H]
    \setlength\extrarowheight{2pt} % for a bit of visual "breathing space"
    \begin{tabularx}{\textwidth}{|l|X|l|}
        \hline
        \textbf{ID} &  \textbf{Anforderung} & \textbf{Muss/Kann}  \\
        \hline
        S.1 & Die Applikation muss so dokumentiert werden, dass eine spätere Weiterentwicklung anhand dieser problemlos möglich ist. & Muss \\
        \hline
    \end{tabularx}
    \caption{ \label{tbl: Anforderungen an die Wartungen und den Support}Anforderungen, Quelle: Autoren}
\end{table}
\subsubsection{Weitere Anforderungen}
\begin{table}[H]
    \setlength\extrarowheight{2pt} % for a bit of visual "breathing space"
    \begin{tabularx}{\textwidth}{|l|X|l|}
        \hline
        \textbf{ID} &  \textbf{Anforderung} & \textbf{Muss/Kann}  \\
        \hline
        W.1 & Die Applikation soll in Form einer Webapplikation realisiert werden, die auf einem Webserver als eigenständige Applikation betrieben werden kann. & Muss \\
        \hline
        W.2 & In der Applikation muss das Logo der Hochschule Luzern sowie der Studiengang Gebäudetechnik ersichtlich sein. & Muss \\
        \hline
    \end{tabularx}
    \caption{ \label{tbl: Weitere Anforderungen}Anforderungen, Quelle: Autoren}
\end{table}
\newpage
