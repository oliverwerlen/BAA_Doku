\section{Sitzungsprotokolle}
Auf den nachfolgenden Seiten sind alle Protokolle von den durchgeführten Sitzungen ersichtlich.

\subsection{23.02.2021}
Kick Off Meeting
\subsubsection{Ordnungsaufruf}
Eine Besprechung aller Projektbeteiligten fand online als Zoom-Meeting am 23.02.2021 um 15:00 Uhr statt.
\subsubsection{Teilnehmer}
\begin{table}[H]
	\setlength\extrarowheight{2pt} % for a bit of visual "breathing space"
	\begin{tabularx}{\textwidth}{|X|X|}
		\hline
		\textbf{Anwesende Mitglieder} &  \textbf{Nicht anwesende Mitglieder} \\
		\hline
		Meier Rene, Betreuungsperson &  -\\
		Handschuh Michael, Betreuungsperson &  \\
		Oliver Werlen, Projektleiter &  \\
		Philipp Arnold, Elektrotechnik & \\
		Elena Nujic, Auftraggeber & \\
		Rolf Hohermut, Auftraggeber &\\
		\hline
	\end{tabularx}
	\caption{ \label{tbl: Teilnehmerliste vom 23.02.2021}Sitzungsprotokoll, Quelle: Autoren}
\end{table}
\subsubsection{Genehmigung des Protokolls}
Es handelt sich hierbei um die erste Sitzung in diesem Projekt. Es ist noch kein Protokoll vorhanden. 
\subsubsection{Ankündigungen}
Es handelt sich hierbei um das Kickoff Meeting. Als Ziel wird die Finalisierung der Aufgabenstellung genannt. 
\subsubsection{besprochene Punkte}
\paragraph{Aufgabenstellung}
	\begin{itemize}
		\item Register, Altersüberprüfung -> Bereits vorhanden, Shop (Brands, Produkt), Warenkorb, Bezahlung mit Kreditkarte, Beim Kauf definieren, wo abgeholt werden soll, reserviert in PickUp Station, An PickUp-> Mittels QR Code, Bestellung ausgeben
		\item Zahlung mit Twint, Kreditkarte, keine Nachnahme -> data trans, payrex
		\item Lösungen bereits in Onlineshop 
		\item Postautomat als Beispiel
		\item Automat muss wissen, welche Artikel er noch hat
		\item Preis muss Variabel sein, Gratis Paket
		\item Website als Informationsquelle
	 	\end{itemize}
\paragraph{Sitzungen}
	\begin{itemize}
		\item Abhängig von Projektphase, alle 2-3 Wochen
		\item Zwischenpräsentation von 20 Min, Resultate vorstellen, Fragen
		\item Schlusspräsentation 20-25 Minuten
	\end{itemize}
	
\paragraph{Fragen zur Dokumentation}
\begin{itemize}
	\item Benutzen von vorhandenen Texten aus WiPro?
	\item Auftrag Start WiPro auch in BAA?
\end{itemize}
\subsubsection{Tagesordnung der nächsten Sitzung}
\begin{itemize}
	\item Besprechung Anforderungen 
	\item Rahmenplan, erste Doku Teile
\end{itemize}


\subsubsection{Unterschriften}

Mit der Unterschrift gilt das Sitzungsprotokoll als bestätigt.\\

Ort, Datum:	\underline{\hspace*{6cm}}	\hspace*{1cm} Visum:	\underline{\hspace*{4cm}}
%\includegraphics[width=1\textwidth]{bilder/unterschriften}
\newpage


