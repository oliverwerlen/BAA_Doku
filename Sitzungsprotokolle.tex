\section{Sitzungsprotokolle}
Auf den nachfolgenden Seiten sind alle Protokolle von den durchgeführten Sitzungen ersichtlich.

\subsection{14.09.2020}
\subsubsection{Ordnungsaufruf}
Eine Besprechung aller Projektbeteiligten fand online als Zoom-Meeting am 14.09.2020 um 11:00 Uhr statt.
\subsubsection{Teilnehmer}
\begin{table}[H]
	\setlength\extrarowheight{2pt} % for a bit of visual "breathing space"
	\begin{tabularx}{\textwidth}{|X|X|}
		\hline
		\textbf{Anwesende Mitglieder} &  \textbf{Nicht anwesende Mitglieder} \\
		\hline
		Jordan Sucur, Betreuungsperson &  -\\
		Roger Buser, Auftraggeber &  \\
		Oliver Werlen, Projektleiter &  \\
		Frederico Fischer, Scrum Master &  \\
		\hline
	\end{tabularx}
	\caption{ \label{tbl: Teilnehmerliste vom 14.09.2020}Sitzungsprotokoll, Quelle: Autoren}
\end{table}
\subsubsection{Genehmigung des Protokolls}
Es handelt sich hierbei um die erste Sitzung in diesem Projekt. Es ist noch kein Protokoll vorhanden. 
\subsubsection{Ankündigungen}
Gemäss dem Zeitplan wird in zwei Wochen die Systemkontextphase abgeschlossen. Dabei erfolgt der erste Release eines GUI-Prototypen.
\subsubsection{besprochene Punkte}
\paragraph{Registrierung}
\begin{itemize}
    \item Lehrer bestimmt, wer im Kurs ist und kann Teilnehmer nachträglich bearbeiten
    \item Umsetzung ist dabei frei
\end{itemize}
\paragraph{Anforderungen}
\begin{itemize}
	
	\item Anforderungen müssen einheitlicher formuliert werden. Alle als Epics oder Userstories.
	\item Anforderungen müssen strukturiert werden, basierend auf dem Strukturplan.
    \item Kategorie für "weitere Anforderungen" definieren. 
    \item In der Software sollen Fragen nicht nur importiert, sondern auch exportiert werden. Format ist dabei frei.
    \item Als Sponsor soll die Hochschule Luzern inklusive dem Studiengang Gebäudetechnik gut sichtbar sein.
    \item Antworten sollen im Übungsmodus direkt auf Korrektheit überprüft werden können. Das Ergebnis ist direkt sichtbar. Sollte dies vom Schüler nicht gewünscht werden, kann eine Übungsprüfung absolviert werden.
    \item Anstatt eines Handbuchs wird ein interaktives Tutorial zur Verfügung gestellt. Ein Handbuch ist nicht nötig.
	\item Spezifizieren wie man Applikation erweitern müsste, falls doch mehr als 120 Nutzern vorhanden sind. 
    \item Applikation muss parallel nutzbar sein (als Anforderung spezifizieren).
    \item Quizes als Kann-Funktion erweitern
 
 \end{itemize}
\paragraph{Stand der Praxis in Dokumentation}
	\begin{itemize}
		\item Beschreiben, wieso alte Software nicht mehr Nutzbar ist.
		\item Was könnte man besser machen?
		\item Gibt es noch andere vergleichbare Applikationen auf dem Markt?
	\end{itemize}
\paragraph{Kapitelaufteilung in Dokumentation}
\begin{itemize}
	\item Projektmanagementplan so belassen, Änderungen notieren und jeweils auf den Anhang verweisen
\end{itemize}
\paragraph{Anderes}
\begin{itemize}
	\item Blockdiagramme: Datenfluss spezifizieren
\end{itemize}
\subsubsection{Tagesordnung der nächsten Sitzung}
\begin{itemize}
	\item Analyse GUI-Prototyp
	\item UML-Diagramme besprechen
	\item Besprechung CI/CD
\end{itemize}

\subsubsection{Unterschriften}

Mit der Unterschrift gilt das Sitzungsprotokoll als bestätigt.\\

Ort, Datum:	\underline{\hspace*{6cm}}	\hspace*{1cm} Visum:	\underline{\hspace*{4cm}}
%\includegraphics[width=1\textwidth]{bilder/unterschriften}
\newpage


