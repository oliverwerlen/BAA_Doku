\section{Sitzungsprotokolle}
Auf den nachfolgenden Seiten sind alle Protokolle von den durchgeführten Sitzungen ersichtlich.

\subsection{23.02.2021}\label{kickOff}
Kick Off Meeting
\subsubsection{Ordnungsaufruf}
Eine Besprechung aller Projektbeteiligten fand online als Zoom-Meeting am 23.02.2021 um 15:00 Uhr statt.
\subsubsection{Teilnehmer}
\begin{table}[H]
	\setlength\extrarowheight{2pt} % for a bit of visual "breathing space"
	\begin{tabularx}{\textwidth}{|X|X|}
		\hline
		\textbf{Anwesende Mitglieder} &  \textbf{Nicht anwesende Mitglieder} \\
		\hline
		Meier Rene, Betreuungsperson &  -\\
		Handschuh Michael, Betreuungsperson &  \\
		Oliver Werlen, Projektleiter &  \\
		Philipp Arnold, Elektrotechnik & \\
		Elena Nujic, Auftraggeber & \\
		Rolf Hohermut, Auftraggeber &\\
		\hline
	\end{tabularx}
	\caption{ \label{tbl: Teilnehmerliste vom 23.02.2021}Sitzungsprotokoll, Quelle: Autoren}
\end{table}
\subsubsection{Genehmigung des Protokolls}
Es handelt sich hierbei um die erste Sitzung in diesem Projekt. Es ist noch kein Protokoll vorhanden. 
\subsubsection{Ankündigungen}
Es handelt sich hierbei um das Kickoff Meeting. Als Ziel wird die Finalisierung der Aufgabenstellung genannt. 
\subsubsection{besprochene Punkte}
\paragraph{Aufgabenstellung}
	\begin{itemize}
		\item Register, Altersüberprüfung -> Bereits vorhanden, Shop (Brands, Produkt), Warenkorb, Bezahlung mit Kreditkarte, Beim Kauf definieren, wo abgeholt werden soll, reserviert in PickUp Station, An PickUp-> Mittels QR Code, Bestellung ausgeben
		\item Zahlung mit Twint, Kreditkarte, keine Nachnahme -> data trans, payrex
		\item Lösungen bereits in Onlineshop 
		\item Postautomat als Beispiel
		\item Automat muss wissen, welche Artikel er noch hat
		\item Preis muss Variabel sein, Gratis Paket
		\item Website als Informationsquelle
	 	\end{itemize}
\paragraph{Sitzungen}
	\begin{itemize}
		\item Abhängig von Projektphase, alle 2-3 Wochen
		\item Zwischenpräsentation von 20 Min, Resultate vorstellen, Fragen
		\item Schlusspräsentation 20-25 Minuten
	\end{itemize}
	
\paragraph{Fragen zur Dokumentation}
\begin{itemize}
	\item Benutzen von vorhandenen Texten aus WiPro?
	\item Auftrag Start WiPro auch in BAA?
\end{itemize}
\subsubsection{Tagesordnung der nächsten Sitzung}
\begin{itemize}
	\item Besprechung Anforderungen 
	\item Rahmenplan, erste Doku Teile
\end{itemize}


\subsubsection{Unterschriften}

Mit der Unterschrift gilt das Sitzungsprotokoll als bestätigt.\\

Ort, Datum:	\underline{\hspace*{6cm}}	\hspace*{1cm} Visum:	\underline{\hspace*{4cm}}
%\includegraphics[width=1\textwidth]{bilder/unterschriften}
\newpage


\subsection{04.03.2021}\label{04.03.2021}
\subsubsection{Ordnungsaufruf}
Eine Besprechung mit den Auftraggebern fand online als Microsoft-Teams-Meeting am 04.03.2021 um 15:00 Uhr statt.
\subsubsection{Teilnehmer}
\begin{table}[H]
	\setlength\extrarowheight{2pt} % for a bit of visual "breathing space"
	\begin{tabularx}{\textwidth}{|X|X|}
		\hline
		\textbf{Anwesende Mitglieder} &  \textbf{Nicht anwesende Mitglieder} \\
		\hline
		& Meier Rene, Betreuungsperson  \\
		& Handschuh Michael, Betreuungsperson   \\
		Oliver Werlen, Projektleiter &  \\
		Philipp Arnold, Elektrotechnik & \\
		Elena Nujic, Auftraggeber & \\
		Rolf Hohermut, Auftraggeber & \\
		\hline
	\end{tabularx}
	\caption{ \label{tbl: Teilnehmerliste vom 04.03.2021}Sitzungsprotokoll, Quelle: Autoren}
\end{table}
\subsubsection{Genehmigung des Protokolls}
Es handelt sich hierbei um die erste offiziell zu protokollierende Sitzung in diesem Projekt. Auf eine Genehmigung des Protokolls zum Kick-Off Meeting \ref{kickOff} wird daher verzichtet. 
\subsubsection{Ankündigungen}
In diesem Meeting werden offene Punkte besprochen, welche in der Inititialisierungsphase aufgetaucht sind. 
\subsubsection{besprochene Punkte}
\paragraph{Deployment}
\begin{itemize}
	\item Wo soll die Applikation laufen? Enterpriselab für Entwicklung ausreichend, evt. Deployment auf Umgebung von JTI
	\item Vorgabe vom Auftraggeber? 
\end{itemize}
\paragraph{vorhandene Anbieter}
\begin{itemize}
	\item Anbieter für Altersverifikation? Winston-Camel Registrieren -> Swisscom, Sunrise
	\item Neue Lösung mittels -> Jumio
	\item Anbieter für Bezahlvorgang? Datatrans, Payment von Six Payment 
	\item Kontakt zu Experten von JTI
\end{itemize}
\paragraph{Requirements}
\begin{itemize}
	\item Durchgehen, finalisieren, erweitern
	\item F.13 Im Kiosk im Aussenbereich -> Nachfüllanfrage an Kioskbetreiber -> Muss Features
	\item Ändern und hinzufügen von Produkten
\end{itemize}
\paragraph{Probleme}
	\begin{itemize}
	\item Probleme bei aktuellem Absatzkanal -> Wichtigsten Punkte gefunden, AVEC mit Ihrem Produkt reinnehmen. 
	\end{itemize}

\paragraph{Kommunikation der Pick-Up Station}
	\begin{itemize}
	\item Internet und Strom vorhanden
	\item Einlesen des Abholcodes mit RFID
\end{itemize}

\subsubsection{Tagesordnung der nächsten Sitzung}
Die nächste Sitzung wird in zwei Wochen angesetzt. 

\subsubsection{Unterschriften}

Mit der Unterschrift gilt das Sitzungsprotokoll als bestätigt.\\

Ort, Datum:	\underline{\hspace*{6cm}}	\hspace*{1cm} Visum:	\underline{\hspace*{4cm}}
%\includegraphics[width=1\textwidth]{bilder/unterschriften}


\subsection{11.03.2021}\label{Beteuermeeting1}
\subsubsection{Ordnungsaufruf}
Eine Besprechung mit der Betreuungsperson fand online als Zoom-Meeting am 04.03.2021 um 13:00 Uhr statt.
\subsubsection{Teilnehmer}
\begin{table}[H]
	\setlength\extrarowheight{2pt} % for a bit of visual "breathing space"
	\begin{tabularx}{\textwidth}{|X|X|}
		\hline
		\textbf{Anwesende Mitglieder} &  \textbf{Nicht anwesende Mitglieder} \\
		\hline
		& Meier Rene, Betreuungsperson  \\
		Handschuh Michael, Betreuungsperson &   \\
		&Oliver Werlen, Projektleiter \\
		&Philipp Arnold, Elektrotechnik \\
		&Elena Nujic, Auftraggeber \\
		&Rolf Hohermut, Auftraggeber \\
		\hline
	\end{tabularx}
	\caption{ \label{tbl: Teilnehmerliste vom 04.03.2021}Sitzungsprotokoll, Quelle: Autoren}
\end{table}
\subsubsection{Genehmigung des Protokolls}
Es handelt sich hierbei um die erste offiziell zu protokollierende Sitzung in diesem Projekt. Auf eine Genehmigung des Protokolls zum Kick-Off Meeting \ref{kickOff} wird daher verzichtet. 
\subsubsection{Ankündigungen}
In diesem Meeting werden offene Punkte besprochen, welche in der Inititialisierungsphase aufgetaucht sind. 
\subsubsection{besprochene Punkte}
\paragraph{Deployment}
\begin{itemize}
	\item Wo soll die Applikation laufen? Enterpriselab für Entwicklung ausreichend, evt. Deployment auf Umgebung von JTI
	\item Vorgabe vom Auftraggeber? 
\end{itemize}
\paragraph{Allgemeine Punkte}
\begin{itemize}
	\item Schnittstelle spezifizieren mit ET und Maschinen evt. als Meilenstein
	\item Meilenstein Zwischenpräsentation
	\item Meilenstein Abschlusspräsentation
	\item Wieso NFC und nicht QR-Code
	\item Datenschutz bei jumio -> 
	\item Schnittstelle 3rd Party Systeme in Risikoanalyse
	\item Datenschutz bei 3rd Party
	\item Doku im github Michael Handschuh freigeben
\end{itemize}

\subsubsection{Tagesordnung der nächsten Sitzung}
Die nächste Sitzung mit dem Betreuer wird in 2 Wochen stattfinden. 
\subsubsection{Unterschriften}

Mit der Unterschrift gilt das Sitzungsprotokoll als bestätigt.\\

Ort, Datum:	\underline{\hspace*{6cm}}	\hspace*{1cm} Visum:	\underline{\hspace*{4cm}}
%\includegraphics[width=1\textwidth]{bilder/unterschriften}


\subsection{18.03.2021}
\subsubsection{Ordnungsaufruf}
Eine Besprechung mit den Auftraggebern fand online als Microsoft-Teams-Meeting am 18.03.2021 um 13:00 Uhr statt.
\subsubsection{Teilnehmer}
\begin{table}[H]
	\setlength\extrarowheight{2pt} % for a bit of visual "breathing space"
	\begin{tabularx}{\textwidth}{|X|X|}
		\hline
		\textbf{Anwesende Mitglieder} &  \textbf{Nicht anwesende Mitglieder} \\
		\hline
		& Meier Rene, Betreuungsperson  \\
		& Handschuh Michael, Betreuungsperson   \\
		Oliver Werlen, Projektleiter &  \\
		Philipp Arnold, Elektrotechnik & \\
		Elena Nujic, Auftraggeber & \\
		Rolf Hohermut, Auftraggeber & \\
		\hline
	\end{tabularx}
	\caption{ \label{tbl: Teilnehmerliste vom 18.03.2021}Sitzungsprotokoll, Quelle: Autoren}
\end{table}
\subsubsection{Genehmigung des Protokolls}
Das Protokoll der letzten Sitzung wurde von allen Anwesenden bestätigt
\subsubsection{Ankündigungen}
In diesem Meeting wird der Abschluss der Inititialisierungsphase besprochen. Zusätzlich werden die offenen Punkte zu \ref{SpezifikationSchnittstelleAbholung} besprochen und finalisiert. 
\subsubsection{besprochene Punkte}
\paragraph{Spezifikation Schnittstelle Abholung}
	\begin{itemize}
		\item Problem mit NFC/Bluetooth
		\item Vorstellen der Alternativen
		\item Besprechung Alternativen -> Alternative 1 wird bevorzugt, QR-Code auf Maschine 
		\item Wenn keine Bestellung, dann auf Webpage
		\item Produkte, Liste von Produkten wird gesendet, jtiproducts.ch sind alle Produkte vorhanden
		\item Keine Policy von JTI vorhanden, jtiproducts als Referenz
		\item Statusmeldungen an Nachfüller
	\end{itemize}

\paragraph{Schnittstelle Produktbestand}
	\begin{itemize}
		\item Finalisierung mit Philipp Arnold individuell
		\item Übertragung mittels JSON Format
	\end{itemize}

\subsubsection{Tagesordnung der nächsten Sitzung}
Die nächste Sitzung mit den Auftraggebern findet in zwei Wochen statt. 
\subsubsection{Unterschriften}

Mit der Unterschrift gilt das Sitzungsprotokoll als bestätigt.\\

Ort, Datum:	\underline{\hspace*{6cm}}	\hspace*{1cm} Visum:	\underline{\hspace*{4cm}}
%\includegraphics[width=1\textwidth]{bilder/unterschriften}