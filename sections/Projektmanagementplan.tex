% !TEX root = PREN2_Dokumentation.tex
\documentclass[12pt,titlepage]{article}

%Packages
\usepackage[ngerman]{babel}
\usepackage[utf8]{inputenc}
\usepackage{cite}
\usepackage{color}
\usepackage[a4paper,lmargin={3cm},rmargin={3cm},
tmargin={2.5cm},bmargin = {2.5cm}]{geometry}
\usepackage{amssymb}
\usepackage{amsthm}
\usepackage{graphicx}
\usepackage[printonlyused, withpage]{acronym}
\usepackage{abstract}
\usepackage{tabularx,ragged2e}
\usepackage{float}
\restylefloat{table}
\newcolumntype{C}{>{\Centering\arraybackslash}X}

\begin{document}

\section{Projektmanagementplan}
\subsection{Projektorganisation}
\subsubsection{Organisationsplan, Rollen, Zust\"andigkeiten}\\
\begin{figure}[htbp]
    \centering
    \includegraphics[width=1\textwidth]{bilder/Organigramm.png}
    \caption[Organigramm]{Organigramm\\ Quelle: Autoren}
    \label{img: OrganigrammWiPro}
\end{figure}

\subsubsection{Projektstrukturplan}
\subsection{Projektf\"uhrung}
\subsubsection{Rahmenplan}

Im untenstehenden Rahmenplan wird mittels Zeitplan eine Grobplanung dargestellt. 

\begin{figure}[htbp]
    \centering
    \includegraphics[width=1\textwidth]{bilder/SoDa_Zeitstrahl.png}
    \caption[SoDa Rahmenplan]{Rahmenplan,\\ Quelle: Autoren}
    \label{img: SoDa Rahmenplan}
\end{figure}

\subsubsection{Meilensteine}
Wie in Abbildung \ref{img: SoDa Rahmenplan} zu sehen gibt es insgesamt sieben Meilensteine. Diese werden in folgender Tabelle beschrieben sowie die nötigen Deliverables aufgezeigt. 

\begin{table}[H]
\setlength\extrarowheight{2pt} % for a bit of visual "breathing space"
\begin{tabularx}{\textwidth}{|C|C|C|}
\hline
\textbf{Meilenstein} &  \textbf{Beschreibung} & \textbf{Deliverables}  \\

\hline
Projektstart & Bei Meilenstein eins wird das Kickoff-Meeting mit allen Projektteilnehmern durchgeführt. & finale Aufgabenstellung\\

\hline
Start Umsetzung & Bei Meilenstein zwei wird vom klassischen Projektmanagement zum agilen Projektmanagement übergegangen. Dazu muss die Initialisierungsphase abgeschlossen sein& Projektmanagementplan, Systemspezifikation, Anforderungsliste\\

\hline
Abschluss Systemkontext & Zu diesem Zeitpunkt ist alles bereit, um mit der Entwicklung zu beginnen. Es wurden bereits erste GUI Entwürfe erarbeitet sowie die Systemarchitektur definiert.  & CI/CD Umgebung eingerichtet, GUI-Prototyp, UML-Diagramme  \\

\hline
Abschluss Schülermodus & Der Entwicklung des Schülermodus ist abgeschlossen. Es können vorgegebene Fragen beantwortet werden sowie eine Statistik zu bisher Gelerntem eingesehen werden.  & Testprotokolle zu Schülermodus, Demo Schülermodus, Release Schülermodus  \\

\hline
Abschluss Lehrermodus & Der Entwicklung des Lehrermodus ist abgeschlossen. Es können Fragen und Prüfungen erstellt und verteilt werden. Zusätzlich sind die Fragen aus der alten Applikation integriert. & Testprotokolle zu Lehrermodus, Testprotokolle Prüfungsmodus, Integration alte Daten, Demo verschiedene Modis, Release Lehrermodus  \\

\hline
Start Einführung & Der Auftraggeber erhält eine Einführung in die Software & Sitzungsprotokoll zum Ende der Einführungsphase  \\

\hline
Projektende & Der Auftraggeber erhält eine Einführung in die Software & Fertige Projektdokumentation, Abgeschlossene Testprotokolle   \\
\hline
\end{tabularx}
\caption{ \label{tbl: Meilensteine}Meilensteine, Quelle: Autoren}
\end{table}


\subsubsection{Projektkontrolle}
In der Regel Verweise auf Burndown-Charts in der Entwicklungsumgebung und Sprintreview-Protokolle im Anhang.
\subsubsection{Risikomanagement}
\subsubsection{Definition of done}
In jedem Sprint müssen die nachfolgenden Punkte zwingend erreicht werden, um ein potenziell auslieferbares Produkt zu erhalten:

\begin{itemize}
\item Review durchgeführt
\item Akzeptanzkriterien erfüllt
\item Unit Tests Grün
\item CI/CD ohne Fehler
\item keine kritischen Bugs
\item Clean Code Guidelines eingehalten
\item Dokumentation aktuell
\end{itemize}
\subsection{Projektunterst\"utzung}
\subsubsection{Tools f\"ur Entwicklung, Test und Abnahme}
\paragraph{Entwicklungstools}
Bei der Entwicklung des Projekts kommen folgende Programme zum Einsatz: 

\begin{table}[H]
\setlength\extrarowheight{2pt} % for a bit of visual "breathing space"
\begin{tabularx}{\textwidth}{|C|C|C|}
\hline
\textbf{Typ} &\textbf{Tool} & \textbf{Version}  \\

\hline
IDE & Intelij Ultimate  & 2020.1\\
\hline
IDE & Visual Studio Code & 1.48.2\\
\hline
Versionsverwaltung & Git & 2.27.0\\
\hline
\end{tabularx}
\caption{ \label{tbl: Entwicklungstools}Entwicklungstools, Quelle: Autoren}
\end{table}
\paragraph{Testtools}
<tbd>
Beim Testing kommen folgende Tools zum Einsatz

\begin{table}[H]
\setlength\extrarowheight{2pt} % for a bit of visual "breathing space"
\begin{tabularx}{\textwidth}{|C|C|C|}
\hline
\textbf{Typ} &\textbf{Tool} & \textbf{Version}  \\

\hline
IDE & Intelij Ultimate  & 2020.1\\
\hline
\end{tabularx}
\caption{ \label{tbl: Testtools}Testtools, Quelle: Autoren}
\end{table}
\subsubsection{Konfigurationsmanagement}
\subsection{Teststrategie und Drehbuch}
Bei der Teststrategie wird auf den IEE 829 Standard gesetzt. 
\subsubsection{Teststrategie}
Es wird bei diesem Projekt hauptsächlich auf Automated Testing gesetzt. Unit Tests werden dabei Integration Tests vorgezogen. Hierzu wird auf das bewährte JUnit Framework gesetzt. Es wird dabei das Test-First-Prinzip verwendet. 

\paragraph{Automated Testing der REST-Schnittstelle}
Zum Testen der Rest-Schnittstelle wird Unirest sowie JUnit verwendet.  


\subsubsection{Testdrehbuch}
Wie oben genannt wird hautpsächlich auf Automated Testing gesetzt. Daher werden nur sehr wenige manuelle Tests durchgeführt. Nachfolgend werden diese beschrieben. 
\
\begin{landscape}
\begin{table}

\begin{tabularx}{\textwidth}{|C|}
  \hline
  \multicolumn{2}{|c|}{Test Prüfungsmodus} \\
  \hline
  Test Nr. & 1\\
  Beschreibung & Durch diesen Test wird das Durchführen einer Prüfung durch eine Lehrperson und dessen Auswertung getestet\\
  Randbedingungen & Es wird ein bereits eingerichteter Lehreraccount mit einer ihm zugewiesenen Klasse von 10 Schülern sowie eine vordefinierte Prüfung zur Verfügung gestellt \\
  erwartete Resultate & Der Lehrer kann die Prüfung den Lernenden freischalten. Nach vorgegebener Zeit wird die Prüfung geschlossen. Im Lehreraccount sind die einzelnen Resultate aufgelistet. Diese können einfach pro Schüler exportiert werden.  \\
  Testperson & tbd \\
  Datum & tbd \\
  Unterschrift & tbd \\
   \hline
\end{tabularx}
\caption{ \label{tbl: Test Pruefungsmodus}Test Prüfungsmodus , Quelle: Autoren}
\end{table}

\newpage


\end{document}