% !TEX root = PREN2_Dokumentation.tex
\section{Testprotokolle}\label{Testprotokolle}

\subsection{Testprotokolle Bestellung}\label{testprotokolleBestellung}
\begin{table}[H]
    \setlength\extrarowheight{2pt} % for a bit of visual "breathing space"
    \begin{tabularx}{\textwidth}{|l|X|}
        \hline
        \textbf{Test Nr.} & 1 \\
        \hline
        \textbf{Beschreibung} & Durch diesen Test wird die Registrierung eines neuen Kunden getestet. \\
        \hline
        \textbf{Randbedingungen} &
        \begin{minipage}[t]{0.6\textwidth}
            \begin{itemize}
                \item Die Testperson hat sich mit ihrer Email-Adresse und Benutzernamen noch nie registriert.
                \item Die Testperson nutzt den Nutzernamen \glqq test\grqq{} , die Email-Adresse \glqq test@gmail.com\grqq{} und das Passwort \glqq ABC*1234\grqq{} ein.\\
            \end{itemize}
        \end{minipage} \\
        \hline
        \textbf{erwartete Resultate}  &
        \begin{minipage}[t]{0.6\textwidth}
            \begin{itemize}
                \item Das Konto der Testperson wird korrekt angelegt. 
                \item Ein Popup zeigt dem Nutzer den Status an.\\
            \end{itemize}
        \end{minipage} \\
        \hline
        \textbf{Testperson} & Oliver Werlen \\
        \hline
        \textbf{Datum} & 18.04.2021 \\
        \hline
        \textbf{Durchführung} &
        \begin{minipage}[t]{0.6\textwidth}
            \begin{enumerate}
                \item Die Testperson gibt die angegebenen Daten ein. Die restlichen Daten werden zufällig gewählt.
                \item Die Testperson klickt auf den Button "Registrieren".
                \item Es wird in einem Dialog die Erstellung des Nutzers dargestellt. 
                \item Es wird automatisch zur Login Page gewechselt.\\
            \end{enumerate}
        \end{minipage} \\
        \hline
        \textbf{erhaltenes Resultat} &
        \begin{minipage}[t]{0.6\textwidth}
            \begin{itemize}
                \item Das Erstellen des Nutzers wurde vom System via Pop-Up bestätigt. \\
            \end{itemize}
        \end{minipage} \\
        \hline
        \textbf{Test bestanden} & Ja \\
        \hline
    \end{tabularx}
    \caption{ \label{tbl: testprotokoll1}Testprotokoll Test 1, Quelle: Autor}
\end{table}
\newpage
\begin{table}[H]
	\setlength\extrarowheight{2pt} % for a bit of visual "breathing space"
	\begin{tabularx}{\textwidth}{|l|X|}
		\hline
		\textbf{Test Nr.} & 2\\
		\hline
		\textbf{Beschreibung} & Durch diesen Test wird die Registrierung eines neuen Kunden getestet, wobei die Email Adresse und der Nutzername bereits genutzt werden. \\
		\hline
		\textbf{Randbedingungen} &
		\begin{minipage}[t]{0.6\textwidth}
			\begin{itemize}
				\item Der Test \ref{tbl: testprotokoll1} ist erfolgreich durchgeführt worden.
				\item Die Testperson nutzt den Nutzernamen test , die Email-Adresse test@gmail.com und das Passwort ABC*1234 ein.\\
			\end{itemize}
		\end{minipage} \\
		\hline
		\textbf{erwartete Resultate}  &
		\begin{minipage}[t]{0.6\textwidth}
			\begin{itemize}
				\item Das System gibt dem Nutzer die Antwort, dass ein Benutzer mit dieser Email oder Benutzernamen bereits existiert. \\
			\end{itemize}
		\end{minipage} \\
		\hline
		\textbf{Testperson} & Oliver Werlen \\
		\hline
		\textbf{Datum} & 18.04.2021 \\
		\hline
		\textbf{Durchführung} &
		\begin{minipage}[t]{0.6\textwidth}
			\begin{enumerate}
				\item Die Testperson gibt die angegebenen Daten ein. Die restlichen Daten werden zufällig gewählt.
				\item Die Testperson klickt auf den Button "Registrieren".
				\item Es wird in einem Dialog mit der Meldung "Benutzer mit dieser Email oder Benutzernamen existiert bereits". \\
			\end{enumerate}
		\end{minipage} \\
		\hline
		\textbf{erhaltenes Resultat} &
		\begin{minipage}[t]{0.6\textwidth}
			\begin{itemize}
				\item Das Popup wird wie geplant angezeigt. \\
			\end{itemize}
		\end{minipage} \\
		\hline
		\textbf{Test bestanden} & Ja \\
		\hline
	\end{tabularx}
	\caption{ \label{tbl: testprotokoll2}Testprotokoll Test 2, Quelle: Autor}
\end{table}

\begin{table}[H]
	\setlength\extrarowheight{2pt} % for a bit of visual "breathing space"
	\begin{tabularx}{\textwidth}{|l|X|}
		\hline
		\textbf{Test Nr.} & 3\\
		\hline
		\textbf{Beschreibung} & Durch diesen Test wird die Login-Funktion getestet.  \\
		\hline
		\textbf{Randbedingungen} &
		\begin{minipage}[t]{0.6\textwidth}
			\begin{itemize}
				\item Der Test \ref{tbl: testprotokoll1} ist erfolgreich durchgeführt worden.
				\item Die Testperson nutzt den Benutzernamen test und das Passwort ABC*1234.\\
			\end{itemize}
		\end{minipage} \\
		\hline
		\textbf{erwartete Resultate}  &
		\begin{minipage}[t]{0.6\textwidth}
			\begin{itemize}
				\item Eine Meldung wird angezeigt, mit welcher das erfolgreiche Erstellen des Nutzers bestätigt wird.
				\item Im Sidenav wird der Logout Button angezeigt. \\
			\end{itemize}
		\end{minipage} \\
		\hline
		\textbf{Testperson} & Oliver Werlen \\
		\hline
		\textbf{Datum} & 18.04.2021 \\
		\hline
		\textbf{Durchführung} &
		\begin{minipage}[t]{0.6\textwidth}
			\begin{enumerate}
				\item Die Testperson gibt die angegebenen Daten ein.
				\item Der Benutzer klickt auf den Login Button.\\
			\end{enumerate}
		\end{minipage} \\
		\hline
		\textbf{erhaltenes Resultat} &
		\begin{minipage}[t]{0.6\textwidth}
			\begin{itemize}
				\item Das Popup wird wie geplant angezeigt.
				\item Das Sidenav wird entsprechend angepasst.\\  
			\end{itemize}
		\end{minipage} \\
		\hline
		\textbf{Test bestanden} & Ja \\
		\hline
	\end{tabularx}
	\caption{ \label{tbl: testprotokoll3}Testprotokoll Test 3, Quelle: Autor}
\end{table}
\begin{table}[H]
	\setlength\extrarowheight{2pt} % for a bit of visual "breathing space"
	\begin{tabularx}{\textwidth}{|l|X|}
		\hline
		\textbf{Test Nr.} & 4\\
		\hline
		\textbf{Beschreibung} & Durch diesen Test wird die Login-Funktion getestet. Es wird eine falsche Kombination eingetragen.  \\
		\hline
		\textbf{Randbedingungen} &
		\begin{minipage}[t]{0.6\textwidth}
			\begin{itemize}
				\item Die Testperson nutzt den Benutzernamen \glqq testNoAccess\grqq{} und das Passwort \glqq ABC*1234*noAccess\grqq{}. \\
			\end{itemize}
		\end{minipage} \\
		\hline
		\textbf{erwartete Resultate}  &
		\begin{minipage}[t]{0.6\textwidth}
			\begin{itemize}
				\item Eine Meldung Falscher Benutzername oder Passwort wird angezeigt\\
			\end{itemize}
		\end{minipage} \\
		\hline
		\textbf{Testperson} & Oliver Werlen \\
		\hline
		\textbf{Datum} & 18.04.2021 \\
		\hline
		\textbf{Durchführung} &
		\begin{minipage}[t]{0.6\textwidth}
			\begin{enumerate}
				\item Die Testperson gibt die angegebenen Daten ein.
				\item Der Benutzer klickt auf den Login Button\\
			\end{enumerate}
		\end{minipage} \\
		\hline
		\textbf{erhaltenes Resultat} &
		\begin{minipage}[t]{0.6\textwidth}
			\begin{itemize}
				\item Das Popup wird wie geplant angezeigt. \\
			\end{itemize}
		\end{minipage} \\
		\hline
		\textbf{Test bestanden} & Ja \\
		\hline
	\end{tabularx}
	\caption{ \label{tbl: testprotokoll4}Testprotokoll Test 4, Quelle: Autor}
\end{table}
\newpage
\begin{table}[H]
	\setlength\extrarowheight{2pt} % for a bit of visual "breathing space"
	\begin{tabularx}{\textwidth}{|l|X|}
		\hline
		\textbf{Test Nr.} & 5\\
		\hline
		\textbf{Beschreibung} & Durch diesen Test wird die Logout Funktion getestet.  \\
		\hline
		\textbf{Randbedingungen} &
		\begin{minipage}[t]{0.6\textwidth}
			\begin{itemize}
				\item Die Testperson ist erfolgreich eingeloggt. \\
			\end{itemize}
		\end{minipage} \\
		\hline
		\textbf{erwartete Resultate}  &
		\begin{minipage}[t]{0.6\textwidth}
			\begin{itemize}
				\item Popup mit der Meldung erfolgreich ausgeloggt.\\
			\end{itemize}
		\end{minipage} \\
		\hline
		\textbf{Testperson} & Oliver Werlen \\
		\hline
		\textbf{Datum} & 18.04.2021 \\
		\hline
		\textbf{Durchführung} &
		\begin{minipage}[t]{0.6\textwidth}
			\begin{enumerate}
				\item Die Testperson klickt auf den Logout Button.\\
			\end{enumerate}
		\end{minipage} \\
		\hline
		\textbf{erhaltenes Resultat} &
		\begin{minipage}[t]{0.6\textwidth}
			\begin{itemize}
				\item Das Popup wird wie geplant angezeigt. \\
			\end{itemize}
		\end{minipage} \\
		\hline
		\textbf{Test bestanden} & Ja \\
		\hline
	\end{tabularx}
	\caption{ \label{tbl: testprotokoll5}Testprotokoll Test 5, Quelle: Autor}
\end{table}
\begin{table}[H]
	\setlength\extrarowheight{2pt} % for a bit of visual "breathing space"
	\begin{tabularx}{\textwidth}{|l|X|}
		\hline
		\textbf{Test Nr.} & 6\\
		\hline
		\textbf{Beschreibung} & Durch diesen Test wird das Hinzufügen von Produkten in den Warenkorb getestet.  \\
		\hline
		\textbf{Randbedingungen} &
		\begin{minipage}[t]{0.6\textwidth}
			\begin{itemize}
				\item Es sind Artikel im Shop vorhanden \\
			\end{itemize}
		\end{minipage} \\
		\hline
		\textbf{erwartete Resultate}  &
		\begin{minipage}[t]{0.6\textwidth}
			\begin{itemize}
				\item Der Artikel ist im Warenkorb vorhanden.\\
			\end{itemize}
		\end{minipage} \\
		\hline
		\textbf{Testperson} & Oliver Werlen \\
		\hline
		\textbf{Datum} & 18.04.2021 \\
		\hline
		\textbf{Durchführung} &
		\begin{minipage}[t]{0.6\textwidth}
			\begin{enumerate}
				\item Die Testperson befindet sich auf der Produktübersicht.
				\item Die Testperson klickt bei einem zufälligen Produkt auf das Warenkorbsymbol unten rechts.
				\item Die Person klickt ein anderes Produkt an.
				\item Die Testperson klickt auf den Button \glqq in den Warenkorb\grqq{}.
				\item Die Testperson wechselt zum Warenkorb (Item oben rechts).\\
			\end{enumerate}
		\end{minipage} \\
		\hline
		\textbf{erhaltenes Resultat} &
		\begin{minipage}[t]{0.6\textwidth}
			\begin{itemize}
				\item Die beiden Produkte sind im Warenkorb zu finden.
				\item Die beiden Produkte befinden sich in einfacher Ausführung im Warenkorb.
				\item Das Total ist korrekt summiert worden. \\
			\end{itemize}
		\end{minipage} \\
		\hline
		\textbf{Test bestanden} & Ja \\
		\hline
	\end{tabularx}
	\caption{ \label{tbl: testprotokoll6}Testprotokoll Test 6, Quelle: Autor}
\end{table}
\begin{table}[H]
	\setlength\extrarowheight{2pt} % for a bit of visual "breathing space"
	\begin{tabularx}{\textwidth}{|l|X|}
		\hline
		\textbf{Test Nr.} & 7\\
		\hline
		\textbf{Beschreibung} & Durch diesen Test wird die Persistierung des Warenkorbs getestet.  \\
		\hline
		\textbf{Randbedingungen} &
		\begin{minipage}[t]{0.6\textwidth}
			\begin{itemize}
				\item Der Test \ref{tbl: testprotokoll6} ist erfolgreich abgeschlossen worden. \\
			\end{itemize}
		\end{minipage} \\
		\hline
		\textbf{erwartete Resultate}  &
		\begin{minipage}[t]{0.6\textwidth}
			\begin{itemize}
				\item Die Artikel sind auch nach dem Verlassen der Website und einem erneuten Aufruf immer noch vorhanden. \\
			\end{itemize}
		\end{minipage} \\
		\hline
		\textbf{Testperson} & Oliver Werlen \\
		\hline
		\textbf{Datum} & 18.04.2021 \\
		\hline
		\textbf{Durchführung} &
		\begin{minipage}[t]{0.6\textwidth}
			\begin{enumerate}
				\item Die Testperson befindet sich im Warenkorb.
				\item Die Testperson schliesst das Browserfenster.
				\item Die Testperson öffnet das Browserfenster wieder und navigiert zum Warenkorb.
				\item Der Warenkorbinhalt bleibt bestehen.\\
			\end{enumerate}
		\end{minipage} \\
		\hline
		\textbf{erhaltenes Resultat} &
		\begin{minipage}[t]{0.6\textwidth}
			\begin{itemize}
				\item Die Produkte sind immer noch im Warenkorb.\\
			\end{itemize}
		\end{minipage} \\
		\hline
		\textbf{Test bestanden} & Ja \\
		\hline
	\end{tabularx}
	\caption{ \label{tbl: testprotokoll7}Testprotokoll Test 7, Quelle: Autor}
\end{table}
\begin{table}[H]
	\setlength\extrarowheight{2pt} % for a bit of visual "breathing space"
	\begin{tabularx}{\textwidth}{|l|X|}
		\hline
		\textbf{Test Nr.} & 8\\
		\hline
		\textbf{Beschreibung} & Durch diesen Test werden die Funktionen des Warenkorbs getestet.  \\
		\hline
		\textbf{Randbedingungen} &
		\begin{minipage}[t]{0.6\textwidth}
			\begin{itemize}
				\item Der Test \ref{tbl: testprotokoll6} ist erfolgreich abgeschlossen worden. \\
			\end{itemize}
		\end{minipage} \\
		\hline
		\textbf{erwartete Resultate}  &
		\begin{minipage}[t]{0.6\textwidth}
			\begin{itemize}
				\item Die Artikelanzahl wird erhöht. 
				\item Das Gesamttotal wird erhöht.
				\item Die Artikelanzahl wird reduziert.
				\item Beim Erreichen von 0 wird der Artikel entfernt. \\
			\end{itemize}
		\end{minipage} \\
		\hline
		\textbf{Testperson} & Oliver Werlen \\
		\hline
		\textbf{Datum} & 18.04.2021 \\
		\hline
		\textbf{Durchführung} &
		\begin{minipage}[t]{0.6\textwidth}
			\begin{enumerate}
				\item Die Testperson befindet sich im Warenkorb.
				\item Die Testperson erhöht die Artikelanzahl auf 4.
				\item Die Testperson reduziert die Artikelanzahl auf 0.\\
				\end{enumerate}
		\end{minipage} \\
		\hline
		\textbf{erhaltenes Resultat} &
		\begin{minipage}[t]{0.6\textwidth}
			\begin{itemize}
				\item Es befinden sich keine Artikel mehr im Warenkorb.
				\item Es wurde ein Popup mit einer entsprechenden Meldung angezeigt.  \\
			\end{itemize}
		\end{minipage} \\
		\hline
		\textbf{Test bestanden} & Ja \\
		\hline
	\end{tabularx}
	\caption{ \label{tbl: testprotokoll8}Testprotokoll Test 8, Quelle: Autor}
\end{table}
\begin{table}[H]
	\setlength\extrarowheight{2pt} % for a bit of visual "breathing space"
	\begin{tabularx}{\textwidth}{|l|X|}
		\hline
		\textbf{Test Nr.} & 9\\
		\hline
		\textbf{Beschreibung} & Durch diesen Test wird die Bezahlfunktion getestet.  \\
		\hline
		\textbf{Randbedingungen} &
		\begin{minipage}[t]{0.6\textwidth}
			\begin{itemize}
				\item Der Benutzer ist erfolgreich eingeloggt. 
				\item Der Benutzer hat Produkte im Warenkorb. \\
			\end{itemize}
		\end{minipage} \\
		\hline
		\textbf{erwartete Resultate}  &
		\begin{minipage}[t]{0.6\textwidth}
			\begin{itemize}
				\item Der Benutzer wird auf die Bezahlseite weitergeleitet. 
				\item Das Gesamttotal wird bei der Bezahlung korrekt angezeigt. 
				\item Der Benutzer wird nach dem Bezahlabschluss auf eine entsprechende Seite weitergeleitet. \\
			\end{itemize}
		\end{minipage} \\
		\hline
		\textbf{Testperson} & Oliver Werlen \\
		\hline
		\textbf{Datum} & 18.04.2021 \\
		\hline
		\textbf{Durchführung} &
		\begin{minipage}[t]{0.6\textwidth}
			\begin{enumerate}
				\item Die Testperson befindet sich im Warenkorb.
				\item Die Testperson klickt auf checkout.
				\item Die Testperson wird auf die Bezahlseite umgeleitet. 
				\item Die Testperson klickt durch den Bezahlprozess.\\
			\end{enumerate}
			\end{minipage} \\
			\hline
			\textbf{erhaltenes Resultat} &
			\begin{minipage}[t]{0.6\textwidth}
				\begin{itemize}
					\item Der Testperson wird eine Meldung zur Bezahlbestätigung ausgegeben. \\
				\end{itemize}
			\end{minipage} \\
			\hline
			\textbf{Test bestanden} & Ja \\
			\hline
		\end{tabularx}
		\caption{ \label{tbl: testprotokoll9}Testprotokoll Test 9, Quelle: Autor}
	\end{table}
\begin{table}[H]
	\setlength\extrarowheight{2pt} % for a bit of visual "breathing space"
	\begin{tabularx}{\textwidth}{|l|X|}
		\hline
		\textbf{Test Nr.} & 10\\
		\hline
		\textbf{Beschreibung} & Durch diesen Test wird die Bezahlfunktion bei einem Abbruch getestet.  \\
		\hline
		\textbf{Randbedingungen} &
		\begin{minipage}[t]{0.6\textwidth}
			\begin{itemize}
				\item Der Benutzer ist erfolgreich eingeloggt. 
				\item Der Benutzer hat Produkte im Warenkorb. \\
			\end{itemize}
		\end{minipage} \\
		\hline
		\textbf{erwartete Resultate}  &
		\begin{minipage}[t]{0.6\textwidth}
			\begin{itemize}
				\item Der Benutzer wird auf die Bezahlseite weitergeleitet. 
				\item Das Gesamttotal wird bei der Bezahlung korrekt ausgegeben. 
				\item Der Benutzer bricht den Bezahlvorgang mittels Cancel Button ab. 
				\item Der Benutzer wird auf eine entsprechende Seite weitergeleitet. \\
			\end{itemize}
		\end{minipage} \\
		\hline
		\textbf{Testperson} & Oliver Werlen \\
		\hline
		\textbf{Datum} & 18.04.2021 \\
		\hline
		\textbf{Durchführung} &
		\begin{minipage}[t]{0.6\textwidth}
			\begin{enumerate}
				\item Die Testperson befindet sich im Warenkorb.
				\item Die Testperson klickt auf checkout.
				\item Die Testperson wird auf die Bezahlseite umgeleitet. 
				\item Die Testperson klickt beim Bezahlvorgang auf den cancel Button. \\
			\end{enumerate}
		\end{minipage} \\
			\hline
			\textbf{erhaltenes Resultat} &
			\begin{minipage}[t]{0.6\textwidth}
				\begin{itemize}
					\item Der Testperson wird auf eine Page not found Seite umgeleitet. \\
				\end{itemize}
			\end{minipage} \\
			\hline
			\textbf{Test bestanden} & Ja \\
			\hline
		\end{tabularx}
		\caption{ \label{tbl: testprotokoll10}Testprotokoll Test 10, Quelle: Autor}
	\end{table}
\newpage
\subsection{Testprotokolle Abschluss Realisierungsphase}\label{testprotokolleRealisierungsphase}
Es werden die Features getestet, welche in \ref{testprotokolleBestellung} noch nicht getestet wurden. 
\subsubsection{Altersverifikation}\label{altersverifikation}
\begin{table}[H]
	\setlength\extrarowheight{2pt} % for a bit of visual "breathing space"
	\begin{tabularx}{\textwidth}{|l|X|}
		\hline
		\textbf{Test Nr.} & 11\\
		\hline
		\textbf{Beschreibung} & Durch diesen Test wird die Altersverifikation getestet.  \\
		\hline
		\textbf{Randbedingungen} &
		\begin{minipage}[t]{0.6\textwidth}
			\begin{itemize}
				\item Der Benutzer hat erfolgreich die Registrierungsdaten eingegeben und befindet sich am Beginn der Altersverifikation. 
				\item Die Person ist über 18 Jahre alt. \\
			\end{itemize}
		\end{minipage} \\
		\hline
		\textbf{erwartete Resultate}  &
		\begin{minipage}[t]{0.6\textwidth}
			\begin{itemize}
				\item Der Benutzer muss seinen Ausweis einlesen, 
				\item Der Benutzer wird nach dem Abschluss der Verifikation auf die Zielseite weitergeleitet. 
				\item Der Benutzer wird nach einer Minute über das Resultat seiner Verifikation informiert. 
				\item Der Benutzer kann sich mit Email und Passwort einloggen. \\
			\end{itemize}
		\end{minipage} \\
		\hline
		\textbf{Testperson} & Oliver Werlen \\
		\hline
		\textbf{Datum} & 25.05.2021 \\
		\hline
		\textbf{Durchführung} &
		\begin{minipage}[t]{0.6\textwidth}
			\begin{enumerate}
				\item Die Testperson hat die Registrierung abgeschlossen. 
				\item Die Testperson befindet sich auf dem Jumio-Startbildschirm. 
				\item Die Testperson lädt ein Foto von der Ausweis Vorder- und Rückseite hoch.
				\item Die Testperson scannt ihr Gesicht ein. 
				\item Die Testperson wird auf die Zielseite weitergeleitet. 
				\item Die Testperson wird nach einer Minute über das Resultat seiner Verifikation informiert.
				\item Die Testperson kann sich mit seinen Daten einloggen. \\
			\end{enumerate}
		\end{minipage} \\
		\hline
		\textbf{erhaltenes Resultat} &
		\begin{minipage}[t]{0.6\textwidth}
			\begin{itemize}
				\item Das erwartete Resultat ist eingetreten.  \\
			\end{itemize}
		\end{minipage} \\
		\hline
		\textbf{Test bestanden} & Ja \\
		\hline
	\end{tabularx}
	\caption{ \label{tbl: testprotokoll11}Testprotokoll Test 11, Quelle: Autor}
\end{table}

\begin{table}[H]
	\setlength\extrarowheight{2pt} % for a bit of visual "breathing space"
	\begin{tabularx}{\textwidth}{|l|X|}
		\hline
		\textbf{Test Nr.} & 12\\
		\hline
		\textbf{Beschreibung} & Durch diesen Test wird die Altersverifikation getestet, wenn die Person nicht 18 Jahre alt ist.  \\
		\hline
		\textbf{Randbedingungen} &
		\begin{minipage}[t]{0.6\textwidth}
			\begin{itemize}
				\item Der Benutzer hat erfolgreich die Registrierungsdaten eingegeben und befindet sich am Beginn der Altersverifikation. 
				\item Die Person ist unter 18 Jahre alt. \\
			\end{itemize}
		\end{minipage} \\
		\hline
		\textbf{erwartete Resultate}  &
		\begin{minipage}[t]{0.6\textwidth}
			\begin{itemize}
				\item Der Benutzer muss seinen Ausweis einlesen.
				\item Der Benutzer wird nach dem Abschluss der Verifikation auf die Zielseite weitergeleitet. 
				\item Der Benutzer wird nach einer Minute über das Resultat seiner Verifikation informiert. Die Meldung sagt aus, das der Benutzer zu jung ist. 
				\item Der Benutzer kann sich nicht mit Email und Passwort einloggen. \\
			\end{itemize}
		\end{minipage} \\
		\hline
		\textbf{Testperson} & Elena Nujic \\
		\hline
		\textbf{Datum} & 31.05.2021 \\
		\hline
		\textbf{Durchführung} &
		\begin{minipage}[t]{0.6\textwidth}
			\begin{enumerate}
				\item Die Testperson hat die Registrierung abgeschlossen. 
				\item Die Testperson befindet sich auf dem Jumio-Startbildschirm. 
				\item Die Testperson lädt ein Foto von der Ausweis Vorder- und Rückseite hoch.
				\item Die Testperson scannt ihr Gesicht ein. 
				\item Die Testperson wird auf die Zielseite der Altersverifikation weitergeleitet. 
				\item Der Benutzer wird nach einer Minute über das Resultat seiner Verifikation informiert. Die Meldung sagt aus, das der Benutzer zu jung ist. 
				\item Die Testperson kann sich nicht mit seinen Daten einloggen. \\
			\end{enumerate}
		\end{minipage} \\
		\hline
		\textbf{erhaltenes Resultat} &
		\begin{minipage}[t]{0.6\textwidth}
			\begin{itemize}
				\item Das erwartete Resultat ist eingetreten.  \\
			\end{itemize}
		\end{minipage} \\
		\hline
		\textbf{Test bestanden} & Ja \\
		\hline
	\end{tabularx}
	\caption{ \label{tbl: testprotokoll12}Testprotokoll Test 12, Quelle: Autor}
\end{table}
\begin{table}[H]
	\setlength\extrarowheight{2pt} % for a bit of visual "breathing space"
	\begin{tabularx}{\textwidth}{|l|X|}
		\hline
		\textbf{Test Nr.} & 13\\
		\hline
		\textbf{Beschreibung} & Durch diesen Test wird die Altersverifikation getestet, wenn die Person einen Ausweis einer anderen Person benutzt.   \\
		\hline
		\textbf{Randbedingungen} &
		\begin{minipage}[t]{0.6\textwidth}
			\begin{itemize}
				\item Der Benutzer hat erfolgreich die Registrierungsdaten eingegeben und befindet sich am Beginn der Altersverifikation. 
				\item Die Person besitzt einen Ausweis von einer mindestens 18 Jahre alten Person, die nicht sie selbst ist. \\
			\end{itemize}
		\end{minipage} \\
		\hline
		\textbf{erwartete Resultate}  &
		\begin{minipage}[t]{0.6\textwidth}
			\begin{itemize}
				\item Der Benutzer muss den Ausweis einlesen.
				\item Die Testperson wird auf die Zielseite weitergeleitet. 
				\item Die Testperson wird nach einer Minute über das Resultat seiner Verifikation informiert.
				\item Der Benutzer kann sich nicht mit Email und Passwort einloggen. \\
			\end{itemize}
		\end{minipage} \\
		\hline
		\textbf{Testperson} & Oliver Werlen \\
		\hline
		\textbf{Datum} & 31.05.2021 \\
		\hline
		\textbf{Durchführung} &
		\begin{minipage}[t]{0.6\textwidth}
			\begin{enumerate}
				\item Die Testperson befindet sich auf dem Jumio-Startbildschirm. 
				\item Die Testperson lädt ein Foto von der Ausweis Vorder- und Rückseite hoch.
				\item Die Testperson scannt ihr Gesicht ein. 
				\item Die Testperson wird auf die Zielseite der 
				\item Die Testperson kann sich nicht mit seinen Daten einloggen. \\
			\end{enumerate}
		\end{minipage} \\
		\hline
		\textbf{erhaltenes Resultat} &
		\begin{minipage}[t]{0.6\textwidth}
			\begin{itemize}
				\item Das erwartete Resultat ist eingetreten.  \\
			\end{itemize}
		\end{minipage} \\
		\hline
		\textbf{Test bestanden} & Ja \\
		\hline
	\end{tabularx}
	\caption{ \label{tbl: testprotokoll13}Testprotokoll Test 13, Quelle: Autor}
\end{table}

\subsubsection{Station und Kartenfunktionalität}
\begin{table}[H]
	\setlength\extrarowheight{2pt} % for a bit of visual "breathing space"
	\begin{tabularx}{\textwidth}{|l|X|}
		\hline
		\textbf{Test Nr.} & 14\\
		\hline
		\textbf{Beschreibung} & Durch diesen Test wird die Anzeige einer Station und die Produktverfügbarkeit an einer Station getestet. \\
		\hline
		\textbf{Randbedingungen} &
		\begin{minipage}[t]{0.6\textwidth}
			\begin{itemize}
				\item Es ist eine Station mit Artikeln verfügbar. \\
			\end{itemize}
		\end{minipage} \\
		\hline
		\textbf{erwartete Resultate}  &
		\begin{minipage}[t]{0.6\textwidth}
			\begin{itemize}
				\item Dem Benutzer wird die Station auf der Karte angezeigt
				\item Bei einem Klick auf die Station erscheint ein Popup mit den Produktverfügbarkeiten. 
				\item Bei einem Klick auf Station auswählen wird die Station ausgewählt und auf die Produktseite gewechselt. 
				\item Auf der Produktseite wird die Verfügbarkeit ebenfalls angezeigt. \\
			\end{itemize}
		\end{minipage} \\
		\hline
		\textbf{Testperson} & Oliver Werlen \\
		\hline
		\textbf{Datum} & 31.05.2021 \\
		\hline
		\textbf{Durchführung} &
		\begin{minipage}[t]{0.6\textwidth}
			\begin{enumerate}
				\item Die Testperson wählt  Nächste Station auswählen
				\item Die Testperson wählt eine Station aus. Das Popup erscheint. 
				\item Die Testperson klickt auf Station auswählen.
				\item Die Testperson wird auf die Produktseite weitergeleitet. Die Verfügbarkeiten werden angezeigt. 
				\item Die Testperson klickt auf ein Produkt. Auch hier wird die Verfügbarkeit identisch angezeigt. \\
			\end{enumerate}
		\end{minipage} \\
		\hline
		\textbf{erhaltenes Resultat} &
		\begin{minipage}[t]{0.6\textwidth}
			\begin{itemize}
				\item Das erwartete Resultat ist eingetreten.  \\
			\end{itemize}
		\end{minipage} \\
		\hline
		\textbf{Test bestanden} & Ja \\
		\hline
	\end{tabularx}
	\caption{ \label{tbl: testprotokoll14}Testprotokoll Test 14, Quelle: Autor}
\end{table}
\begin{table}[H]
	\setlength\extrarowheight{2pt} % for a bit of visual "breathing space"
	\begin{tabularx}{\textwidth}{|l|X|}
		\hline
		\textbf{Test Nr.} & 15\\
		\hline
		\textbf{Beschreibung} & Durch diesen Test wird die Kartenanzeige bei ausgeschalteten System-Preferences getestet.  \\
		\hline
		\textbf{Randbedingungen} &
		\begin{minipage}[t]{0.6\textwidth}
			\begin{itemize}
				\item In den Systempreferences ist die Standortberechtigung für den Browser deaktiviert. 
			\end{itemize}
		\end{minipage} \\
		\hline
		\textbf{erwartete Resultate}  &
		\begin{minipage}[t]{0.6\textwidth}
			\begin{itemize}
				\item Dem Benutzer wird die Karte mit einer Default-Initial-Position angezeigt. \\
			\end{itemize}
		\end{minipage} \\
		\hline
		\textbf{Testperson} & Oliver Werlen \\
		\hline
		\textbf{Datum} & 31.05.2021 \\
		\hline
		\textbf{Durchführung} &
		\begin{minipage}[t]{0.6\textwidth}
			\begin{enumerate}
				\item Die Testperson deaktiviert den Zugriff auf den Standort für den gewählten Browser.
				\item Der Testperson wird die Karte angezeigt. Der Fokus liegt im Wallis. \\
			\end{enumerate}
		\end{minipage} \\
		\hline
		\textbf{erhaltenes Resultat} &
		\begin{minipage}[t]{0.6\textwidth}
			\begin{itemize}
				\item Das erwartete Resultat ist eingetreten.  \\
			\end{itemize}
		\end{minipage} \\
		\hline
		\textbf{Test bestanden} & Ja \\
		\hline
	\end{tabularx}
	\caption{ \label{tbl: testprotokoll15}Testprotokoll Test 15, Quelle: Autor}
\end{table}

\begin{table}[H]
	\setlength\extrarowheight{2pt} % for a bit of visual "breathing space"
	\begin{tabularx}{\textwidth}{|l|X|}
		\hline
		\textbf{Test Nr.} & 16\\
		\hline
		\textbf{Beschreibung} & Durch diesen Test wird die Kartenanzeige bei blockiertem Standortzugriff im Browser getestet.  \\
		\hline
		\textbf{Randbedingungen} &
		\begin{minipage}[t]{0.6\textwidth}
			\begin{itemize}
				\item Es sind keine Randbedingungen bekannt. 
			\end{itemize}
		\end{minipage} \\
		\hline
		\textbf{erwartete Resultate}  &
		\begin{minipage}[t]{0.6\textwidth}
			\begin{itemize}
				\item Dem Benutzer wird die Karte mit einer Default-Initial-Position angezeigt. \\
			\end{itemize}
		\end{minipage} \\
		\hline
		\textbf{Testperson} & Oliver Werlen \\
		\hline
		\textbf{Datum} & 31.05.2021 \\
		\hline
		\textbf{Durchführung} &
		\begin{minipage}[t]{0.6\textwidth}
			\begin{enumerate}
				\item Die Testperson wählt im Popup zur Berechtigung auf den Standort Block.
			\end{enumerate}
		\end{minipage} \\
		\hline
		\textbf{erhaltenes Resultat} &
		\begin{minipage}[t]{0.6\textwidth}
			\begin{itemize}
				\item Das erwartete Resultat ist eingetreten.  \\
			\end{itemize}
		\end{minipage} \\
		\hline
		\textbf{Test bestanden} & Ja \\
		\hline
	\end{tabularx}
	\caption{ \label{tbl: testprotokoll16}Testprotokoll Test 15, Quelle: Autor}
\end{table}
\begin{table}[H]
	\setlength\extrarowheight{2pt} % for a bit of visual "breathing space"
	\begin{tabularx}{\textwidth}{|l|X|}
		\hline
		\textbf{Test Nr.} & 17\\
		\hline
		\textbf{Beschreibung} & Durch diesen Test wird die Produktverfügbarkeit ohne ausgewählte Station getestet. \\
		\hline
		\textbf{Randbedingungen} &
		\begin{minipage}[t]{0.6\textwidth}
			\begin{itemize}
				\item Der Benutzer öffnet eine neue Instanz der App ohne gecachete Werte. 
			\end{itemize}
		\end{minipage} \\
		\hline
		\textbf{erwartete Resultate}  &
		\begin{minipage}[t]{0.6\textwidth}
			\begin{itemize}
				\item Alle Produktverfügbarkeiten werden rot markiert. 
			\end{itemize}
		\end{minipage} \\
		\hline
		\textbf{Testperson} & Oliver Werlen \\
		\hline
		\textbf{Datum} & 31.05.2021 \\
		\hline
		\textbf{Durchführung} &
		\begin{minipage}[t]{0.6\textwidth}
			\begin{enumerate}
				\item Die Testperson klickt direkt nach dem Starten der App auf die Produktansicht. 
			\end{enumerate}
		\end{minipage} \\
		\hline
		\textbf{erhaltenes Resultat} &
		\begin{minipage}[t]{0.6\textwidth}
			\begin{itemize}
				\item Die Produktverfügbarkeit ist bei jedem Produkt rot.  
				\item Die Produkte können nicht dem Warenkorb hinzugefügt werden. \\
			\end{itemize}
		\end{minipage} \\
		\hline
		\textbf{Test bestanden} & Ja \\
		\hline
	\end{tabularx}
	\caption{ \label{tbl: testprotokoll17}Testprotokoll Test 17, Quelle: Autor}
\end{table}
\begin{table}[H]
	\setlength\extrarowheight{2pt} % for a bit of visual "breathing space"
	\begin{tabularx}{\textwidth}{|l|X|}
		\hline
		\textbf{Test Nr.} & 18\\
		\hline
		\textbf{Beschreibung} & Durch diesen Test wird der Bestellabschluss ohne aktive Station oder ohne Produkte getestet.  \\
		\hline
		\textbf{Randbedingungen} &
		\begin{minipage}[t]{0.6\textwidth}
			\begin{itemize}
				\item Der Benutzer öffnet eine neue Instanz der App ohne gecachete Werte. 
				\item Vorgängig werden beliebige Produkte dem Warenkorb hinzugefügt. 
			\end{itemize}
		\end{minipage} \\
		\hline
		\textbf{erwartete Resultate}  &
		\begin{minipage}[t]{0.6\textwidth}
			\begin{itemize}
				\item Ohne aktive Station wird eine entsprechende Meldung ausgegeben. 
				\item Ohne Produkte wird eine entsprechende Meldung ausgeben. 
			\end{itemize}
		\end{minipage} \\
		\hline
		\textbf{Testperson} & Oliver Werlen \\
		\hline
		\textbf{Datum} & 31.05.2021 \\
		\hline
		\textbf{Durchführung} &
		\begin{minipage}[t]{0.6\textwidth}
			\begin{enumerate}
				\item Die Testperson klickt direkt auf den Warenkorb und den Checkout-Button. Eine entsprechende Meldung wird ausgegeben. 
				\item Die Testperson löscht alle Artikel aus dem Warenkorb. Er wählt eine Station aus. Beim versuchten Checkout wird eine entsprechende Meldung ausgegeben. 
			\end{enumerate}
		\end{minipage} \\
		\hline
		\textbf{erhaltenes Resultat} &
		\begin{minipage}[t]{0.6\textwidth}
			\begin{itemize}
				\item Die entsprechenden Meldungen wurden ausgegeben.
			\end{itemize}
		\end{minipage} \\
		\hline
		\textbf{Test bestanden} & Ja \\
		\hline
	\end{tabularx}
	\caption{ \label{tbl: testprotokoll18}Testprotokoll Test 18, Quelle: Autor}
\end{table}
\subsubsection{Inventur}\label{inventur}
\begin{table}[H]
	\setlength\extrarowheight{2pt} % for a bit of visual "breathing space"
	\begin{tabularx}{\textwidth}{|l|X|}
		\hline
		\textbf{Test Nr.} & 19\\
		\hline
		\textbf{Beschreibung} & Durch diesen Test wird die Überprüfung der entsprechenden Verfügbarkeiten im Warenkorb getestet.  \\
		\hline
		\textbf{Randbedingungen} &
		\begin{minipage}[t]{0.6\textwidth}
			\begin{itemize}
				\item Der Benutzer besitzt einen aktivierten Account.  
				\item Die Station ist mit 10 verfügbaren Produkten befüllt. 
				\item Der Benutzer hat eine Station ausgewählt. \\
			\end{itemize}
		\end{minipage} \\
		\hline
		\textbf{erwartete Resultate}  &
		\begin{minipage}[t]{0.6\textwidth}
			\begin{itemize}
				\item Die Testperson kann maximal 10 Produkte dem Warenkorb hinzufügen. \\
			\end{itemize}
		\end{minipage} \\
		\hline
		\textbf{Testperson} & Oliver Werlen \\
		\hline
		\textbf{Datum} & 31.05.2021 \\
		\hline
		\textbf{Durchführung} &
		\begin{minipage}[t]{0.6\textwidth}
			\begin{enumerate}
				\item Die Testperson erhöht im Warenkorb die Anzahl. 
			\end{enumerate}
		\end{minipage} \\
		\hline
		\textbf{erhaltenes Resultat} &
		\begin{minipage}[t]{0.6\textwidth}
			\begin{itemize}
				\item Bei 10 Produkten wird eine entsprechende Meldung ausgegeben. Ein weiteres Hinzufügen von diesem Produkt ist nicht möglich. 
			\end{itemize}
		\end{minipage} \\
		\hline
		\textbf{Test bestanden} & Ja \\
		\hline
	\end{tabularx}
	\caption{ \label{tbl: testprotokoll19}Testprotokoll Test 19, Quelle: Autor}
\end{table}
\begin{table}[H]
	\setlength\extrarowheight{2pt} % for a bit of visual "breathing space"
	\begin{tabularx}{\textwidth}{|l|X|}
		\hline
		\textbf{Test Nr.} & 20\\
		\hline
		\textbf{Beschreibung} & Durch diesen Test wird die Aktualisierung des Warenbestandes getestet. \\
		\hline
		\textbf{Randbedingungen} &
		\begin{minipage}[t]{0.6\textwidth}
			\begin{itemize}
				\item Der Testfall 18 ist abgeschlossen worden und 10 Produkte von einer Sorte bestellt worden. \\
			\end{itemize}
		\end{minipage} \\
		\hline
		\textbf{erwartete Resultate}  &
		\begin{minipage}[t]{0.6\textwidth}
			\begin{itemize}
				\item Das Inventar auf der Karte, bzw. der Produktübersicht ist rot markiert. Der Artikel kann an dieser Station nicht mehr bestellt werden.\\
			\end{itemize}
		\end{minipage} \\
		\hline
		\textbf{Testperson} & Oliver Werlen \\
		\hline
		\textbf{Datum} & 31.05.2021 \\
		\hline
		\textbf{Durchführung} &
		\begin{minipage}[t]{0.6\textwidth}
			\begin{enumerate}
				\item Die Testperson wählt die Station aus. 
			\end{enumerate}
		\end{minipage} \\
		\hline
		\textbf{erhaltenes Resultat} &
		\begin{minipage}[t]{0.6\textwidth}
			\begin{itemize}
				\item Das vorhin bestellte Produkt ist nicht mehr verfügbar und wird entsprechend markiert. \\
			\end{itemize}
		\end{minipage} \\
		\hline
		\textbf{Test bestanden} & Ja \\
		\hline
	\end{tabularx}
	\caption{ \label{tbl: testprotokoll20}Testprotokoll Test 20, Quelle: Autor}
\end{table}
\begin{table}[H]
	\setlength\extrarowheight{2pt} % for a bit of visual "breathing space"
	\begin{tabularx}{\textwidth}{|l|X|}
		\hline
		\textbf{Test Nr.} &21\\
		\hline
		\textbf{Beschreibung} & Durch diesen Test wird die Aktualisierung des Warenbestandes bei einem Abbruch des Bezahlvorgangs getestet. \\
		\hline
		\textbf{Randbedingungen} &
		\begin{minipage}[t]{0.6\textwidth}
			\begin{itemize}
				\item Es wird ein Produkt gewählt, welches verfügbar ist. 
				\item Der Testfall wird lokal ausgeführt. Das Backend läuft in der IDE
				\item Vom gewählten Artikel sind 10 Stück verfügbar.\\
			\end{itemize}
		\end{minipage} \\
		\hline
		\textbf{erwartete Resultate}  &
		\begin{minipage}[t]{0.6\textwidth}
			\begin{itemize}
				\item Nach dem Klick auf den Checkout-Button wird die Quantity in der Datenbank aktualisiert. 
				\item Nach dem Abbrechen des Bezahlvorgangs befindet sich dieser Wert wieder beim Startwert. \\
			\end{itemize}
		\end{minipage} \\
		\hline
		\textbf{Testperson} & Oliver Werlen \\
		\hline
		\textbf{Datum} & 31.05.2021 \\
		\hline
		\textbf{Durchführung} &
		\begin{minipage}[t]{0.6\textwidth}
			\begin{enumerate}
				\item Die Testperson bestellt 5 Stück einen Artikel und klickt auf Checkout.
				\item In der Datenbank wird der entsprechende Eintrag gesucht. Er wurde aktualisiert. 
				\item Die Bezahlung wird abgebrochen. 
				\item n der Datenbank wird der entsprechende Eintrag gesucht. Er befindet sich wieder beim Startwert. \\
			\end{enumerate}
		\end{minipage} \\
		\hline
		\textbf{erhaltenes Resultat} &
		\begin{minipage}[t]{0.6\textwidth}
			\begin{itemize}
				\item Das erwartete Resultat ist eingetreten. \\
			\end{itemize}
		\end{minipage} \\
		\hline
		\textbf{Test bestanden} & Ja \\
		\hline
	\end{tabularx}
	\caption{ \label{tbl: testprotokoll21}Testprotokoll Test 21, Quelle: Autor}
\end{table}

\subsubsection{Initialisierung neue Station}
\begin{table}[H]
	\setlength\extrarowheight{2pt} % for a bit of visual "breathing space"
	\begin{tabularx}{\textwidth}{|l|X|}
		\hline
		\textbf{Test Nr.} & 22\\
		\hline
		\textbf{Beschreibung} & Durch diesen Test wird die Initialisierung einer neuen Pick-Up Station von einem Administrator getestet. \\
		\hline
		\textbf{Randbedingungen} &
		\begin{minipage}[t]{0.6\textwidth}
			\begin{itemize}
				\item Es ist noch keine Station initialisiert. 
				\item Die Station ist mit dem Zerotier-Netzwerk verbunden. \\
			\end{itemize}
		\end{minipage} \\
		\hline
		\textbf{erwartete Resultate}  &
		\begin{minipage}[t]{0.6\textwidth}
			\begin{itemize}
				\item Die neue Station wird angezeigt. 
				\item Die erforderlichen Parameter der Station können gesetzt werden. 
				\item Die Station wird als initialisiert markiert. \\
			\end{itemize}
		\end{minipage} \\
		\hline
		\textbf{Testperson} & Oliver Werlen \\
		\hline
		\textbf{Datum} & 31.05.2021 \\
		\hline
		\textbf{Durchführung} &
		\begin{minipage}[t]{0.6\textwidth}
			\begin{enumerate}
				\item Der Administrator startet das entsprechende Script auf dem Raspberry Pi. 
				\item Der Administrator loggt sich in der WebApp ein. 
				\item Der Administrator wechselt zum Administrator-Tab.
				\item Der Administrator wählt die Station unter neue Stations aus. 
				\item Der Administrator gibt die gewünschten Werte ein und klickt auf Initialisieren. \\
			\end{enumerate}
		\end{minipage} \\
		\hline
		\textbf{erhaltenes Resultat} &
		\begin{minipage}[t]{0.6\textwidth}
			\begin{itemize}
				\item Das erwartete Resultat ist eingetreten. \\
			\end{itemize}
		\end{minipage} \\
		\hline
		\textbf{Test bestanden} & Ja \\
		\hline
	\end{tabularx}
	\caption{ \label{tbl: testprotokoll22}Testprotokoll Test 22, Quelle: Autor}
\end{table}
\subsubsection{Abholung einer Bestellung}
\begin{table}[H]
	\setlength\extrarowheight{2pt} % for a bit of visual "breathing space"
	\begin{tabularx}{\textwidth}{|l|X|}
		\hline
		\textbf{Test Nr.} & 23\\
		\hline
		\textbf{Beschreibung} & Durch diesen Test wird die Abholung einer Bestellung überprüft. \\
		\hline
		\textbf{Randbedingungen} &
		\begin{minipage}[t]{0.6\textwidth}
			\begin{itemize}
				\item Es ist eine Station verfügbar. 
				\item Der Nutzer hat bereits eine Bestellung an dieser Station platziert.
				\item Die Bestellung ist bezahlt. 
			\end{itemize}
		\end{minipage} \\
		\hline
		\textbf{erwartete Resultate}  &
		\begin{minipage}[t]{0.6\textwidth}
			\begin{itemize}
				\item Es wird der Bestellung bereit zur Entnahme-Bildschirm angezeigt. 
			\end{itemize}
		\end{minipage} \\
		\hline
		\textbf{Testperson} & Oliver Werlen \\
		\hline
		\textbf{Datum} & 31.05.2021 \\
		\hline
		\textbf{Durchführung} &
		\begin{minipage}[t]{0.6\textwidth}
			\begin{enumerate}
				\item Der Benutzer wechselt zu seinem Profil.
				\item Der Benutzer sieht die offene Bestellung. 
				\item Der Benutzer klickt auf PickUp
				\item Der Benutzer scannt mit der Kamera den QR-Code der Station ein.
				\item Der entsprechende Bildschirm wird angezeigt. \\
			\end{enumerate}
		\end{minipage} \\
		\hline
		\textbf{erhaltenes Resultat} &
		\begin{minipage}[t]{0.6\textwidth}
			\begin{itemize}
				\item Das erwartete Resultat ist eingetreten. \\
			\end{itemize}
		\end{minipage} \\
		\hline
		\textbf{Test bestanden} & Ja \\
		\hline
	\end{tabularx}
	\caption{ \label{tbl: testprotokoll23}Testprotokoll Test 23, Quelle: Autor}
\end{table}