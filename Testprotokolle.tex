% !TEX root = PREN2_Dokumentation.tex
\section{Testprotokolle}\label{Testprotokolle}

\subsection{Testprotokolle Schülermodus}\label{testprotokolleSchuelermodus}

\begin{table}[H]
    \setlength\extrarowheight{2pt} % for a bit of visual "breathing space"
    \begin{tabularx}{\textwidth}{|l|X|}
        \hline
        \textbf{Test Nr.} & 1 \\
        \hline
        \textbf{Beschreibung} & Durch diesen Test wird die Lernfunktion sowie die Lernstatistik für Lernende manuell getestet.\\
        \hline
        \textbf{Randbedingungen} &
        \begin{minipage}[t]{0.6\textwidth}
            \begin{itemize}
                \item Die Testperson besitzt einen Account bzw. besitzt den Zugriff auf die Applikation.
                \item Die Testperson kann selbstständig auf die Fragen zugreifen.\\
            \end{itemize}
        \end{minipage} \\
        \hline
        \textbf{erwartete Resultate}  &
        \begin{minipage}[t]{0.6\textwidth}
            \begin{itemize}
                \item Wenn die Testperson ein Quiz startet, wird diesem eine Frage angezeigt. \\
                \item Der Testperson wird neben der Frage auch die Antwortmöglichkeiten angezeigt. \\
            \end{itemize}
        \end{minipage} \\
        \hline
        \textbf{Testperson} & Frederico Fischer \\
        \hline
        \textbf{Datum} & 01.10.2020 \\
        \hline
        \textbf{Durchführung} &
        \begin{minipage}[t]{0.6\textwidth}
            \begin{enumerate}
                \item Die Testperson gibt die URL ein.
                \item Die Testperson drückt auf den Reiter Übungen.
                \item Auf dem Bildschirm erscheint die Frage mit den möglichen Antworten.\\
            \end{enumerate}
        \end{minipage} \\
        \hline
        \textbf{erhaltenes Resultat} &
        \begin{minipage}[t]{0.6\textwidth}
            \begin{enumerate}
                \item Die Testperson hat eine Frage angezeigt bekommen.
                \item Die Testperson hatte eine Übersicht über alle möglichen Antworten.\\
            \end{enumerate}
        \end{minipage} \\
        \hline
        \textbf{Test bestanden} & Ja \\
        \hline
    \end{tabularx}
    \caption{ \label{tbl: testprotokoll1}Testprotokoll Test 1, Quelle: Autoren}
\end{table}

\newpage