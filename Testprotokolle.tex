% !TEX root = PREN2_Dokumentation.tex
\section{Testprotokolle}\label{Testprotokolle}

\subsection{Testprotokolle Bestellung}\label{testprotokolleBestellung}

\begin{table}[H]
    \setlength\extrarowheight{2pt} % for a bit of visual "breathing space"
    \begin{tabularx}{\textwidth}{|l|X|}
        \hline
        \textbf{Test Nr.} & 1 \\
        \hline
        \textbf{Beschreibung} & Durch diesen Test wird die Registrierung eines neuen Kunden getestet. \\
        \hline
        \textbf{Randbedingungen} &
        \begin{minipage}[t]{0.6\textwidth}
            \begin{itemize}
                \item Die Testperson hat sich mit ihrer Email-Adresse und Benutzernamen noch nie Registriert.
                \item Die Testperson nutzt den Nutzernamen "test" , die Email-Adresse "test@gmail.com" und das Passwort "ABC*1234" ein
            \end{itemize}
        \end{minipage} \\
        \hline
        \textbf{erwartete Resultate}  &
        \begin{minipage}[t]{0.6\textwidth}
            \begin{itemize}
                \item Das Konto der Testperson wird korrekt angelegt. 
                \item Ein Popup zeigt dem Nutzer den Status an
            \end{itemize}
        \end{minipage} \\
        \hline
        \textbf{Testperson} & Oliver Werlen \\
        \hline
        \textbf{Datum} & 18.04.2021 \\
        \hline
        \textbf{Durchführung} &
        \begin{minipage}[t]{0.6\textwidth}
            \begin{enumerate}
                \item Die Testperson gibt die angegebenen Daten ein. Die restlichen Daten werden zufällig gewählt
                \item Die Testperson klickt auf den Button "Registrieren"
                \item Es wird in einem Dialog die Erstellung des Nutzers dargestellt. 
                \item Es wird automatisch zur Login Page gewechselt
            \end{enumerate}
        \end{minipage} \\
        \hline
        \textbf{erhaltenes Resultat} &
        \begin{minipage}[t]{0.6\textwidth}
            \begin{enumerate}
                \item Das Erstellen des Nutzers wurde vom System via Popup bestätigt. 
            \end{enumerate}
        \end{minipage} \\
        \hline
        \textbf{Test bestanden} & Ja \\
        \hline
    \end{tabularx}
    \caption{ \label{tbl: testprotokoll1}Testprotokoll Test 1, Quelle: Autoren}
\end{table}

\begin{table}[H]
	\setlength\extrarowheight{2pt} % for a bit of visual "breathing space"
	\begin{tabularx}{\textwidth}{|l|X|}
		\hline
		\textbf{Test Nr.} & 2\\
		\hline
		\textbf{Beschreibung} & Durch diesen Test wird die Registrierung eines neuen Kunden getestet, wobei die Email Adresse und der Nutzername bereits genutzt werden. \\
		\hline
		\textbf{Randbedingungen} &
		\begin{minipage}[t]{0.6\textwidth}
			\begin{itemize}
				\item Der Test \ref{tbl: testprotokoll1} ist erfolgreich durchgeführt worden.
				\item Die Testperson nutzt den Nutzernamen test , die Email-Adresse test@gmail.com und das Passwort ABC*1234 ein\\
			\end{itemize}
		\end{minipage} \\
		\hline
		\textbf{erwartete Resultate}  &
		\begin{minipage}[t]{0.6\textwidth}
			\begin{itemize}
				\item Das System gibt dem Nutzer die Antwort, dass ein Benutzer mit dieser Email oder Benutzernamen bereits existiert. 
			\end{itemize}
		\end{minipage} \\
		\hline
		\textbf{Testperson} & Oliver Werlen \\
		\hline
		\textbf{Datum} & 18.04.2021 \\
		\hline
		\textbf{Durchführung} &
		\begin{minipage}[t]{0.6\textwidth}
			\begin{enumerate}
				\item Die Testperson gibt die angegebenen Daten ein. Die restlichen Daten werden zufällig gewählt
				\item Die Testperson klickt auf den Button "Registrieren"
				\item Es wird in einem Dialog mit der Meldung "Benutzer mit dieser Email oder Benutzernamen existiert bereits" \\
			\end{enumerate}
		\end{minipage} \\
		\hline
		\textbf{erhaltenes Resultat} &
		\begin{minipage}[t]{0.6\textwidth}
			\begin{enumerate}
				\item Das Popup wird wie geplant angezeigt. \\
			\end{enumerate}
		\end{minipage} \\
		\hline
		\textbf{Test bestanden} & Ja \\
		\hline
	\end{tabularx}
	\caption{ \label{tbl: testprotokoll2}Testprotokoll Test 2, Quelle: Autoren}
\end{table}

\begin{table}[H]
	\setlength\extrarowheight{2pt} % for a bit of visual "breathing space"
	\begin{tabularx}{\textwidth}{|l|X|}
		\hline
		\textbf{Test Nr.} & 3\\
		\hline
		\textbf{Beschreibung} & Die diesen Test wird die Login-Funktion getestet.  \\
		\hline
		\textbf{Randbedingungen} &
		\begin{minipage}[t]{0.6\textwidth}
			\begin{itemize}
				\item Der Test \ref{tbl: testprotokoll1} ist erfolgreich durchgeführt worden.
				\item Die Testperson nutzt den Benutzernamen test und das Passwort ABC*1234\\
			\end{itemize}
		\end{minipage} \\
		\hline
		\textbf{erwartete Resultate}  &
		\begin{minipage}[t]{0.6\textwidth}
			\begin{itemize}
				\item Eine Meldung wird angezeigt, mit welcher das erfolgreiche Erstellen des Nutzers bestätigt wird
				\item Im Sidenav wird der Logout Button angezeigt. \\
			\end{itemize}
		\end{minipage} \\
		\hline
		\textbf{Testperson} & Oliver Werlen \\
		\hline
		\textbf{Datum} & 18.04.2021 \\
		\hline
		\textbf{Durchführung} &
		\begin{minipage}[t]{0.6\textwidth}
			\begin{enumerate}
				\item Die Testperson gibt die angegebenen Daten ein.
				\item Der Benutzer klickt auf den Login Button\\
			\end{enumerate}
		\end{minipage} \\
		\hline
		\textbf{erhaltenes Resultat} &
		\begin{minipage}[t]{0.6\textwidth}
			\begin{enumerate}
				\item Das Popup wird wie geplant angezeigt.
				\item Das Sidenav wird entsprechend angepasst.\\  
			\end{enumerate}
		\end{minipage} \\
		\hline
		\textbf{Test bestanden} & Ja \\
		\hline
	\end{tabularx}
	\caption{ \label{tbl: testprotokoll3}Testprotokoll Test 3, Quelle: Autoren}
\end{table}
\begin{table}[H]
	\setlength\extrarowheight{2pt} % for a bit of visual "breathing space"
	\begin{tabularx}{\textwidth}{|l|X|}
		\hline
		\textbf{Test Nr.} & 4\\
		\hline
		\textbf{Beschreibung} & Die diesen Test wird die Login-Funktion getestet. Es wird eine falsche Kombination eingetragen.  \\
		\hline
		\textbf{Randbedingungen} &
		\begin{minipage}[t]{0.6\textwidth}
			\begin{itemize}
				\item Die Testperson nutzt den Benutzernamen testNoAccess und das Passwort ABC*1234*noAccess \\
			\end{itemize}
		\end{minipage} \\
		\hline
		\textbf{erwartete Resultate}  &
		\begin{minipage}[t]{0.6\textwidth}
			\begin{itemize}
				\item Eine Meldung Falscher Benutzername oder Passwort wird angezeigt\\
			\end{itemize}
		\end{minipage} \\
		\hline
		\textbf{Testperson} & Oliver Werlen \\
		\hline
		\textbf{Datum} & 18.04.2021 \\
		\hline
		\textbf{Durchführung} &
		\begin{minipage}[t]{0.6\textwidth}
			\begin{enumerate}
				\item Die Testperson gibt die angegebenen Daten ein.\\
				\item Der Benutzer klickt auf den Login Button\\
			\end{enumerate}
		\end{minipage} \\
		\hline
		\textbf{erhaltenes Resultat} &
		\begin{minipage}[t]{0.6\textwidth}
			\begin{enumerate}
				\item Das Popup wird wie geplant angezeigt. \\
			\end{enumerate}
		\end{minipage} \\
		\hline
		\textbf{Test bestanden} & Ja \\
		\hline
	\end{tabularx}
	\caption{ \label{tbl: testprotokoll4}Testprotokoll Test 4, Quelle: Autoren}
\end{table}
\begin{table}[H]
	\setlength\extrarowheight{2pt} % for a bit of visual "breathing space"
	\begin{tabularx}{\textwidth}{|l|X|}
		\hline
		\textbf{Test Nr.} & 5\\
		\hline
		\textbf{Beschreibung} & Bei diesem Test wird die Logout Funktion getestet.  \\
		\hline
		\textbf{Randbedingungen} &
		\begin{minipage}[t]{0.6\textwidth}
			\begin{itemize}
				\item Die Testperson ist erfolgreich eingeloggt. \\
			\end{itemize}
		\end{minipage} \\
		\hline
		\textbf{erwartete Resultate}  &
		\begin{minipage}[t]{0.6\textwidth}
			\begin{itemize}
				\item Popup mit der Meldung erfolgreich ausgeloggt\\
			\end{itemize}
		\end{minipage} \\
		\hline
		\textbf{Testperson} & Oliver Werlen \\
		\hline
		\textbf{Datum} & 18.04.2021 \\
		\hline
		\textbf{Durchführung} &
		\begin{minipage}[t]{0.6\textwidth}
			\begin{enumerate}
				\item Die Testperson klickt auf den Logout Button.\\
			\end{enumerate}
		\end{minipage} \\
		\hline
		\textbf{erhaltenes Resultat} &
		\begin{minipage}[t]{0.6\textwidth}
			\begin{enumerate}
				\item Das Popup wird wie geplant angezeigt. \\
			\end{enumerate}
		\end{minipage} \\
		\hline
		\textbf{Test bestanden} & Ja \\
		\hline
	\end{tabularx}
	\caption{ \label{tbl: testprotokoll5}Testprotokoll Test 5, Quelle: Autoren}
\end{table}
\begin{table}[H]
	\setlength\extrarowheight{2pt} % for a bit of visual "breathing space"
	\begin{tabularx}{\textwidth}{|l|X|}
		\hline
		\textbf{Test Nr.} & 6\\
		\hline
		\textbf{Beschreibung} & Bei diesem Test wird das hinzufügen von Produkten in den Warenkorb getestet.  \\
		\hline
		\textbf{Randbedingungen} &
		\begin{minipage}[t]{0.6\textwidth}
			\begin{itemize}
				\item Es sind Artikel im Shop vorhanden \\
			\end{itemize}
		\end{minipage} \\
		\hline
		\textbf{erwartete Resultate}  &
		\begin{minipage}[t]{0.6\textwidth}
			\begin{itemize}
				\item Der Artikel ist im Warenkorb vorhanden\\
			\end{itemize}
		\end{minipage} \\
		\hline
		\textbf{Testperson} & Oliver Werlen \\
		\hline
		\textbf{Datum} & 18.04.2021 \\
		\hline
		\textbf{Durchführung} &
		\begin{minipage}[t]{0.6\textwidth}
			\begin{enumerate}
				\item Die Testperson befindet sich auf der Produktübersicht\\
				\item Die Testperson klickt bei einem zufälligen Produkt auf das Warenkorbsymbol unten rechts\\
				\item Die Person klickt ein anderes Produkt an\\
				\item Die Testperson klickt auf den Button "in den Warenkorb"\\
				\item Die Testperson wechselt zum Warenkorb (Item oben rechts)\\
			\end{enumerate}
		\end{minipage} \\
		\hline
		\textbf{erhaltenes Resultat} &
		\begin{minipage}[t]{0.6\textwidth}
			\begin{enumerate}
				\item Die beiden Produkte sind im Warenkorb zu finden\\
				\item Die beiden Produkte befinden sich in einfacher Ausführung im Warenkorb\\
				\item Das Total ist korrekt summiert worden. \\
			\end{enumerate}
		\end{minipage} \\
		\hline
		\textbf{Test bestanden} & Ja \\
		\hline
	\end{tabularx}
	\caption{ \label{tbl: testprotokoll6}Testprotokoll Test 6, Quelle: Autoren}
\end{table}
\begin{table}[H]
	\setlength\extrarowheight{2pt} % for a bit of visual "breathing space"
	\begin{tabularx}{\textwidth}{|l|X|}
		\hline
		\textbf{Test Nr.} & 7\\
		\hline
		\textbf{Beschreibung} & Bei diesem Test wird die Persistierung des Warenkorbs getestet.  \\
		\hline
		\textbf{Randbedingungen} &
		\begin{minipage}[t]{0.6\textwidth}
			\begin{itemize}
				\item Der Test \ref{tbl: testprotokoll6} ist erfolgreich abgeschlossen worden. \\
			\end{itemize}
		\end{minipage} \\
		\hline
		\textbf{erwartete Resultate}  &
		\begin{minipage}[t]{0.6\textwidth}
			\begin{itemize}
				\item Die Artikel sind auch nach dem Verlassen der Website und einem erneuten Aufruf immer noch vorhanden. \\
			\end{itemize}
		\end{minipage} \\
		\hline
		\textbf{Testperson} & Oliver Werlen \\
		\hline
		\textbf{Datum} & 18.04.2021 \\
		\hline
		\textbf{Durchführung} &
		\begin{minipage}[t]{0.6\textwidth}
			\begin{enumerate}
				\item Die Testperson befindet sich im Warenkorb\\
				\item Die Testperson schliesst das Browserfenster\\
				\item Die Testperson öffnet das Browserfenster wieder und navigiert zum Warenkorb\\
				\item Der Warenkorbinhalt bleibt bestehen\\
			\end{enumerate}
		\end{minipage} \\
		\hline
		\textbf{erhaltenes Resultat} &
		\begin{minipage}[t]{0.6\textwidth}
			\begin{enumerate}
				\item Die Produkte sind immer noch im Warenkorb\\
			\end{enumerate}
		\end{minipage} \\
		\hline
		\textbf{Test bestanden} & Ja \\
		\hline
	\end{tabularx}
	\caption{ \label{tbl: testprotokoll7}Testprotokoll Test 7, Quelle: Autoren}
\end{table}
\begin{table}[H]
	\setlength\extrarowheight{2pt} % for a bit of visual "breathing space"
	\begin{tabularx}{\textwidth}{|l|X|}
		\hline
		\textbf{Test Nr.} & 8\\
		\hline
		\textbf{Beschreibung} & Bei diesem Test werden die Funktionen des Warenkorbs getestet.  \\
		\hline
		\textbf{Randbedingungen} &
		\begin{minipage}[t]{0.6\textwidth}
			\begin{itemize}
				\item Der Test \ref{tbl: testprotokoll6} ist erfolgreich abgeschlossen worden. \\
			\end{itemize}
		\end{minipage} \\
		\hline
		\textbf{erwartete Resultate}  &
		\begin{minipage}[t]{0.6\textwidth}
			\begin{itemize}
				\item Die Artikelanzahl wird erhöht. \\
				\item Das Gesamttotal wird erhöht.\\
				\item Die Artikelanzahl wird reduziert.\\
				\item Beim Erreichen von 0 wird der Artikel entfernt. \\
			\end{itemize}
		\end{minipage} \\
		\hline
		\textbf{Testperson} & Oliver Werlen \\
		\hline
		\textbf{Datum} & 18.04.2021 \\
		\hline
		\textbf{Durchführung} &
		\begin{minipage}[t]{0.6\textwidth}
			\begin{enumerate}
				\item Die Testperson befindet sich im Warenkorb\\
				\item Die Testperson erhöht die Artikelanzahl auf 4\\
				\item Die Testperson reduziert die Artikelanzahl auf 0\\
				\end{enumerate}
		\end{minipage} \\
		\hline
		\textbf{erhaltenes Resultat} &
		\begin{minipage}[t]{0.6\textwidth}
			\begin{enumerate}
				\item Es befinden sich keine Artikel mehr im Warenkorb
				\item Es wurde ein Popup mit einer entsprechenden Meldung angezeigt.  
			\end{enumerate}
		\end{minipage} \\
		\hline
		\textbf{Test bestanden} & Ja \\
		\hline
	\end{tabularx}
	\caption{ \label{tbl: testprotokoll8}Testprotokoll Test 8, Quelle: Autoren}
\end{table}

\begin{table}[H]
	\setlength\extrarowheight{2pt} % for a bit of visual "breathing space"
	\begin{tabularx}{\textwidth}{|l|X|}
		\hline
		\textbf{Test Nr.} & 9\\
		\hline
		\textbf{Beschreibung} & Bei diesem Test wird die Bezahlfunktion getestet.  \\
		\hline
		\textbf{Randbedingungen} &
		\begin{minipage}[t]{0.6\textwidth}
			\begin{itemize}
				\item Der Benutzer ist erfolgreich eingeloggt. \\
				\item Der Benutzer hat Produkte im Warenkorb. \\
			\end{itemize}
		\end{minipage} \\
		\hline
		\textbf{erwartete Resultate}  &
		\begin{minipage}[t]{0.6\textwidth}
			\begin{itemize}
				\item Der Benutzer wird auf die Bezahlseite weitergeleitet. \\
				\item Das Gesamttotal wird bei der Bezahlung korrekt ausgegeben. \\
				\item Der Benutzer wird nach dem Bezahlabschluss auf eine entsprechende Seite weitergeleitet. \\
			\end{itemize}
		\end{minipage} \\
		\hline
		\textbf{Testperson} & Oliver Werlen \\
		\hline
		\textbf{Datum} & 18.04.2021 \\
		\hline
		\textbf{Durchführung} &
		\begin{minipage}[t]{0.6\textwidth}
			\begin{enumerate}
				\item Die Testperson befindet sich im Warenkorb\\
				\item Die Testperson klickt auf checkout\\
				\item Die Testperson wird auf die Bezahlseite umgeleitet. \\
				\item Die Testperson klickt durch den Bezahlprozess.\\
			\end{enumerate}
			\end{minipage} \\
			\hline
			\textbf{erhaltenes Resultat} &
			\begin{minipage}[t]{0.6\textwidth}
				\begin{enumerate}
					\item Der Testperson wird eine Meldung zur Bezahlbestätigung ausgegeben. 
				\end{enumerate}
			\end{minipage} \\
			\hline
			\textbf{Test bestanden} & Ja \\
			\hline
		\end{tabularx}
		\caption{ \label{tbl: testprotokoll9}Testprotokoll Test 9, Quelle: Autoren}
	\end{table}

\begin{table}[H]
	\setlength\extrarowheight{2pt} % for a bit of visual "breathing space"
	\begin{tabularx}{\textwidth}{|l|X|}
		\hline
		\textbf{Test Nr.} & 10\\
		\hline
		\textbf{Beschreibung} & Bei diesem Test wird die Bezahlfunktion bei einem Abbruch getestet.  \\
		\hline
		\textbf{Randbedingungen} &
		\begin{minipage}[t]{0.6\textwidth}
			\begin{itemize}
				\item Der Benutzer ist erfolgreich eingeloggt. \\
				\item Der Benutzer hat Produkte im Warenkorb. \\
			\end{itemize}
		\end{minipage} \\
		\hline
		\textbf{erwartete Resultate}  &
		\begin{minipage}[t]{0.6\textwidth}
			\begin{itemize}
				\item Der Benutzer wird auf die Bezahlseite weitergeleitet. \\
				\item Das Gesamttotal wird bei der Bezahlung korrekt ausgegeben. \\
				\item Der Benutzer bricht den Bezahlvorgang mittels Cancel Button ab. 
				\item Der Benutzer wird auf eine entsprechende Seite weitergeleitet. \\
			\end{itemize}
		\end{minipage} \\
		\hline
		\textbf{Testperson} & Oliver Werlen \\
		\hline
		\textbf{Datum} & 18.04.2021 \\
		\hline
		\textbf{Durchführung} &
		\begin{minipage}[t]{0.6\textwidth}
			\begin{enumerate}
				\item Die Testperson befindet sich im Warenkorb\\
				\item Die Testperson klickt auf checkout\\
				\item Die Testperson wird auf die Bezahlseite umgeleitet. \\
				\item Die Testperson klickt beim Bezahlvorgang auf den cancel Button. \\
			\end{enumerate}
		\end{minipage} \\
			\hline
			\textbf{erhaltenes Resultat} &
			\begin{minipage}[t]{0.6\textwidth}
				\begin{enumerate}
					\item Der Testperson wird auf eine Page not found Seite umgeleitet. 
				\end{enumerate}
			\end{minipage} \\
			\hline
			\textbf{Test bestanden} & Ja \\
			\hline
		\end{tabularx}
		\caption{ \label{tbl: testprotokoll10}Testprotokoll Test 10, Quelle: Autoren}
	\end{table}
\newpage