\section{Stand der Technik}\label{StandDerTechnik}
\subsection{MyPost 24}
\subsubsection{Allgemein}
Mit der Pick Post und MyPost 24 können Briefe und Pakete an die Pick-Up Station gesendet werden. Es besteht auch das Angebot, Pakete von einer Pick-Up Station aus zu versenden. 
Die Auswahl einer Pick-Up Station geschieht mit der Angabe der entsprechenden Station. Der Dienst lässt sich bei jedem Onlineshop nutzen.\\
Die Abholung der Artikel muss innerhalb von 10 Tagen geschehen. Sobald die Artikel zur Abholung bereit sind, erhält der Kunde wahlweise eine Bestätigungsmail oder eine SMS. Mit dem darin enthaltenen Abholcode lässt sich die Ware entnehmen [\cite{postPickUp}].\\

Im Grossraum Luzern befinden sich acht MyPost 24-Abholstellen [\cite{myPost24Stations}]. 
Das Design der Abholstelle sieht konventionellen Briefkästen der Post sehr ähnlich. 
\begin{figure}[H]
	\centering
	\includegraphics[width=1\textwidth]{images/myPostImage.jpg}
	\caption[MyPost 24-Abholstelle]{MyPost 24-Abholstelle,Quelle: \cite{myPost24StationsImage}}
	\label{img: My Post 24-Abholstelle}
\end{figure}
Die Lösung der Post behebt nicht die in Kapitel \ref{Problem} beschriebenen Probleme. Das Angebot richtet sich hauptsächlich an Personen, welche bei der Lieferung der Post nicht zuhause sind. So kann das Abholen an der Poststelle vermieden werden. Lieferzeit sowie Lieferkosten bleiben vorhanden. Der Bestell- und Bezahlvorgang wird beim jeweiligen Onlineshop durchgeführt. 
\newpage 
\subsection{avec}
\subsubsection{avec now}
" — Dein Online Lieferservice von avec —
 Mit avec now haben wir rund um das beliebte Angebot unseres Convenience-Formats avec einen Online-Store lanciert.
 Zur Auswahl steht ein breites Convenience-Sortiment. Die bestellten Waren werden direkt in unseren Stores zusammengestellt und so schnell wie möglich ausgeliefert." [\cite{avecNow}]
 
 Auf der Website von avec now wird eine Lieferzeit von 60 Minuten aufgeführt, der Mindestbestellwert beträgt Fr. 20.00 [\cite{avecNowMain}].
 Das Angebot gilt für einen grossen Teil des Sortiments, darunter auch Tabakwaren. Es befindet sich noch in der Pilotphase und wird nur im Raum Zürich angeboten [\cite{avecNowShipping}].
 
 \subsubsection{avec Box}\label{avecBox}
 Der Anbieter geht bei diesem Angebot einen neuen Weg. Das gesamte Einkaufen wird mittels App durchgeführt. Durch diese können Produkte dem Warenkorb hinzugefügt und so bezogen werden. In einer ersten Phase sind zu Stosszeiten Mitarbeiter präsent, um Kunden beim Einkauf zu unterstützen.  \\
 Um Tabakprodukte zu kaufen, befindet sich in der App eine integrierte Altersverifikation. Die Tabakwaren werden im Store via Touchscreen ausgewählt [\cite{avecBoxTabak}].
 \begin{figure}[H]
 	\centering
 	\includegraphics[width=1\textwidth]{images/tabakkaufAvecBox.jpg}
 	\caption[Tabakkauf avec box]{Tabakkauf avec box, Quelle: \cite{avecBoxTabak}}
 	\label{img: avec box tabakkauf}
 \end{figure}
 
 Nach momentanem Stand steht die avec box am Campus der ETH Zürich. Die Box ist von Montag bis Sonntag jeweils von 6:00 bis 22:00 in Betrieb. Es ist ein Rollout in weitere Regionen der Schweiz vorgesehen [\cite{avecBoxStand}].\\
 Das Angebot von avec löst einige der genannten Probleme. So kann die Ware direkt bezogen werden, es fallen keine Versandkosten an. Zur Zeit ist die avec box nur an einem Standort verfügbar, was die Verfügbarkeit erheblich einschränkt. Zudem ist sie nur von 6:00 bis 22:00 in Betrieb. 
 
\subsubsection{Ablauf Tabakkauf}
Die avec box ist sehr ähnlich zur JTI Pick-Up Station. Aus diesem Grund wird der Registrierungsvorgang nachfolgend genauer betrachtet. 
\paragraph{Registrierung}
Das Anlegen eines avec-Kontos verläuft analog zur Erstellung von anderen Accounts. Zur Verifikation wird ein Bestätigungscode an die Handynummer gesendet. Anschliessend wird die Alterverifikation durchgeführt. Hierzu muss die Identitätskarte mit der Kamera eingelesen werden. In einem nächsten Schritt kann die Kreditkarte hinterlegt werden. Der Bezahldienst wird dabei von Datatrans bereitgestellt. Die Registierung ist abgeschlossen und der Einkauf in der avec box könnte beginnen. \\

\begin{figure}[h]
	\begin{subfigure}[b]{0.4\textwidth}
		\includegraphics[scale=0.15]{images/ID-Scan.PNG}
		\caption[Einlesen der Identitätskarte]{Einlesen der Identitätskarte, Quelle: \cite{avecApp}}
		\label{img: Einlesen der Identitaetskarte}
	\end{subfigure}
	\hfill
	 \begin{subfigure}[b]{0.4\textwidth}
		\includegraphics[scale=0.15]{images/creditCard.PNG}
		\caption[Einlesen der Kreditkarte in der avec App]{Einlesen der Kreditkarte in der avec App, Quelle: \cite{avecApp}}
		\label{img: Einlesen der Kreditkarte in der avec App}
	\end{subfigure}
\end{figure}

Leider wird in der Applikation keine Auskunft über den Anbieter der Alterverifikation gegeben. Das Vorgehen ist sehr intuitiv und schnell. 

\subsection{Starbucks \gls{PWA}}
Gemäss \cite{bestPWA} befindet sich die Starbucks \gls{PWA} in den Top-12 der besten \gls{PWA}'s. Die Applikation ermöglicht es dem Nutzer, die angebotenen Produkte zu bestellen und diese im Store abzuholen. Der Anwendungszweck ist somit ähnlich zur JTI Pick-Up Station. Sie ist sehr nahe an einer nativen App, wodurch dem Benutzer die Bedienung sehr leicht fällt. Die Applikation reagiert sehr schnell, es sind keine Ladezeiten zu bemerken. Zudem ist die \gls{PWA} auch offline nutzbar. Hierbei kommt es zwar zu Einschränkungen in der Nutzung, jedoch lässt sie sich weiterhin bedienen. 
\begin{figure}[h]
	\begin{subfigure}[b]{0.4\textwidth}
		\includegraphics[scale=0.15]{images/starbucks_main.PNG}
		\caption[Startseite der Starbucks \gls{PWA}]{Startseite der Starbucks \gls{PWA}, Quelle: \cite{starbucksPwaMain}}
		\label{img: Startseite der Starbucks PWA}
	\end{subfigure}
	\hfill
	\begin{subfigure}[b]{0.4\textwidth}
		\includegraphics[scale=0.15]{images/starbucks_angebot.PNG}
		\caption[Produktübersicht der Starbucks \gls{PWA}]{Produktübersicht der Starbucks \gls{PWA}, Quelle: \cite{starbucksPwaMenu}}
		\label{img: Produktübersicht der Starbucks PWA}
	\end{subfigure}
\end{figure}
\newpage
\subsubsection{Standortsuche}
Die Applikation bietet auch ein Feature, um die nächstgelegene Starbucksfiliale anzuzeigen. Hierbei wird auf die aktuelle Position des Nutzers zugegriffen. 
 \begin{figure}[H]
	\centering
	\includegraphics[scale=0.10]{images/starbucks_standort.jpeg}
	\caption[Standort in der Starbucks \gls{PWA}]{Standort in der Starbucks \gls{PWA}, Quelle: \cite{starbucksPwaStandort}}
	\label{img: Standort in der Starbucks PWA}
\end{figure}

\subsubsection{Fazit}
Die Applikation von Starbucks zeigt auf, was eine \gls{PWA} leisten kann. Sie dient als Vorbild für die Applikation der \ac{JTI} Pick-Up Station. 


\subsection{Fazit}
Es existieren diverse Produkte, welche einen ähnlichen Ansatz verfolgen wie dieses Projekt. Besonders hervorzuheben ist die avec box \ref{avecBox}. Der Betreiber verfolgt einen ähnlichen Ansatz. 
Der Kaufvorgang bei Tabakwaren unterscheidet sich kaum von dem in diesem Projekt umzusetzenden. Durch die Analyse des dort verwendeten Vorgehens konnte ein guter Überblick gewonnen werden. Es konnte beobachtet werden, wie die Integration der Altersverifikation umgesetzt wurde. Dieses Wissen ist für die spätere, eigene Umsetzung sehr wichtig. \\
Die Applikation von Starbucks liefert einen sehr guten Überblick über die Möglichkeiten von \gls{PWA}'s. Besonders designtechnisch ist diese Anwendung sehr wertvoll.\\
Die anderen analysierten Angebote lieferten keinen Mehrwert für das Projekt, da sie die gestellte Problematik nur bedingt oder gar nicht lösen.  

\newpage 