%Packages
\documentclass[titlepage]{article}
\usepackage[ngerman]{babel}
\usepackage[utf8]{inputenc}
\usepackage{color}
\usepackage{import}
\usepackage{amssymb}
\usepackage{wrapfig}
\usepackage{amsthm}
\usepackage{graphicx}
\usepackage[printonlyused, withpage]{acronym}
\usepackage{abstract}
\usepackage{tabularx,ragged2e}
\usepackage{float}
\restylefloat{table}
\usepackage{pdfpages}
\usepackage{titleref}
\usepackage{hyperref}
\usepackage{enumitem}
\usepackage[style=apa]{biblatex}
\usepackage{glossaries}
% compile with pdflatex deckball_bachelorarbeit.tex
\usepackage{amssymb} 
\usepackage{color}
\usepackage[a4paper,bindingoffset=0.2in,%
	left=3.35cm,right=2.12cm,top=3.75cm,bottom=2.88cm,%
	footskip=.25in]{geometry}
\usepackage[T1]{fontenc}
\usepackage{graphicx,wrapfig,lipsum}
\usepackage{subcaption}
\captionsetup{compatibility=false}


\newcolumntype{C}{>{\Centering\arraybackslash}X}
\usepackage{fancyhdr}
\pagestyle{fancy}
\fancyhf{}
\rhead{BAA}
\lhead{Oliver Werlen}
\rfoot{Page \thepage}
\addbibresource{biblio.bib}

\makeglossaries

\newglossaryentry{React}
{
	name=React,
	description={Java Script Library zum Erstellen von User Interfaces [\cite{react}]}
}

\newglossaryentry{Angular}
{
	name=Angular,
	description={Framework für Cross Plattform Developing [\cite{angular}]}
}

\newglossaryentry{Vue}
{
	name=VueJS,
	description={Progressives JavaScript Framework[\cite{vue}]}
}

\newglossaryentry{EnterpriseLab}
{
	name=Enterprise Lab,
	description={Das Enterprise Lab stellt Studierenden modernste Computerressourcen zur Verfügung. Das Enterprise Lab ist als Rechenzentrum mittlerer grösse aufgebaut und implementiert neueste Technologien [\cite{enterpriselab}] }
}

\newglossaryentry{LocalStorage}
{
	name=Local Storage,
	description={Datenspeicher im Browser, Domänen- und Protokollabhängig, über die Sessiondauer hinaus verfügbar}
}

\newglossaryentry{SessionStorage}
{
	name=Session Storage,
	description={Datenspeicher im Browser, Domänen- und Protokollabhängig, nur während der Session verfügbar }
}

\newglossaryentry{mixedContent}
{
	name=mixed Content,
	description={Abfragen einer http-Ressource aus einem https-Context [\cite{mixedContent}] }
}

\newglossaryentry{Spring}
{
	name=Spring,
	description={Java Framework, um schnellere, eifachere und sicherere Java Applikationen zu erstellen. [\cite{springBoot}]}
}

\newglossaryentry{DevOps}
{
	name=DevOps,
	description={"Der Begriff DevOps setzt sich aus „Dev“ (Development, Entwicklung) und „Ops“ (Operations, Vorgänge) zusammen und vereint Menschen, Prozesse und Technologien, damit Kunden kontinuierlich hochwertige Produkte erhalten."}
}
\newglossaryentry{zerotier}
{
	name=zerotier,
	description={Kombination aus SD-WAN und VPN, Traffic ist end-to-end Verschlüsselt }
}

\newglossaryentry{Micronaut}
{
	name=Micronaut,
	description={Ein modernes, JVM basiertes full-stack Framework zum Bauen von Microservices und Serverless Applikationen [\cite{micronaut}]}
}

\newglossaryentry{Spring Boot}
{
	name=Spring Boot, 
	description={Spring Boot nutzt das Spring-Framework und bietet zusätlich einen eingebauten Tomcat Server. Mit Spring Boot bleibt Spring auch weiterhin interessant für Java Enterprise Applications. [\cite{springBoot}]}
}

\newglossaryentry{Tomcat}
{
	name=Tomcat, 
	description={Open Source Implementation von Java Servlet, Java Server Pages, Java Expression Language und JavaWebSocket [\cite{tomcat}]}
}

\newglossaryentry{MobileFirst}
{
	name=Mobile First , 
	description={Es wird die Applikation zuerst für mobile Geräte entwickelt. Erst in einem zweiten Schritt werden Desktops berücksichtigt}
}

\newglossaryentry{KISS}
{
	name=Keep it simple stupid, 
	description={Zu einem Problem eine möglichst einfache Lösung herstellen.}
}
\newglossaryentry{Hibernate}
{
	name=Hibernate ORM, 
	description={Hibernate Object/Relational Mapping ist ein Framework, das es Entwicklern ermöglicht, einfach Applikationen mit persistenten Daten zu entwickeln [\cite{hibernate}]}
}
\newglossaryentry{PWA}
{
	name=Progressive Web App, 
	description={Kombination von Webapplikation und nativer Applikation, untersteht den Kriterien von Google \href{https://web.dev/pwa-checklist/}{PWA-Checklist} [\cite{pwaWhat}]}
}

\newglossaryentry{SoDa}
{
	name=SoDa, 
	description={Hybrides Projektmanagementvorgehen der Hochschule Luzern}
}

\newglossaryentry{SDWAN}
{
	name=Software defined wide area network, 
	description={Virtuelle WAN-Architektur, mit der beliebige Wege für die Datenübertragung kombiniert werden können [\cite{sdwan}]}
}

\newglossaryentry{fairUsageAlgorithmus}
{
	name=fair usage algorithmus, 
	description={Die Anzahl der momentan auszuführenden Jobs pro Projekt entscheidet die Auswahl des Runners. Das Projekt mit den am wenigsten laufenden Jobs kommt zuerst [\cite{runnersGitlab}]}
}

\newglossaryentry{Integration Test}
{
	name=Integrations Test, 
	description={Test, der das Zusammenspiel von mehreren Komponenten testet. Integration Tests können durch Mocking zu Unit Tests gemacht werden }
}

\newglossaryentry{Unit Test}
{
	name=Unit Test, 
	description={Test, der nur eine einzelne Unit testet, beispielsweise nur eine Methoden}
}
\newglossaryentry{Postman}
{
	name=Postman, 
	description={Collaberationsplattform für API Development, bietet auch Funktionen für automated Testing}
}
\newglossaryentry{TypeScript}
{
	name=TypeScript, 
	description={Basiert auf JavaScript, fügt zusätzlich statische Typisierung hinzu [\cite{typescript}]}
}
\newglossaryentry{DockerHub}
{
	name=DockerHub, 
	description={Plattform, um Container Applikationen zu verteilen und zur Verfügung zu stellen}
}

\newglossaryentry{nginx}
{
	name=nginx, 
	description={Open Source Web Server}
}


\newglossaryentry{WebHooks}
{
	name=WebHooks, 
	description={WebHooks werden zur Kommunikation zwischen zwei Diensten genutzt. Das Vorgehen ist vergleichbar mit dem aus der Programmierung bekannten Observer Pattern. Es wird das Push-Verhalten genutzt. Der Event wird von der API durch einen POST-Request ausgelöst. Es handelt sich um eine asynchrone Kommunikation [\cite{webhooksExplained}]}
}


\newglossaryentry{war}
{
	name=Web Applikation Archive,
	description={Enthält alle Inhalte einer Web Applikation, r eduziert die Übertragungszeit [\cite{war}]}
}
\newglossaryentry{Latex}
{
	name=Latex, 
	description={Markup Sprache, besonders geeignet für wissenschaftliche Arbeiten}
}
\newglossaryentry{Github}
{
	name=GitHub, 
	description={grösste Versionskontrollplattform [\cite{Github}]}
}

\newglossaryentry{User Story}
{
	name=User Story, 
	description={Element in SoDa, werden aus Epics im Format Als Rolle möchte ich Ziel/Wunsch, um Nutzen erstellt [\cite{userStory}]}
}

\newglossaryentry{Intellij}
{
	name=Intellij, 
	description={\ac{IDE} von JetBrains}
}

\newglossaryentry{RxJS}
{
	name=RxJS, 
	description={Eine der meistgenutzten Libraries beim Webdevelopment. Ermöglicht den Einsatz von mächtigen funktionale Ansätzen wie Observables [\cite{rxjs}]}
}

\newglossaryentry{RestTemplate}
{
	name=RestTemplate, 
	description={Ermöglicht das Senden eines API-Request von einer Spring Applikation aus }
}
\newglossaryentry{SpringBoot Initlializer}
{
	name=Spring Boot Initlializer, 
	description={Tool, das ein vordefiniertes Spring Boot Projekt mit Mavenunterstützung erstellt }
}

\newglossaryentry{Propertie-File}
{
	name=Propertie-File, 
	description={Datei, welche bei Java die Konfigurationen enthält. Hat die Dateiendung .propertie}
}

\newglossaryentry{MySQL}
{
	name=MySQL, 
	description={Datenbankmanagementsystem von Oracle}
}
\newglossaryentry{GitLab}
{
	name=GitLab, 
	description={Versionskontrollsystem aufbauend auf git, alternative zu \gls{Github}}
}

\newglossaryentry{Docker}
{
	name=Docker, 
	description={Software zur Containervirtualisierung}
}

\newglossaryentry{Proxy}
{
	name=Proxy, 
	description={Kommunikationsschnittstelle in einem Netzwerk}
}

\newglossaryentry{Spring Security}
{
	name=Spring Security, 
	description={De-Facto Standardframework für die Sicherung von \gls{Spring} basierten Applikationen [\cite{SpringSecurity}]}
}

\newglossaryentry{Token Based Authentication}
{
	name=Token Based Authentication, 
	description={Authentifizierungart bei API's, ein definierter Token wird dabei bei jedem Request mitgesendet. Anhand von diesem Entscheidet die API, ob eine Berechtigung vorliegt}
}


\begin{document}
%TITELSEITE
\include{Titlepage}

\begin{abstract}
Die JT International AG (Japan Tobacco International) will einen neuen Absatzmarkt erschliessen. Die Produkte sollen via Internet bestellbar sein und an einer Pick-Up Station abgeholt werden.\\ Ziel der Arbeit ist, die Machbarkeit anhand eines Prototypen zu bewerten. Er besteht aus drei eigenständigen Arbeiten aus den Bereichen Informatik, Elektrotechnik und Maschinenbau an der Hochschule Luzern. In diesem Teil wird die Informatik bearbeitet.\\\\
Basierend auf der Projektmanagementmethode \gls{SoDa} werden bestehende Lösungen analysiert und in Zusammenarbeit mit dem Auftraggeber die Anforderungen erhoben. Mittels Domain Driven Design wird die Software entworfen und mit modernsten Frameworks umgesetzt. Es handelt sich um eine vollwertige \gls{PWA}, welche den entsprechenden Kriterien entspricht. Durch die Integration einer Altersverifikation werden die rechtlichen Voraussetzungen erfüllt, die Bezahlung wird nach aktuellsten Standards durch den Anbieter Six durchgeführt. Der Prototyp ist vollständig in die Partnersysteme integrierbar.  \\\\

Der Prototyp beweist die Erschliessbarkeit des neuen Marktes für \ac{JTI}. Der Zigarettenkauf via Mobiltelefon wird durch dieses Projekt salonfähig. Er bietet eine solide Basis für einen erfolgreichen Start von \ac{JTI} in diesen neuen Absatzmarkt. 
 
\end{abstract}
\tableofcontents 
\newpage

% !TEX root = PREN2_Dokumentation.tex
\section{Problem und Vision}\label{Problem}
\subsection{Problem}
Der Onlinekauf von Zigaretten wird von Kunden sehr selten genutzt. Vor allem die nachfolgenden Punkte sind dafür verantwortlich. 
\begin{itemize}
	\item Versandkosten/Mindestbestellwert
	\item Dauer bis Ware beim Endkonsumenten
	\item Angebot nur in begrenztem Zeitraum möglich
\end{itemize}

\subsubsection{Versandkosten/Mindestbestellwert}
Bei diversen Onlineshops kommen bei zu geringer Bestellmenge erhebliche Versandkosten hinzu. So kostet der Versand per Paket in der Regel Fr. 9.00. Bei Kioskolino ist der Versand ab einem Bestellwert von Fr. 139.00 portofrei [\cite{kioskolino}]. 
\\Bei Coop ist die Liefergebühr höher, sie beträgt Fr. 17.90. Die Versandkosten nehmen mit zunehmendem Bestellwert ab. Bei einer Bestellung ab Fr. 500.00 ist der Versand kostenlos [\cite{coop}]. 
\subsubsection{Dauer bis Ware bei Endkonsumenten}
Bei der Bestellung bei Kioskolino wird die Ware innerhalb von drei Werktagen ausgeliefert [\cite{kioskolino}]. 
Coop verspricht die Lieferung am selben Tag, bei der Bestellung können verschiedene Zeitfenster ausgewählt werden. Die Verfügbarkeit ist von der Region abhängig. Die Ware muss aber frühzeitig bestellt werden, um die Lieferung am gleichen Tag garantieren zu können. Zudem bietet Coop auch die Möglichkeit, die Produkte direkt in der Filiale abzuholen [\cite{coop}].\\
Das Problem ist bei beiden Anbietern identisch. Die Lieferung dauert zwischen vier Stunden bis hin zu drei Tagen. 

\subsubsection{Angebot  nur in begrenztem Zeitraum möglich}
Die bestellte Ware wird nur zu bestimmten Zeiten ausgeliefert. Eine Lieferung an Sonn- und Feiertagen nicht möglich. 

\subsection{Vision}
Durch die JTI Pick-Up Station wird es dem Kunden möglich, seine Ware bequem im Onlineshop zu bestellen und ohne Wartezeit an der gewünschten Pick-Up Station abzuholen. \\
Die Artikel werden für den Kunden an der gewählten Pick-Up Station reserviert, durch das Vorzeigen der Bestellbestätigung wird der Artikel freigegeben und kann entnommen werden. \\ 
Ein Mindestbestellwert muss nicht erreicht werden, es fallen keine zusätzlichen Gebühren an.  \\
Die Applikation soll durch eine einfache und intuitive Bedienung eine optimale Benutzerexperience bieten. Die Bestellung soll schnell und einfach ablaufen, die Abholung in kurzer Zeit abgewickelt werden. Durch die Umsetzung als \gls{PWA} ist die Applikation ohne aktive Internetverbindung begrenzt nutzbar, sie unterscheidet sich nur geringfügig von einer nativen App. \\
Die rechtlichen Anforderungen können durch den Einsatz eines Altersverifikationsanbieters erreicht werden. Der Bezahlvorgang wird durch einen etablierten Anbieter durchgeführt. Das garantiert eine sichere und zuverlässige Bezahlabwicklung. 

\newpage
\section{Stand der Technik}\ref{StandDerTechnik}
Here comes some text
\newpage 
% !TEX root = PREN2_Dokumentation.tex
\section{Ideen und Konzepte}
\subsection{Vorwort}
Durch die progressive Web App \ac{JTI} Pick-Up Station ist es dem Kunden möglich, seine Ware bequem im Onlineshop zu bestellen und direkt und ohne Wartezeit an der gewünschten Pick-Up Station abzuholen. Durch den Prototyp sollen die Funktionalität und Zweckmässigkeit dieses für die Firma neuen Absatzkanals aufgezeigt werden. Im besten Fall findet die Applikation nicht nur in der Schweiz Verwendung, sondern wird von \ac{JTI} auch in anderen Märkten weltweit eingesetzt. Das Hauptaugenmerk der Arbeit liegt auf der Implementierung eines Prototyps mit den folgenden Schwerpunkten: Bestellung, Kauf, Nutzererfassung, Suche nach Pick-Up Stations und der Abholung an der Station. Bei der Nutzererfassung muss das Alter des Nutzers verifiziert werden. Die Lösung soll so weit als möglich in die Projektpartner-Systeme integriert werden. Hinzu kommt die Recherche von artverwandten Technologien und das Requirements Engineering. Die BDA wird als interdisziplinäre Bachelorarbeit durchgeführt. 

\subsection{Systemarchitektur}
Als Systemarchitektur stand die Erweiterbarkeit im Vordergrund. Allerdings sollte durch die Architektur die Applikation nicht unnötig komplex werden. Aus diesem Grund wurde bewusst gegen eine Microservicearchitektur entschieden. Die Umsetzung der Applikation mit Microservices würde zwar zu einer besseren Verteilbarkeit und Skalierung führen, der Aufwand der Umsetzung würde jedoch erheblich steigen. \\
Es wurde auf eine Schichtenarchitektur gesetzt. Die klassische logische drei Schichtenarchitektur wurde noch weiter verfeinert, final wurde eine sechs Schichtenarchitektur entworfen. Die Architektur wird zusätzlich in 3 physische Tier aufgeteilt. 
\begin{figure}[H]
	\centering
	\includegraphics[width=\linewidth]{images/architectureWithImages.png}
	\caption[Architektur]{Architektur, Quelle: Autor}
	\label{img: Architektur}
\end{figure}
Die Architektur ermöglicht eine sehr gute Erweiterbarkeit. Zudem kann die Software verteilt werden und unabhängig voneinander skalieren. Durch die Aufteilung in sechs Tiers wird zudem vermieden, dass die einzelnen Tiers zu breit werden. Durch das Verwenden von Frameworks kann die Komplexität gering gehalten werden. Die Schnittstellen innerhalb der einzelnen Layer sind klar vorgegeben. Durch die Kommunikation via REST, bzw. Repositories sind die Komponenten untereinander austauschbar. Die Kopplung ist sehr gering. 

\paragraph{UI}
Es wird eine Single-Page Applikation umgesetzt. Sie wird vorab im Browser geladen, später wird nur der entsprechende Inhalt neu aktualisiert. Das bringt eine verbesserte Nutzerexperience, da keine Abhängigkeiten zu den Serverladezeiten bestehen. Hingegen dauert das initiale Laden länger [\cite{spa}]. 

\paragraph{UI Logic}
Aufgrund der gewählten Architektur werden die Daten der Applikation via \ac{REST}-Schnittstelle geladen. Dies wird asynchron durchgeführt.

\paragraph{REST-Controller}
Die einzelnen \ac{REST}-Controller definieren die \ac{API}. Es wurden die folgenden Punkte bei der Erstellung berücksichtigt: 
  \begin{itemize}
	\item Konsistenz: Namensgebung und Regeln konsequent einhalten.
	\item Namensgebung: Einfache, eingängige, treffende Namen wählen.
	\item Verhalten: Klare Erwartungen erfüllen, ohne Nebeneffekte.
	\item Erweiterbarkeit: Offen für Weiterentwicklung (der API).
	\item Dokumentation: Einfache, hilfreiche, kompakte Dokumentation.
	\item Perspektive: Vom Anbieter für den Nutzer - es soll für den Nutzer einfach werden!
	\item KISS-Prinzip: Schnittstelle möglichst einfach halten.
	\item Sicherheit: Die Nutzung ist sicher zu gestalten.
\end{itemize}
[\cite{appeAPIDesign}]

Es wird gemäss Richardson Maturity Model \ref{img: richardsonMaturity} eine \ac{REST}-Schnittstelle von Level 3 angestrebt. 

\paragraph{Business Services}
Die Business Services kommunizieren mit den Domänenmodellen. Weitere Details werden im Abschnitt \ref{services} beschrieben. 

\paragraph{Business Models}
Es handelt sich um die Entities des Projekts. Sie werden aus dem Domänenmodell erarbeitet und sind einzigartig. Weitere Details werden im Abschnitt \ref{entity} beschrieben. 

\paragraph{Data Logic}
Zur Persistierung der Entities werden Repositories eingesetzt. Sie bilden die Schnittstelle zwischen Applikation und Datenbank. 

\subsubsection{Zusammenspiel der einzelnen Layer}
\begin{figure}[H]
	\centering
	\includegraphics[width=\linewidth]{images/springFlow.png}
	\caption[Zusammenspiel von zentralen Layern]{Zusammenspiel von zentralen Layern, Quelle: Autor}
	\label{img: layer}
\end{figure}

\subsubsection{Infrastruktur}
Die gesamten Infrastruktur läuft im \gls{EnterpriseLab} der Hochschule Luzern. Das Front- und das Backend laufen auf unterschiedlichen virtuellen Maschinen. Die einzelnen Teilapplikationen sind als Docker-Container umgesetzt. 
\begin{figure}[H]
	\centering
	\includegraphics[width=\linewidth]{images/system.png}
	\caption[Container-Infrastruktur]{Container-Infrastruktur, Quelle: Autor}
	\label{img: Containerinfrastruktur}
\end{figure}

Die beiden Diagramme \ref{img: Architektur} und \ref{img: Containerinfrastruktur} sind farblich identisch gehalten. 
Gesamthaft umfasst das System sechs Container.\\
Der Station Client in diesem Diagramm befindet sich auf der physischen Station. Er ist das Bindeglied zwischen Informatik und Elektrotechnik und übernimmt Aufgaben wie die Produktausgabe oder das Inventar. 

\subsection{Mobile First}
Es werden erheblich mehr Websiteaufrufe von mobilen, als von Desktop Geräten registriert. Der mobile First Ansatz nimmt sich dieser Thematik an und besagt, dass die Applikation in einem ersten Schritt für mobile Geräte entwickelt wird. Der Fokus liegt auf dem Design der Seite, wohingegen die Performance ebenfalls berücksichtigt wird.\\
Die Bedeutung von mobile First hat durch Google dazugewonnen. Der mobile Index ist der primäre Index, die mobile Friendliness der ausschlaggebende Punkt beim Ranking von Suchergebnissen [\cite{mobileFirst}]. 
\subsection{Frameworks}
\subsubsection{\gls{Spring Boot}}
Spring Boot baut auf dem \gls{Spring} Framework auf. Dieses bietet sich an, um performante Enterprise-Applikationen mit Java zu erstellen. Spring Boot kann sehr einfach erweitert werden, sodass Spring Security, Spring MVC oder Spring Data eingesetzt werden können.\\
Spring Boot bringt einen integrierten Tomcat Server mit. Dies verringert den Konfigurationsaufwand. Es ist kein Einpacken in ein \gls{war} nötig und kein zusätzliches deployen. Zudem wird das Debugging erleichtert.  \\
Es übernimmt dabei einen grossen Teil der Beans-Konfiguration von Spring [\cite{springBoot}].\\\\
Spring Boot bietet dabei den grossen Vorteil, dass bereits Erfahrung in der Anwendung vorhanden ist. Es ist somit keine Einarbeitungszeit nötig, gängige Fehler können vermieden werden. Aus diesem Grund fiel die Entscheidung gegen Frameworks wie NodeJS. 

\subsubsection{\gls{Angular}}\label{angularLabel}
Angular ist ein Application Design Framework, um effiziente Single Page Applikationen zu erstellen. Es basiert auf TypeScript [\cite{angular}]. Zudem bietet Angular die Möglichkeit, als \gls{PWA} genutzt zu werden [\cite{angularPWA}]. 
Angular wurde dabei von Google entwickelt und bietet mit dem Material UI bereits viele Elemente, welche von nativen Android-Apps bekannt sind [\cite{angularMaterialUI}]. \\\\
Wie auch bei Spring ist auch bei Angular bereits Projekterfahrung vorhanden. Ein Einarbeiten ist nicht mehr nötig, die gängigsten Funktionen sind bereits bekannt. \\\\
In vorhergehenden Projekten wurde das Frontend mit verschiedensten Technologien umgesetzt. Einerseits kam jQuery zum Einsatz, andererseits wurden auch Technologien wie \gls{React} genutzt. \gls{Angular} überzeugte von diesen Framework am meisten. Besonders durch die Unterstützung von \gls{TypeScript} fiel die Wahl auf dieses Framework. 

\subsection{Weitere Technologien}
\subsubsection{Docker}
Docker ist eine Containertechnologie. Sie erlaubt die Erstellung und den Betrieb von Linux-Containern. Die Container sind dabei sehr leichtgewichtig und modular.\\
Die einzelnen Teile der Applikation werden in verschiedenen Containern betrieben [\cite{docker}].\\
Die Anwendung und Funktion von Docker wird in diversen Modulen an der Hochschule Luzern gelehrt. In vorhergegangenen Projekten wurde Docker bereits im selben Kontext genutzt. 

\subsubsection{MariaDB}\label{mariadb}
MariaDB ist ein OpenSource Datenbankmanagementsystem. Es ist durch die Abspaltung von MySQL entstanden. Es handelt sich um ein Relationale Datenbanksystem. 
Aus Lizenzgründen wird in diesem Projekt MariaDB und nicht MySQL genutzt. Alternativ wäre auch ein Einsatz von PostgreSQL möglich. Der Unterschied zwischen PostgreSQL und MariaDB ist dabei nur marginal. Bei dieser Applikation wurde lediglich aus Erfahrung auf MariaDB gesetzt [\cite{mariadbVsPostgresql}]. \\
Durch die Verwendung von Spring Data wäre es möglich, die Datenbank im Verlauf des Projektes auszutauschen. 

\subsubsection{Hibernate ORM}
Hibernate ORM ist ein Object Relational Mapper. 
Er ermöglicht dem Entwickler, einfacher mit der Datenpersistierung umzugehen. Hibernate ist zudem ein Teil der \ac{JPA} und der Spring Data JPA. 
Ein wichtiger Punkt ist zudem die Performance, welche durch den Einsatz des Lazy Loading Patterns erreicht wird. Die Initialisierung des Objektes wird dabei solange als möglich hinausgezögert. [\cite{hibernateORM}]

\subsubsection{\gls{GitLab}}
Zur Versionsverwaltung kam \gls{GitLab} zum Einsatz. Es wird dabei von der Hochschule Luzern zur Verfügung gestellt. Zudem sind zur Integration in die \gls{DevOps}-Umgebung bereits GitLab Runner vorhanden, um Docker Images zu Builden. 

\subsubsection{Bezahldienst}
Als Bezahldienst wurde Saferpay von Six Payment Services genutzt. Dabei wurde mit der Testversion gearbeitet, diese unterscheidet sich nicht von der Produktiven. Ein Wechsel wäre zudem innert kurzer Zeit durchführbar. 

\subsubsection{Alterverifikation}
Bei der Altersüberprüfung wurde auf Jumio gesetzt. Das Produkt wurde vom Auftraggeber vorgegeben und eine Lizenz bereitgestellt. 

\subsubsection{Karte}
Für die Implementierung der Kartenfunktionalität wurde Leaflet genutzt. Die Kartendaten stammen von OpenStreetMap. Zudem wurde Geoapify als MapsAPI eingesetzt. 

\newpage
% !TEX root = PREN2_Dokumentation.tex
\section{Methoden}
\subsection{\gls{SoDa}}
Das Projektmanagement wird mit dem hybriden Projektvorgehen \gls{SoDa} der Hochschule Luzern durchgeführt. 
SoDa wurde dabei bereits schon in vorhergehenden Softwareentwicklungsmodulen eingesetzt und hat sich hier bewährt. Die Userstories werden dabei auch im \gls{GitLab} erfasst.  
\begin{figure}[H]
	\centering
	\includegraphics[width=1\textwidth]{images/boardGitlab.png}
	\caption[GitLab Board]{GitLab Board,\\ Quelle: Autor}
	\label{img: GitlLabBoard}
\end{figure}
Für die Sprint Planung werden einzelne User Stories verschoben. Beim Sprint Review werden die Stories in "reviewing" mit den definierten Akzeptanzkriterien mit den Resultaten verglichen. Basierend darauf wird entschieden, ob an der \gls{User Story} noch weiter gearbeitet werden, das heisst zurück zu "doing" oder die Story in "done" verschoben werden kann. Bei Beginn des nächsten Sprints wird das Vorgehen wiederholt. \\
Die einzelnen Items sind dabei priorisiert. Elemente mit einer hohen Einstufung werden dabei bei der Bearbeitung vorgezogen. 

\subsection{CI/CD}
\begin{figure}[H]
	\centering
	\includegraphics[width=1\textwidth]{images/sequenceCicd.png}
	\caption[Ablauf der CI/CD Pipeline]{Ablauf der CI/CD Pipeline,\\ Quelle: Autor}
	\label{img: cicdPipeline}
\end{figure}

\subsection{Data Transfer Objects}\label{DTO}
\newpage
% !TEX root = PREN2_Dokumentation.tex
\section{Realisierung}\label{realisierung}
\newpage

% !TEX root = PREN2_Dokumentation.tex
\section{Evaluation und Validation}\label{Evaluation}
\subsection{Ziel der Arbeit}
Durch die progressive Web App «JTI Pick-Up Station» ist es dem Kunden möglich, seine Ware bequem im Onlineshop zu bestellen und direkt und ohne Wartezeit an der gewünschten Pick-Up Station abzuholen. Durch den Prototyp sollen die Funktionalität und Zweckmässigkeit dieses für die Firma neuen Absatzkanals aufgezeigt werden. Im besten Fall findet die Applikation nicht nur in der Schweiz Verwendung, sondern wird von JTI auch in anderen Märkten weltweit eingesetzt. Das Hauptaugenmerk der Arbeit liegt auf der Implementierung eines Prototyps mit den folgenden Schwerpunkten: Bestellung, Kauf, Nutzererfassung, Suche nach Pick-Up Stations und der Abholung an der Station. Bei der Nutzererfassung muss das Alter des Nutzers verifiziert werden. Die Lösung soll so weit als möglich in die Projektpartner-Systeme integriert werden. Hinzu kommt die Recherche von artverwandten Technologien und das Requirements Engineering. \\\\

\subsection{Validierung der Requirements}
\subsubsection{Funktionale Requirements}
\paragraph{PWA-Checklist}
Es werden die Kriterien von Google genutzt. [\cite{pwaChecklist}]
\subparagraph{Core Progressive Web App checklist}
\begin{itemize}
	\item [{$\times$}] Starts fast, stays fast
	\item [{$\checkmark$}] Works in any browser
	\item [{$\checkmark$}] Responsive to any screen size
	\item [{$\checkmark$}] Provides Custom offline page
	\item [{$\checkmark$}] Is Installable
\end{itemize}
\subparagraph{Optimal Progressive Web App checklist}
Es wurde bei der Implementierung sogar noch weiter gegangen, als dass es dieses Requirement vorschreibt. So wurden die meisten Punkte von der Optimal checklist ebenfalls abgeschlossen. 
\begin{itemize}
	\item [{$\checkmark$}] Provides an offline experience
	\item [{$\checkmark$}] Is fully accessible
	\item [{$\times$}] Can be discovered through search
	\item [{$\checkmark$}] Works with any input type
	\item [{$\checkmark$}] Provides context permission requests
	\item [{$\checkmark$}] Follows best practice for healthy code
\end{itemize}

\paragraph{Abstimmung auf physische Pick-Up Station}
Während dem gesamten Projekt wurde der Kontakt mit der Elektrotechnik gesucht und von Beginn an die Schnittstelle definiert. Die Kommunikation von der Informatik zur Elektrotechnik bei der Abholung einer Bestellung funktioniert. Auch kann in der entgegengesetzten Richtung eine Verbindung hergestellt werden, es werden aber nicht alle Daten erhalten. Eine Lösung für dieses Problem konnte nicht gefunden werden. Daher muss dieses Requirement als nicht erfüllt angesehen werden. 

\paragraph{Registrierung und Login}
Es ist dem Nutzer möglich, sich zu registrieren und ein Login durchzuführen. Die Umsetzung dieses Mechanismus wurde durch \ac{JWT} umgesetzt. \\
Damit der Nutzer sich nach der Registrierung anmelden kann, ist es zwingen nötig, die Altersverifikation erfolgreich abzuschliessen. Erst dann ist ein Login möglich. 

\paragraph{Altersverifikation}
Die Altersverifikation war einer der zentralen Punkte bei diesem Projekt. Die Implementierung dient als Referenz für andere Projekte des Auftraggebers, bei denen eine Altersverifikation zum Einsatz kommt. \\
Die Implementierung der ID-Verifikation verlieft Problemlos und konnte sehr einfach umgesetzt werden. Um die Altersverifikation durchzuführen, war ein erneuter API-Call nötig. Hier wurde bemerkt, dass die Daten erst nach einer bestimmten Zeit zur Abfrage bereit stehen. Somit kann die Altersverifikation erst nach einer bestimmten Zeit ausgelöst werden, was den Prozess erheblich verlängert, auch weil die Implementierung auf der API-Seite sehr unvorteilhaft ist. Die Altersverifikation wurde mit verschiedenen Testcases getestet, um auch die Zuverlässigkeit gewährleisten zu können. Es wurden alle bestanden. 

\paragraph{Bestellung}
Bei der Umsetzung eines Webshops ist es unumgänglich, dass auch Produkte bestellt werden können. Dem Kunden stehen im Shop verschiedenste Produkte zur Auswahl, abhängig von der ausgewählten Pick-Up Station, mit der entsprechenden Verfügbarkeit. \\Vom Auftraggeber wurde der Bezahldienst saferpay von Six Payment vorgegeben. Die Kommunikation mit der API war dabei kein Problem, jedoch führte \ac{CORS} zu Problemen. Es wurde eine Lösung gefunden, die Bezahlung funktioniert einwandfrei. Da es sich um einen Prototypen handelt, wird ein Testsystem verwendet. Die genaueren Unterschiede werden in Kapitel \ref{testvslive} beschrieben. \\

\paragraph{Pick-Up Station}
Damit ein Nutzer eine Bestellung platzieren kann, muss er zwingend eine Station ausgewählt haben. Dem Nutzer werden auf der Karte alle Station angezeigt. Die Koordinaten, der Name und die Beschreibung werden initial vom Administrator gesetzt. Auf der Station besteht keine Möglichkeit, diese Daten einzugeben. \\Zu einer Station gehört ein Inventar von Produkten. Bei jeder Bestellung wird es angepasst. Dieses vorgehen ist im finite State Machine \ref{img: stateMachine} dargestellt.\\
Die Anzeige von Pick-Up Station ist voll funktionsfähig, es sind keine Fehler bekannt. 

\paragraph{Aktueller Warenbestand}
Dem Administrator wird eine Möglichkeit angeboten, kritische Warenbestände zu bemerken. In einer Tabelle sieht er alle Inventuren, welche weniger als fünf Produkte umfassen. \\
Das Senden von Benachrichtigungen in diesem Fall wurde nur konzeptuell umgesetzt. Daher ist dieses Requirement nicht erfüllt, es wäre hier Nacharbeit zu leisten. 

\paragraph{Hinzufügen und Bearbeiten von Artikeln}
Für das Hinzufügen von Artikel ist keine graphische Oberfläche umgesetzt worden. Selbiges gilt für die Bearbeitung von Produkten.

\paragraph{Abholung}
Zur Abholung muss der Nutzer einen QR-Code auf der Station mit der \gls{PWA} einlesen. Die Verfügbaren Bestellungen werden in seinem Profil angezeigt, von hier aus kann zum QR-Code Scanner gewechselt werden. \\
Bei einer erfolgreichen Abholung kommen die Daten korrekt bei der Station an. Das Senden von den Daten über die \ac{UART}-Schnittstelle ist möglich. Dennoch konnte das gesamte Zusammenspiel aufgrund von fehlender Hardware nicht getestet werden. Das Requirement darf aber dennoch als erreicht betrachtet werden, da von der Informatikteil die Daten korrekt an die Elektrotechnik übergibt. 

\subsubsection{Nicht funktionale Anforderungen}
\paragraph{Mit 5 Klicks bestellen}
Wird davon ausgegangen, dass der Nutzer bereits angemeldet ist. Zudem wird der Bezahlvorgang nicht dazugerechnet. \\
Die Anzahl Klicks kann, abhängig vom Anwendungsfall, sehr stark variieren. Im Durchschnitt benötigt der Nutzer zwischen 5-7 Klicks, um eine Bestellung platzieren zu können. Dass Requirements darf als erreicht angesehen werden. 

\paragraph{Internationaler Einsatz}
Das UI wurde auf Deutsch gestaltet. Um International eingesetzt werden zu können, wäre es nötig, verschiedene Sprachen anzubieten.\\ Die Features haben keine Abhängigkeit zum Standort. Bei der Zeit wird im gesamten System die Zeitzone Europe/Zurich verwendet. Hier wäre es wünschenswert, dass bei der Anzeige im \ac{UI} die aktuelle Zeitzone des Benutzers genutzt wird. \\
In der momentanen Version ist die Applikation nur im deutschsprachigen Raum sinnvoll nutzbar. 

\paragraph{Kommunikation via HTTPS}
Die Kommunikation im gesamten System wird via \ac{HTTPS} durchgeführt. Die Verbindung zwischen Backend und Pick-Up Station ist ebenfalls durch \gls{zerotier} verschlüsselt. 

\paragraph{Authentifizierungsmechanismus}
\ac{JWT} gilt als de-facto Standard beim Sichern von API's. Die Tokens werden zusätzlich mit einem private Key signiert, um sicherzustellen, dass es sich um einen Token von dieser API handelt. Ein Blacklisting von Token ist nicht umgesetzt worden. Der Nutzer wird im Falle von längerer Inaktivität automatisch abgemeldet, der Token aus dem \gls{SessionStorage} entfernt. Die Gültigkeit des Tokens ist mit einer Stunde bewusst gering gehalten. \\

\paragraph{Erweiterbarkeit}
Es wurden durchgängig die empfohlenen Vorgehensweisen der Frameworks genutzt. Durch die Dokumentation mittels \ac{UML}-Diagrammen von bestimmten Teilbereichen wird die Einarbeitungszeit verringert und das Verständnis erhöht. Im Projekt wurden die gängigsten Clean-Code Ansätze verwendet. 

\paragraph{CI/CD Pipeline}
Die beiden Hauptbestandteile besitzen eine CI/CD Pipeline. Die Pipeline übernimmt das Erstellen und die Verteilung des Docker Containers auf die produktive Umgebung. Dabei ist es sehr einfach möglich, ein Deployment in eine andere Umgebung durchzuführen. \\

Bei der Deployment-Stage tritt zuweilen das Problem auf, dass diese nicht erfolgreich durchläuft. Der Fehler ist sehr selten und nicht nachvollziehbar. Bei einem Retry von dieser Stage wird der Container erfolgreich auf die produktive Umgebung verteilt. 

\subsubsection{Übersicht}
\begin{itemize}
	\item [{$\checkmark$}] PWA-Checklist
	\item [{$\checkmark$}] Abstimmung auf physische Pick-Up Station
	\item [{$\checkmark$}] Registrierung und Login
	\item [{$\checkmark$}] Provides Custom offline page
	\item [{$\checkmark$}] Is Installable
\end{itemize}
\newpage
% !TEX root = PREN2_Dokumentation.tex
\section{Ausblick}
Die Bachelordiplomarbeit zeigt auf, dass der Absatzkanal für die \ac{JTI} auf diese Art und Weise erschliessbar ist. Die gewünschten Dienste lassen sich integrieren, die rechtlichen Anforderungen einhalten.\\\\
In diesem Projekt wurde das Vorgehen als Prototyp umgesetzt. Die Entwicklungstätigkeiten wurden entsprechend priorisiert, wobei das automatisierte Testing dieser Strategie zum Opfer fiel. Dennoch ist die Applikation manuell getestet worden, jedoch nur mit einer begrenzten Anzahl von Nutzern. 
Um ein späteres Refactoring zu vereinfachen, müssen automatisierte Unit- und Integrationstests implementiert und entsprechende Testing Stages hinzugefügt werden. Es sollte eine Test- und Entwicklungsumgebung bereitgestellt werden.
\\
Die Usability wurde nur mit sehr wenigen Nutzern getestet. Ein Testing mit einer breiten Nutzergruppe wäre sehr aufschlussreich und würde aufzeigen, wo Nacharbeitet geleistet werden muss. Die Möglichkeit zum Passwortwechsel oder zum erneuten Versuchen der Altersverifikation sollten umgehend hinzugefügt werden. Ebenfalls sollte dem Nutzer nach der erfolgreichen Bezahlung eine Bestätigungsmail gesendet werden. Die Applikation ist momentan auf Deutsch verfügbar. Es müssten entsprechende Versionen für andere Sprachen erstellt werden, damit die Applikation international genutzt werden kann. \\
Damit die Applikation besser Skalieren kann und um eine bessere Verfügbarkeit zu erreichen, wäre ein Deployment auf einen Kubernetes Cluster sehr zu empfehlen. Die Applikationen liegen bereits als Docker-Container vor, der Aufwand wäre gering. \\
Der Administrator wurde nur bedingt berücksichtigt. Das konzeptuelle Vorgehen \ref{verfugbarkeiten} würde einen sehr grossen Komfortgewinn bieten. Der Administrator sollte auch die Möglichkeit haben, neue Produkte hinzufügen und auch bearbeiten zu können. 
Um den vollen Funktionsumfang der Applikation zu ermöglichen, muss mindestens eine physische, voll funktionsfähige Station zur Verfügung stehen. Erst wenn das der Fall ist und dem Nutzer garantiert werden kann, dass er das bezahlte Produkt erhält, kann eine Migration auf ein Livesystem von Sixpayment durchgeführt werden. Es wären auch andere Ansätze möglich, um die Abhängigkeit zur Station zu beheben, beispielsweise eine alternative Abholung im Kiosk. \\\\
Bei diesen Punkten handelt es sich um kleinere Verbesserungen. Das explorative Testing wird weitere Punkte aufzeigen. Besonders in der Funktionsbreite bietet die Applikation enormes Potential, um den Prozess zu optimieren. 
\newpage
% !TEX root = PREN2_Dokumentation.tex
\section{Verzeichnisse}
\listoffigures
\newpage
\printglossary
\newpage
\section{Abkürzungsverzeichnis}
\begin{acronym}[Bash]
    \acro{SoDa}{Software Development Agile}
    \acro{URL}{Uniform Resource Locator}
    \acro{bspw}{beispielsweise}
    \acro{bzw}{beziehungsweise}
    \acro{REST}{Representational State Transfer}
    \acro{API}{Application Programming Interface}
    \acro{HTTP}{Hypertext Transfer Protocoll}
    \acro{HTTPS}{Hypertext Transfer Protocol Secure}
    \acro{SQL}{Structured Query Language}
    \acro{JPA}{Java Persistence API}
    \acro{CI/CD}{Continous Integration and Continous Deployment}
    \acro{CI}{Continous Integration}
    \acro{CD}{Continous Deployment}
    \acro{SSH}{Secure Shell}
    \acro{CA}{Certificate Authority}
    \acro{DNS}{Domain Name System}
    \acro{JSON}{Java Script Object Notation}
    \acro{CORS}{Cross Origin Ressource Sharing}
    \acro{HATEOAS}{Hypermedia as the Engine of Application State \cite[hateoas]}
    \acro{IDE}{Integrierte Entwicklungsumgebung}
    \acro{CRUD}{Create, Read, Update, Delete}
    \acro{JSON}{Java Script Object Notation}
    \acro{JWT}{JSON Web Token}
    \acro{DTO}{Date Transfer Object}
    \acro{CSS}{Cascading Style Sheet}
    \acro{JTI}{Japan Tobacco International}
    \acro{PWA}{Progressive Web App}
    \acro{NFC}{Near Field Communication}
    \acro{IoT}{Internet of Things}
    \acro{PCI DSS}{Payment Card Industry Data Security Standard}
    \acro{SAQ}{Self Assigned Questionary}
    \acro{CORS}{Cross-Origin Ressource Sharing}
    \acro{TLS}{Transport Layer Security}
    \acro{JPA}{Java Persistence API}
    \acro{PWA}{Progressive Web App}
    \acro{IDE}{Integrierte Entwicklungsumgebung}
    \acro{SEO}{Search Engine Optimization}
    \acro{IP}{Internet Protocol}
    \acro{UART}{Univeral Asynchronous Receiver Transmitter}
    \acro{UDP}{User Datagram Protocol}
    \acro{VPN}{Virtual Private Network}
    \acro{UML}{Unified Modeling Language}
    \acro{UI}{User Interface}
\end{acronym}
\listoftables
\newpage
\printbibliography
\appendix
% !TEX root = PREN2_Dokumentation.tex
\section{Testprotokolle}\label{Testprotokolle}

\subsection{Testprotokolle Schülermodus}\label{testprotokolleSchuelermodus}

\begin{table}[H]
    \setlength\extrarowheight{2pt} % for a bit of visual "breathing space"
    \begin{tabularx}{\textwidth}{|l|X|}
        \hline
        \textbf{Test Nr.} & 1 \\
        \hline
        \textbf{Beschreibung} & Durch diesen Test wird die Lernfunktion sowie die Lernstatistik für Lernende manuell getestet.\\
        \hline
        \textbf{Randbedingungen} &
        \begin{minipage}[t]{0.6\textwidth}
            \begin{itemize}
                \item Die Testperson besitzt einen Account bzw. besitzt den Zugriff auf die Applikation.
                \item Die Testperson kann selbstständig auf die Fragen zugreifen.\\
            \end{itemize}
        \end{minipage} \\
        \hline
        \textbf{erwartete Resultate}  &
        \begin{minipage}[t]{0.6\textwidth}
            \begin{itemize}
                \item Wenn die Testperson ein Quiz startet, wird diesem eine Frage angezeigt. \\
                \item Der Testperson wird neben der Frage auch die Antwortmöglichkeiten angezeigt. \\
            \end{itemize}
        \end{minipage} \\
        \hline
        \textbf{Testperson} & Frederico Fischer \\
        \hline
        \textbf{Datum} & 01.10.2020 \\
        \hline
        \textbf{Durchführung} &
        \begin{minipage}[t]{0.6\textwidth}
            \begin{enumerate}
                \item Die Testperson gibt die URL ein.
                \item Die Testperson drückt auf den Reiter Übungen.
                \item Auf dem Bildschirm erscheint die Frage mit den möglichen Antworten.\\
            \end{enumerate}
        \end{minipage} \\
        \hline
        \textbf{erhaltenes Resultat} &
        \begin{minipage}[t]{0.6\textwidth}
            \begin{enumerate}
                \item Die Testperson hat eine Frage angezeigt bekommen.
                \item Die Testperson hatte eine Übersicht über alle möglichen Antworten.\\
            \end{enumerate}
        \end{minipage} \\
        \hline
        \textbf{Test bestanden} & Ja \\
        \hline
    \end{tabularx}
    \caption{ \label{tbl: testprotokoll1}Testprotokoll Test 1, Quelle: Autoren}
\end{table}

\newpage
% !TEX root = PREN2_Dokumentation.tex

\section{Projektmanagementplan}

\subsection{Projektorganisation}
\subsubsection{Organisationsplan, Rollen, Zuständigkeiten}
\begin{figure}[H]
    \centering
    %\includegraphics[width=1\textwidth]{bilder/Organigramm.png}
    \caption[Organigramm]{Organigramm\\ Quelle: Autoren}
    \label{img: OrganigrammWiPro}
\end{figure}

\paragraph{Rollen}
\subsubsection{Projektstrukturplan}

\begin{figure}[H]
    \centering
   % \includegraphics[width=1\textwidth]{bilder/pmp.png}
    \caption[Projektstrukturplan]{Projektstrukturplan,\\ Quelle: Autoren}
    \label{img: Projektstrukturplan}
\end{figure}

\paragraph{Beschreibung}
Im Projektstrukturplan in Abbildung \ref{img: Projektstrukturplan} wird das Projekt in 3 grobe Teile geteilt. Sie sind dabei angepasst an die Grobplanung und decken die Anforderungen ab. 
\newpage
\subsection{Projektführung}
\subsubsection{Rahmenplan}

Im untenstehenden Rahmenplan wird mittels Zeitstrahl eine Grobplanung dargestellt. 

\begin{figure}[H]
    \centering
   % \includegraphics[width=1\textwidth]{bilder/SoDa_Zeitstrahl.png}
    \caption[SoDa Rahmenplan]{Rahmenplan,\\ Quelle: Autoren}
    \label{img: SoDa Rahmenplan}
\end{figure}

Nach 7 absolvierten Sprints wurde der Rahmenplan angepasst. 
\begin{figure}[H]
    \centering
    %\includegraphics[width=1\textwidth]{bilder/Rahmenplan_2.png}
    \caption[SoDa Rahmenplan überarbeitet]{Rahmenplan überarbeitet,\\ Quelle: Autoren}
    \label{img: SoDaRahmenplanUeberarbeitet}
\end{figure}
\newpage
\subsubsection{Meilensteine}\label{Meilensteine}
Wie in Abbildung \ref{img: SoDa Rahmenplan} zu sehen gibt es insgesamt sieben Meilensteine.
Diese werden in folgender Tabelle beschrieben sowie die nötigen Deliverables aufgezeigt.


\begin{table}[H]
\setlength\extrarowheight{2pt} % for a bit of visual "breathing space"
\begin{tabularx}{\textwidth}{|C|C|C|}
\hline
\textbf{Meilenstein} &  \textbf{Beschreibung} & \textbf{Deliverables}  \\

\hline
Projektstart & Bei Meilenstein eins wird das Kickoff-Meeting mit allen Projektteilnehmern durchgeführt. & finale Aufgabenstellung\\

\hline
Start Umsetzung & Bei Meilenstein zwei wird vom klassischen Projektmanagement zum agilen Projektmanagement übergegangen. Dazu muss die Initialisierungsphase abgeschlossen sein& Projektmanagementplan, Systemspezifikation, Anforderungsliste\\

\hline
Abschluss Systemkontext & Zu diesem Zeitpunkt ist alles bereit, um mit der Entwicklung zu beginnen. Es wurden bereits erste GUI Entwürfe erarbeitet sowie die Systemarchitektur definiert.  & CI/CD Umgebung eingerichtet, GUI-Prototyp, UML-Diagramme  
\newline \textbf{Release 1}
\\

\hline
Abschluss Schülermodus & Der Entwicklung des Schülermodus ist abgeschlossen. Es können vorgegebene Fragen beantwortet werden sowie eine Statistik zu bisher Gelerntem eingesehen werden.  & Testprotokolle zu Schülermodus, Demo Schülermodus, Release Schülermodus 
\newline
\textbf{Release 2}
 \\

\hline
Abschluss Lehrermodus & Der Entwicklung des Lehrermodus ist abgeschlossen. Es können Fragen und Prüfungen erstellt und verteilt werden. Zusätzlich sind die Fragen aus der alten Applikation integriert. & Testprotokolle zu Lehrermodus, Testprotokolle Prüfungsmodus, Integration alte Daten, Demo verschiedene Modis, Release Lehrermodus  
\newline
\textbf {Release 3}
\\

\hline
Start Einführung & Der Auftraggeber erhält eine Einführung in die Software & Sitzungsprotokoll zum Ende der Einführungsphase  
\\

\hline
Projektende & Der Auftraggeber erhält eine Einführung in die Software & Fertige Projektdokumentation, Abgeschlossene Testprotokolle   
\newline
\textbf{Release 4}
\\
\hline
\end{tabularx}
\caption{ \label{tbl: Meilensteine}Meilensteine, Quelle: Autoren}
\end{table}
\newpage
\subsubsection{Risikomanagement}
Beim Risikomanagement werden die wichtigsten Risiken für das Projekt ermittelt und passende Gegenmassnahmen ausgearbeitet. 

\begin{table}[H]
\setlength\extrarowheight{2pt} % for a bit of visual "breathing space"
\begin{tabularx}{\textwidth}{|C|C|C|}
\hline
\textbf{Risiko} & \textbf{Eintrittswahrsch.} & \textbf{Schaden} \\

\hline
Falsche Zeiteinschätzung &  70 & 80\\
\hline
Requirements nehmen zu / Requirements ändern sich & 60 & 60\\
\hline
Entwicklerausfall & 20 & 70\\
\hline
Unklare Spezifikationen & 10 & 30\\
\hline
Vernachlässigung Designprozess & 20 & 60\\
\hline
Zeitverlust unnötige Features & 60 & 50\\
\hline
Fehlende technische Kenntnisse & 40 & 90\\
\hline
\end{tabularx}
\caption{ \label{tbl: Risikoanalyse}Risikoanalyse, Quelle: Autoren}
\end{table}

\begin{figure}[H]
    \centering
   % \includegraphics[width=1\textwidth]{bilder/RiskMap.png}
    \caption[Risikomatrix]{Risikomatrix,\\ Quelle: Autoren}
    \label{img: Risikomatrix}
\end{figure}

\paragraph{Beschreibung}
Basierend auf der Risikomatrix in Abbildung \ref{img: Risikomatrix} müssen für die Risiken im rechten oberen Viertel Gegenmassnahmen erarbeitet werden. Dabei handelt es sich um die folgenden Risiken:
\begin{itemize}
\item Falsche Zeiteinschätzung
\item Requirements nehmen zu/Veränderung der Requirements
\item Zeitverlust durch unnötige Features
\item Fehlende technische Kenntnisse
\end{itemize}

\paragraph{Gegenmassnahmen}
\subparagraph{Falsche Zeiteinschätzung}
Um das Risiko einer falschen Zeiteinschätzung zu minimieren wird bei der Planung auf bestehende, erfolgreich abgeschlossene Projekte zurückgegriffen. Basierend auf diesen wird die Zeitplanung durchgeführt.  
\subparagraph{Requirements nehmen zu/Veränderung der Requirements}
Die Requirements werden fortlaufend im Product Backlog überprüft. Die Aufgabenstellung dient dabei als Basis. Mittels vom Product Owner abgesegneten Akzeptanzkriterien ist der Umfang klar abgegrenzt. 
 \subparagraph{Zeitverlust durch unnötige Features}
Durch den Product Backlog sowie den Sprint Backlog mit klaren Beschreibungen sowie Akzeptanzkriterien der Issues sind dem Entwickler zu jedem Zeitpunkt die zu bearbeitenden Punkte klar.
 \subparagraph{Fehlende technische Kenntnisse}
Beim Projekt wird auf vielgenutzte Technologien mit grosser Community gesetzt. 

\begin{table}[H]
\begin{tabularx}{\textwidth}{|C|C|C|}
\hline
\textbf{Risiko} & \textbf{Eintrittswahrsch.} & \textbf{Schaden} \\
\hline
Falsche Zeiteinschätzung &  30 & 80\\
\hline
Requirements nehmen zu / Requirements ändern sich & 20 & 10\\
\hline
Zeitverlust unnötige Features & 30 & 50\\
\hline
Fehlende technische Kenntnisse & 40 & 40\\
\hline
\end{tabularx}
\caption{ \label{tbl: RisikoanalyseNachMassnahmen}Risikoanalyse nach Massnahmen, Quelle: Autoren}
\end{table}

\begin{figure}[H]
    \centering
    %\includegraphics[width=1\textwidth]{bilder/RiskMap_nachher.png}
    \caption[RisikomatrixNach]{Risikomatrix nach Massnahmen,\\ Quelle: Autoren}
    \label{img: RisikomatrixNachher}
\end{figure}

\subsubsection{Definition of done}
In jedem Sprint müssen die nachfolgenden Punkte zwingend erreicht werden, um ein potenziell auslieferbares Produkt zu erhalten:

\begin{itemize}
\item Review durchgeführt
\item Akzeptanzkriterien erfüllt
\item Unit Tests Grün
\item CI/CD ohne Fehler
\item keine kritischen Bugs
\item Clean Code Guidelines eingehalten
\item Dokumentation aktuell
\end{itemize}
\newpage
\subsection{Projektunterst\"utzung}
\subsubsection{Tools f\"ur Entwicklung, Test und Abnahme}
\paragraph{Entwicklungstools}
Bei der Entwicklung des Projekts kommen folgende Programme zum Einsatz: 

\begin{table}[H]
\setlength\extrarowheight{2pt} % for a bit of visual "breathing space"
\begin{tabularx}{\textwidth}{|C|C|C|}
\hline
\textbf{Typ} &\textbf{Tool} & \textbf{Version}  \\

\hline
IDE & Intelij Ultimate  & 2020.1\\
\hline
IDE & Webstorms & 2020.2\\ 
\hline
Versionsverwaltung & Git & 2.27.0\\
\hline
\end{tabularx}
\caption{ \label{tbl: Entwicklungstools}Entwicklungstools, Quelle: Autoren}
\end{table}
\paragraph{Testtools}
Beim Testing kommen folgende Tools zum Einsatz

\begin{table}[H]
\setlength\extrarowheight{2pt} % for a bit of visual "breathing space"
\begin{tabularx}{\textwidth}{|C|C|C|}
\hline
\textbf{Typ} &\textbf{Tool} & \textbf{Version}  \\
\hline
Unit Testing & JUnit  & 5.6.2\\
\hline 
API Testing & Postman & 7.36.0\\
\hline
\end{tabularx}
\caption{ \label{tbl: Testtools}Testtools, Quelle: Autoren}
\end{table}
\subsubsection{Konfigurationsmanagement}
\paragraph{Konfigurationseinheit}

Bei diesem Projekt besteht eine Konfigurationseinheit aus mehreren Teilen. Dabei werden diese bei jedem Release aufgeführt. Zusätzlich dazu kommen noch die Reports der Automatisierten Tests, falls vorhanden auch der Systemtests.
\begin{itemize}
\item API
\item Datenbank
\item Webapplikation
\item Dokument
\end{itemize}
\paragraph{Release 1}

\begin{table}[H]
\setlength\extrarowheight{2pt} % for a bit of visual "breathing space"
\begin{tabularx}{\textwidth}{|C|C|C|}
\hline
\textbf{Typ} &  \textbf{Version}  \\

\hline
API &  1.0.0\\
\hline
Datenbank &  1.0.0\\
\hline
Webapplikation  & 1.0.0\\
\hline
Dokumentation & 0.9\\
\hline
\end{tabularx}
\caption{ \label{tbl: Konfigurationseinheit Release 1}Konfigurationseinheit Release 1, Quelle: Autoren}
\end{table}

\paragraph{Testprotokolle}
Die gesamten Testprotokolle sind im Anhang \ref{Testprotokolle} zu finden. 
\newpage
\subsection{Teststrategie und Drehbuch}
\subsubsection{Teststrategie}
Es wird bei diesem Projekt hauptsächlich auf Automated Testing gesetzt.
Unit Tests werden dabei Integration Tests vorgezogen.
Hierzu wird auf das bewährte JUnit Framework gesetzt.
Es wird dabei das Test-First-Prinzip verwendet.

\paragraph{Automated Testing der REST-Schnittstelle}
Zum Testen der Rest-Schnittstelle wird Unirest sowie JUnit verwendet.  


\subsubsection{Testdrehbuch}\label{testsvonmeilensteine}
Wie oben genannt wird hautpsächlich auf Automated Testing gesetzt.
Daher werden nur sehr wenige manuelle Tests durchgeführt.
Die Tests gehen mit den gleichnamigen Meilensteinen einher.
Nachfolgend werden diese inklusive den erhaltenen Resultate beschrieben.

\begin{table}[H]
\begin{tabularx}{\textwidth}{lX}
  \hline
  \multicolumn{2}{|c|}{Test Lernmodus Frage anzeigen} \\
  \hline
  Test Nr. & 1\\
  Beschreibung & Durch diesen Test wird die Lernfunktion sowie die Lernstatistik für Lernende manuell getestet.\\
  Randbedingungen & Die Testperson hat einen bereits eingerichteten Account mit den für sie relevanten Fragen.\\
  erwartete Resultate & Der Nutzer bekommt eine Frage inklusive den möglichen Antworten angezeigt.  \\
  Testperson & Frederico Fischer \\
  Datum & 01.10.2020 \\
  Testprotokoll & \ref{tbl: testprotokoll1}\\
   \hline
\end{tabularx}
\caption{ \label{tbl: Test Lernmodus Frage anzeigen}Test Lernmodus Frage anzeigen, Quelle: Autoren}
\end{table}


\begin{table}[H]
\begin{tabularx}{\textwidth}{lX}
  \hline
  \multicolumn{2}{|c|}{Test Lernmodus Frage beantworten} \\
  \hline
  Test Nr. & 2\\
  Beschreibung & Durch diesen Test wird die Beantwortung von Fragen getestet.\\
  Randbedingungen & Die Testperson hat einen bereits eingerichteten Account mit für sie relevanten Fragen. Der Test Nr. 1 ist erfolgreich verlaufen\\
  erwartete Resultate & Der Nutzer kann eine Fragen beantworten und gelangt direkt zur nächsten.  \\
  Testperson & Frederico Fischer \\
  Datum & 01.10.2020 \\
  Testprotokoll & \ref{tbl: testprotokoll2}\\
   \hline
\end{tabularx}
\caption{ \label{tbl: Test Lernmodus Frage beantworten}Test Lernmodus Frage anzeigen, Quelle: Autoren}
\end{table}

\begin{table}[H]
\begin{tabularx}{\textwidth}{lX}
  \hline
  \multicolumn{2}{|c|}{Test Lernmodus Frage korrigieren} \\
  \hline
  Test Nr. & 3\\
  Beschreibung & Durch diesen Test wird die Korrektur von Fragen getestet.\\
  Randbedingungen & Die Testperson hat einen bereits eingerichteten Account mit für sie relevanten Fragen. Die Tests Nr. 1 und 2 sind erfolgreich verlaufen\\
  erwartete Resultate & Bei der Beantwortung der Frage wird dem Nutzer angezeigt, ob die angewählte Lösung korrekt war. Falls nicht, wird die richtige Lösung angezeigt.  \\
  Testperson & Frederico Fischer \\
  Datum & 01.10.2020 \\
  Testprotokoll & \ref{tbl: testprotokoll3}\\
   \hline
\end{tabularx}
\caption{ \label{tbl: Test Lernmodus Frage korrigieren}Test Lernmodus Frage korrigieren, Quelle: Autoren}
\end{table}

\begin{table}[H]
\begin{tabularx}{\textwidth}{lX}
  \hline
  \multicolumn{2}{|c|}{Test Lernmodus Statistik zu User anzeigen} \\
  \hline
  Test Nr. & 4\\
  Beschreibung & Durch diesen Test wird die Anzeige einer Userstatistik gestestet.\\
  Randbedingungen & Es sind bereits 3 Statistiken zu unterschiedlichen Fragen in der Datenbank vorhanden. \\
  erwartete Resultate & Dem Nutzer wird seine Nutzerstatistik bezogen auf die beantworteten Fragen angezeigt.  \\
  Testperson & Frederico Fischer \\
  Datum & 01.12.2020 \\
  Testprotokoll & \ref{tbl: testprotokoll4}\\
   \hline
\end{tabularx}
\caption{ \label{tbl: Test Lernmodus Statistik anzeigen}Test Lernmodus Statistik anzeigen, Quelle: Autoren}
\end{table}

\begin{table}[H]
\begin{tabularx}{\textwidth}{lX}
  \hline
  \multicolumn{2}{|c|}{Test Registrierung} \\
  \hline
  Test Nr. & 5\\
  Beschreibung & Durch diesen Test wird die Registrierung eines neuen Nutzers getestet \\
  Randbedingungen & Ein Nutzer mit diesem Benutzernamen ist noch nicht erstellt worden. \\
  erwartete Resultate & Auf dem Bildschirm wird die Meldung "User created successfully" angezeigt.  \\
  Testperson & Frederico Fischer \\
  Datum & 01.10.2020 \\
  Testprotokoll & \ref{tbl: testprotokoll5}\\
   \hline
\end{tabularx}
\caption{ \label{tbl: Test Registrierung}Test Registrierung, Quelle: Autoren}
\end{table}

\begin{table}[H]
\begin{tabularx}{\textwidth}{lX}
  \hline
  \multicolumn{2}{|c|}{Test Login} \\
  \hline
  Test Nr. & 6\\
  Beschreibung & Durch diesen Test wird das Login eines Nutzers getestet \\
  Randbedingungen & Ein Nutzer mit dem Benutzernamen \flqq peterpan\frqq und dem Passwort \flqq peterpan\frqq ist erfolgreich erstellt worden. \\
  erwartete Resultate & Der Nutzer wird erfolgreich angemeldet. Im Userprofil wird der korrekte Nutzer angezeigt.   \\
  Testperson & Frederico Fischer \\
  Datum & 01.10.2020 \\
  Testprotokoll & \ref{tbl: testprotokoll6}\\
   \hline
\end{tabularx}
\caption{ \label{tbl: Test Einloggen}Test Einloggen, Quelle: Autoren}
\end{table}

\begin{table}[H]
\begin{tabularx}{\textwidth}{lX}
  \hline
  \multicolumn{2}{|c|}{Test Zugriff geschütze Ressource} \\
  \hline
  Test Nr. & 7\\
  Beschreibung & Durch diesen Test wird der  Zugriff auf eine geschützte Ressource getestet.  \\
  Randbedingungen & Ein Nutzer mit dem Benutzernamen "peterpan" und dem Passwort "peterpan" ist erfolgreich angemeldet worden. \\
  erwartete Resultate & Der Nutzer greift auf die Lernfunktion zu. Er kann mit einer Lernsession beginnen.   \\
  Testperson & Frederico Fischer \\
  Datum & 01.12.2020 \\
  Testprotokoll & \ref{tbl: testprotokoll7}\\
   \hline
\end{tabularx}
\caption{ \label{tbl: Test Login}Test Zugriff geschütze Ressource, Quelle: Autoren}
\end{table}

\begin{table}[H]
\begin{tabularx}{\textwidth}{lX}
  \hline
  \multicolumn{2}{|c|}{Test Aufgabenstellung hinzufügen} \\
  \hline
  Test Nr. & 8\\
  Beschreibung & Durch diesen Test wird das Hinzufügen einer neuen Aufgabenstellung im Multiple Choice Format getestet. \\
  Randbedingungen & Es wird ein bereits eingerichteter Lehreraccount,  bestehende CategorySets und Categories, sowie die vorhandenen Bilder zur Verfügung gestellt. \\
  erwartete Resultate &  Die Frage wird erstellt und dem gewünschten CategorySet hinzugefügt. \\
  Testperson & Frederico Fischer \\
  Datum & 01.12.2020 \\
  Testprotokoll & \ref{tbl: testprotokoll8}\\
   \hline
\end{tabularx}
\caption{ \label{tbl: Test Aufgabenstellung hinzufuegen}Test Aufgabenstellung hinzufügen,  Quelle: Autoren}
\end{table}

\begin{table}[H]
\begin{tabularx}{\textwidth}{lX}
  \hline
  \multicolumn{2}{|c|}{Test Prüfung aus bestehenden Fragen erstellen} \\
  \hline
  Test Nr. & 9\\
  Beschreibung & Durch diesen Test wird das Erstellen und direkte Freigeben einer Prüfung getestet. \\
  Randbedingungen & Es wird ein bereits eingerichteter Lehreraccount,  bestehende CategorySets und Categories,  Schulklassen mit Schülern,  sowie die vorhandenen Bilder zur Verfügung gestellt. \\
  erwartete Resultate &  Die Prüfung kann mit den gewünschten Fragen und den gewählten Klassen erstellt werden.  \\
  Testperson & Frederico Fischer \\
  Datum & 01.12.2020 \\
  Testprotokoll & \ref{tbl: testprotokoll9}\\
   \hline
\end{tabularx}
\caption{ \label{tbl: Test Pruefung erstellen}Test Prüfung erstellen,  Quelle: Autoren}
\end{table}

\begin{table}[H]
\begin{tabularx}{\textwidth}{lX}
  \hline
  \multicolumn{2}{|c|}{Test Analysieren und bearbeiten einer abgeschlossenen Prüfung} \\
  \hline
  Test Nr. & 10\\
  Beschreibung & Durch diesen Test wird das nachträgliche Bearbeiten von Prüfungsresultaten durch einen Lehrer getestet. \\
  Randbedingungen & Es wird ein bereits eingerichteter Lehreraccount mit einer von Schülern absolvieren Prüfung zur Verfügung gestellt.  \\
  erwartete Resultate & Der Lehrer kann die Antworten von einzelnen Usern dessen Fragen bearbeiten und so die erreichte Punktezahl korrigieren.   \\
  Testperson & Frederico Fischer \\
  Datum & 01.12.2020 \\
  Testprotokoll & \ref{tbl: testprotokoll10}\\
   \hline
\end{tabularx}
\caption{ \label{tbl: Test Analysieren und bearbeiten einer abgeschlossenen Pruefung}Test Analysieren und bearbeiten einer abgeschlossenen Prüfung, Quelle: Autoren}
\end{table}

\begin{table}[H]
\begin{tabularx}{\textwidth}{lX}
  \hline
  \multicolumn{2}{|c|}{Test Absolvieren einer Prüfung nur während bestimmtem Zeitraum} \\
  \hline
  Test Nr. & 11\\
  Beschreibung & Durch diesen Test wird das Absolvieren einer Prüfung ausserhalb eines gültigen Zeitraums getestet. \\
  Randbedingungen & Es ist eine Prüfung mit einem Bearbeitungszeitraum vom 1.1.2020 - 2.1.2020 erstellt worden. Die Prüfung wurde der Klasse des Users freigegeben.  \\
  erwartete Resultate & Der Schüler kann die Prüfung nicht starten.  \\
  Testperson & Frederico Fischer \\
  Datum & 01.12.2020 \\
  Testprotokoll & \ref{tbl: testprotokoll11}\\
   \hline
\end{tabularx}
\caption{ \label{tbl: Test Absolvieren einer Pruefung nur waehrend bestimmtem Zeitpunkt}Test Absolvieren einer Prüfung nur während bestimmtem Zeitpunkt, Quelle: Autoren}
\end{table}

\begin{table}[H]
\begin{tabularx}{\textwidth}{lX}
  \hline
  \multicolumn{2}{|c|}{Test Absolvieren einer Prüfung ohne Berechtigung} \\
  \hline
  Test Nr. & 12\\
  Beschreibung & Durch diesen Test wird das Absolvieren einer Prüfung für einen nicht authorisierten Nutzer getestet.  \\
  Randbedingungen & Es ist eine Prüfung mit einem gütigen Bearbeitungszeitraum erstellt worden. Die Testperson besitzt einen Schüleraccount, für den die entsprechende Prüfung nicht zugewiesen wurde.  \\
  erwartete Resultate & Die Testperson kann die Prüfung nicht starten.  \\
  Testperson & Frederico Fischer \\
  Datum & 01.12.2020 \\
  Testprotokoll & \ref{tbl: testprotokoll12}\\
   \hline
\end{tabularx}
\caption{ \label{tbl: Test Absolvieren einer Pruefung ohne Berechtigungen}Test Absolvieren einer Prüfung ohne Berechtigungen, Quelle: Autoren}
\end{table}

\begin{table}[H]
\begin{tabularx}{\textwidth}{lX}
  \hline
  \multicolumn{2}{|c|}{Test Migration von Daten} \\
  \hline
  Test Nr. & 13\\
  Beschreibung & Durch diesen Test wird die Migration der Daten in die neue Applikation getestet.  \\
  Randbedingungen & Die bestehende Datenbank wurde gelöscht und die Applikation neu gestartet.  Die csv-Dateien werden mit den zur Verfügung gestellten Postman Kommandos an die API gesendet.  \\
  erwartete Resultate & Die Daten werden erfolgreich in die Applikation migriert.  Bei der Migration der Fragen ist nach Auftreten einer Fehlermeldung ein erneutes Senden nötig. In der Datenbank befinden sich rund 3000 Fragen. \\
  Testperson & Frederico Fischer \\
  Datum & 01.12.2020 \\
  Testprotokoll & \ref{tbl: testprotokoll13}\\
   \hline
\end{tabularx}
\caption{ \label{tbl: Test Migration von Daten}Test Migration von Daten, Quelle: Autoren}
\end{table}
\newpage

% !TEX root = PREN2_Dokumentation.tex
\section{System-Spezifikation}\label{SystemSpezifikation}
\subsection{Systemübersicht}
\subsubsection{Systemarchitektur}
\begin{figure}[H]
    \centering
    \includegraphics[width=1\textwidth]{images/system.png}
    \caption[Systemarchitektur]{Systemarchitektur, Quelle: Autor}
    \label{img: Systemarchitektur des Projektes}
\end{figure}
\newpage
\subsubsection{Kontextdiagramm}\label{Kontextdiagram}
\begin{figure}[H]
    \centering
   \includegraphics[width=1\textwidth]{images/kontextdiagramm.png}
    \caption[Kontextdiagramm]{Kontextdiagramm, Quelle: Autor}
    \label{img: Kontextdiagramm des Projektes}
\end{figure}
\newpage
\newpage
\subsection{Architektur und Designentscheide}
\subsubsection{Modelle und Sichten}
In diesem Projekt wird zwischen drei verschiedenen Sichten unterschieden:
\begin{itemize}
    \item \textbf{Kunde} Es handelt sich dabei um die Person, welche in der \ac{PWA} Produkte bestellt und diese abholt. 
    \item \textbf{Administrator} Dem Administrator ist es möglich, Produkt hinzuzufügen, zu verändern oder auch zu löschen. Er kann die kritischen Bestände an den Stationen abfragen. 
    \item \textbf{Programmierer} Dieser konzipiert und realisiert die Applikation gemäss den Anforderungen des Auftraggebers.
\end{itemize}
\newpage
\subsubsection{Daten (Mengengerüst und Strukturen)}
\paragraph{Datenbankschema}
Das Datenbankschema wurde von Intelij generiert. 
\begin{figure}[H]
    \centering
    \includegraphics[width=1\textwidth]{images/databaseSchema.png}
    \caption[Datenbankschema]{Datenbankschema, Quelle: Autor}
    \label{img: datebankschema}
\end{figure}
\subsubsection{Entwurfsentscheide}
\paragraph{Frontend}
\subparagraph{Technologien}
\begin{itemize}
	\item Spring Boot 2.4.3
	\item Java 11
\end{itemize}
\subparagraph{Projektstruktur}
Im Projekt wurde eine einheitliche Projektstruktur verfolgt. Jeder Component befindet sich, zusammen mit den entsprechenden Services, in einem Ordner. Services ohne Component befinden sich in \glqq services\grqq{}. 
\subparagraph{Konfigurationen}
 Es wird zwischen produktiver und Entwicklungsumgebung unterschieden. Es wird nur die produktive Konfiguration aufgezeigt. 
\begin{verbatim}
	export const environment = {
		production: true, 
		appVersion: require('../../package.json').version,
		serverUrl: "https://bdaf21-owerlen-02.enterpriselab.ch:8080/api/v1/",
		paymentSucessfullRedirect: 
		"https://bdaf21-owerlen.enterpriselab.ch/checkPaymentAssert", 
		paymentNotSucessfullRedirect: 
		"https://bdaf21-owerlen.enterpriselab.ch/paymentError", 
		imageFolder: "assets/min/"
	};
	
\end{verbatim}
\paragraph{Backend}
\subparagraph{Technologien}
\begin{itemize}
	\item Spring Boot 2.4.3
	\item Java 11
\end{itemize}
\subparagraph{Projektstruktur}
Das Projekt ist in mehrere Packages aufgeteilt. 
\begin{itemize}
	\item Controller
	\item Services
	\item DTO
	\item Entities
	\item Repositories
\end{itemize}
\subparagraph{Data Transfer Object}\label{DTO}
Es handelt sich hier um ein Enterprise Application Architecture Pattern von Martin Fowler. Genauer handelt es sich um ein Distribution Pattern. \\
Durch den Einsatz dieses Patterns können mehr Daten mit einem Aufruf übertragen werden. Ein weiterer Vorteil ist die klare Trennung zwischen zu serialisierendem Objekt und dem Domain Model. In der Regel wird ein Assembler genutzt, um das DTO auf das Domain Model zu mappen. [\cite{dtoFowler}]
\begin{figure}[H]
	\centering
	\includegraphics[width=1\textwidth]{images/dtoSketch.png}
	\caption[DTO Klassendiagramm von Martin Fowler]{DTO Klassendiagramm von Martin Fowler, Quelle: \cite{dtoFowler}}
	\label{img: dtoFowler}
\end{figure}
\subparagraph{\ac{HATEOAS}}
In diesem Projekt stand der Einsatz des DTO-Pattern jedoch im Konflikt mit der REST-Abstufung von Leonard Richardson und dem von ihm entworfenen Richardson Maturity Model. Gemäss diesem sollen Daten von einem anderen Domain Model als Hyperlink zurückgegeben werden, um das höchste Level zu erreichen. Somit liefert jeder Aufruf nur das angeforderte Modell zurück. Das Data Transfer Object Pattern dient somit nur zur Abgrenzung zwischen dem zu sendenden Objekt und dem Domain Model. 
Dabei wurde auf REST-Level 3 hingearbeitet \ref{img: richardsonMaturity} und mit \ac{HATEOAS} gearbeitet. Dadurch kann das Backend weiter vom Frontend getrennt werden, da bei eine Anpassung der URL auf dem Server keinen Einfluss auf die Funktionalität des Clients hat. Zudem kann die Verständlichkeit der API verbessert und ein Erweitern vereinfacht werden[\cite{richardsonMaturity}]. 
\subparagraph{Datenbank}
Die Datenbank wird durch die Spring Data JPA aus den Entities in diesem Projekt erstellt. 
\subparagraph{Konfigurationen}
Im Backend wurden die Konfigurationen im \gls{Propertie-File} vorgenommen. Es wird zwischen produktiver und Entwicklungsumgebung unterschieden. Vertrauliche Daten wurden bei der Dokumentation entfernt
\begin{verbatim}
	image.path=/var/
	spring.hateoas.use-hal-as-default-json-media-type=false
	
	# App Properties
	pickupbackend.app.jwtSecret= ""
	pickupbackend.app.jwtExpirationMs= 3600000
	pickupbackend.app.saferpay = ""
	
	#saferpay
	pickupbackend.app.saferpay.customerid = 257753
	pickupbackend.app.saferpay.terminalid = 17731797
	pickupbackend.app.saferpay.specVersion = 1.21
	
	#Jumio
	pickupbackend.app.jumio.username = ""
	pickupbackend.app.jumio.password = ""
	
	server.port= 8080
	security.require-ssl=false
	server.ssl.key-store=/etc/letsencrypt/live/
	bdaf21-owerlen-02.enterpriselab.ch/keystore.p12
	server.ssl.key-store-password = jtiPickUp2021
	server.ssl.keyStoreType= PKCS12
	server.ssl.keyAlias= tomcat
	
	springdoc.swagger-ui.path=/api/v1/swagger-ui/swagger-ui-custom.html
	
\end{verbatim}
\paragraph{Raspberry Pi Client}
\subparagraph{Technologien}
\begin{itemize}
	\item Node.js 14.14.0
	\item raspi-serial 6.0.0
	\item Raspberry Pi OS 
\end{itemize}
\subparagraph{Projektstruktur}
Bei diesem Projekt wurde die vorgegebene Projektstruktur von Node.js genutzt.

\subsection{Schnittstellen}
\subsubsection{Externe Schnittstellen}
\paragraph{REST API}
\subparagraph{API-Dokumention}
Die Spring Doc ermöglicht eine sehr einfache und schnelle Dokumentation von REST-API basierend auf der OpenAPI Spezifikation. Um die Dokumentation anzuzeigen, wird das Open Source Tool Swagger UI genutzt. Hiermit lässt sich eine dynamische API-Dokumentation als HTML-Page erstellen. 
\begin{figure}[H]
	\centering
	\includegraphics[scale=0.3]{images/swaggerui.png}
	\caption[Ausschnitt aus der Swagger Dokumentation]{Ausschnitt aus der Swagger Dokumentation, Quelle: Autor}
	\label{img: swaggerUI}
\end{figure}
\paragraph{Benutzerschnittstellen}
Auf ein Beschreiben der Benutzerschnittstelle wird an dieser Stelle verzichtet. 
\paragraph{Saferpay JSON-API}
Die \href{http://saferpay.github.io/jsonapi/}{JSON-API} von Six wird im offiziellen \gls{Github} Repository ausführlich beschrieben. 

\paragraph{Jumio API}
Die Jumio-API wird im \href{https://github.com/Jumio/implementation-guides}{Implementation Guide} beschrieben. 

\subsubsection{Wichtige interne Schnittstellen}
 \paragraph{UART-Übertragung}
 \subparagraph{Steckbrief}
Die Schnittstelle \ac{UART} wird für die Kommunikation zwischen Elektrotechnik und Informatik genutzt. 
\subparagraph{Interaktionen}
\begin{itemize}
	\item write
	\item read
\end{itemize}
 
 \subparagraph{Einsatz, Abläufe, Voraussetzungen und Zusicherung}
 \begin{itemize}
 	\item Um die Schnittstelle nutzen zu können, ist es nötig, am Serial Port \glqq serial0\grqq{} einen passenden Endpunkt angeschlossen zu haben.  
 	\item Das Node.js package \glqq raspi-serial\grqq{} muss installiert sein.
 \end{itemize}
 
 \subparagraph{Aufbau und Konfiguration}
 \begin{verbatim}
 	var serial = new Serial(DEFAULT_PORT = '/dev/serial0');
 \end{verbatim}
 \subparagraph{Fehlerbehandlung}
Eine explizite Fehlerbehandlung ist nicht vorhanden. 
 
 \subparagraph{Format}
 Die Daten müssen im folgenden Format übergeben werden: 
 \begin{verbatim}
 	productCode-productQuantity, 
 \end{verbatim}
 Es dürfen beliebig viele String kaskadiert werden. 
 \subparagraph{Beispielverwendung}
\begin{figure}[H]
	\centering
	\includegraphics[scale=0.6]{images/sendDataUart.png}
	\caption[Senden von Daten via Schnittstelle]{senden von Daten via Schnittstelle, Quelle: Autor}
	\label{img: send}
\end{figure}

\begin{figure}[H]
	\centering
	\includegraphics[scale=0.6]{images/readDataUart.png}
	\caption[Lesen von Daten via Schnittstelle]{Lesen von Daten via Schnittstelle, Quelle: Autor}
	\label{img: read}
\end{figure}

Eine weitere Beispielanwendung ist im Kapitel \ref{kommTiny} zu finden. 
\subsection{Environment-Anforderungen}\label{environmentanforderungen}
\subsubsection{Hardware}
Folgende Hardware wurde für diese Applikation verwendet und kann als ausreichend betrachtet werden:
\begin{itemize}
	\item CPU: Intel(R) Xeon(R) CPU E5-2630 v4 @ 2.20GHz
	\item RAM: 4GB
\end{itemize}
\subsubsection{Software}
Die ganze Applikation läuft auf virtuellen Maschine, auf der folgendes Betriebssystem installiert ist:
\begin{itemize}
	\item Ubuntu 20.04.01
	\item Docker 19.03.8
	\item certbot
\end{itemize}
\subsubsection{Kompatibilität}
\paragraph{Browser}
Die Software wurde auf den folgenden Browsern getestet. 
\begin{itemize}
	\item Safari 12
	\item Chrome 90
\end{itemize}
\newpage
% !TEX root = PREN2_Dokumentation.tex
\section{Anforderungen}\label{Anforderungen}
\subsection{Einführung}

\subsection{Einzelanforderungen}
\subsubsection{Funktionale Anforderungen}
\begin{table}[H]
    \setlength\extrarowheight{2pt} % for a bit of visual "breathing space"
    \begin{tabularx}{\textwidth}{|l|X|l|}
        \hline
        \textbf{ID} &  \textbf{Anforderung} & \textbf{Muss/Kann}  \\
        \hline
        F.1 & Die Web-Applikation muss die Punkte in der von Google aufgestellten Core Progressive Web App checklist \cite{pwaChecklist} erfüllen& Muss \\
        \hline
        F.2 & Die Web-Applikation ist auf eine phyisische Pick-Up Station abgestimmt. & Muss \\
        \hline
        F.3 & Die Web-Applikation ist international einsetzbar & Kann \\
        \hline

    \end{tabularx}
    \caption{ \label{tbl: Funktionsanforderungen}Anforderungen, Quelle: Autoren}
\end{table}
\subsubsection{Anforderungen an den Kaufprozess}
\begin{table}[H]
    \setlength\extrarowheight{2pt} % for a bit of visual "breathing space"
    \begin{tabularx}{\textwidth}{|l|X|l|}
    	\hline
        F.1 & Die Web-Applikation bietet die Möglichkeit, Kunden zu Erfassen & Muss \\
		\hline
		F.2 & Die Web-Applikation bietet die Möglichkeit, durch die Anbindung an eine 3rd Party eine Altersverifikation durchzuführen  & Muss \\
		\hline
		F.3 & Die Web-Applikation ermöglicht es, verschiedene Produkte zu bestellen & Kann \\
		\hline
		F.4 & Die Web-Applikation ermöglicht die Anbindung an einen bereits bekannten Bezahldienst, um eine sichere Bezahlung zu garantieren & Muss \\
		\hline
		F.5 & Die Web-Applikation ermöglicht es, verschiedene Produkte zu bestellen & Kann \\
		\hline
		F.6 & Dem Kunden kann eine Pick-Up Station zur Abholung auswählen, um die für Ihn nächstgelegene Station auswählen zu können  & Kann \\
		\hline
    \end{tabularx}
    \caption{ \label{tbl: Anforderungen an die Benutzbarkeit}Anforderungen, Quelle: Autoren}
\end{table}
\subsubsection{Anforderungen an die Leistungsfähigkeit}
\begin{table}[H]
    \setlength\extrarowheight{2pt} % for a bit of visual "breathing space"
    \begin{tabularx}{\textwidth}{|l|X|l|}
        \hline
        \textbf{ID} &  \textbf{Anforderung} & \textbf{Muss/Kann}  \\
        \hline
        L.1 & Das System muss jederzeit von mindestens 100 bis 120 Benutzern gleichzeitig verwendet werden können. & Muss \\
        \hline
 \end{tabularx}
    \caption{ \label{tbl: Anforderungen an die Leistungsfaehigkeit}Anforderungen, Quelle: Autoren}
\end{table}
\subsubsection{Nicht funktionale Anforderungen}
\begin{table}[H]
    \setlength\extrarowheight{2pt} % for a bit of visual "breathing space"
    \begin{tabularx}{\textwidth}{|l|X|l|}
        \hline
        \textbf{ID} &  \textbf{Anforderung} & \textbf{Muss/Kann}  \\
        \hline
        Q.1 & Als Benutzer muss ich mich bei vorhandener Internetverbindung jederzeit im System einloggen können. & Muss \\
        \hline
    \end{tabularx}
    \caption{ \label{tbl: Anforderungen an die Qualitaet}Anforderungen, Quelle: Autoren}
\end{table}
\newpage

\section{Sitzungsprotokolle}
Auf den nachfolgenden Seiten sind alle Protokolle von den durchgeführten Sitzungen ersichtlich.

\subsection{14.09.2020}
\subsubsection{Ordnungsaufruf}
Eine Besprechung aller Projektbeteiligten fand online als Zoom-Meeting am 14.09.2020 um 11:00 Uhr statt.
\subsubsection{Teilnehmer}
\begin{table}[H]
	\setlength\extrarowheight{2pt} % for a bit of visual "breathing space"
	\begin{tabularx}{\textwidth}{|X|X|}
		\hline
		\textbf{Anwesende Mitglieder} &  \textbf{Nicht anwesende Mitglieder} \\
		\hline
		Jordan Sucur, Betreuungsperson &  -\\
		Roger Buser, Auftraggeber &  \\
		Oliver Werlen, Projektleiter &  \\
		Frederico Fischer, Scrum Master &  \\
		\hline
	\end{tabularx}
	\caption{ \label{tbl: Teilnehmerliste vom 14.09.2020}Sitzungsprotokoll, Quelle: Autoren}
\end{table}
\subsubsection{Genehmigung des Protokolls}
Es handelt sich hierbei um die erste Sitzung in diesem Projekt. Es ist noch kein Protokoll vorhanden. 
\subsubsection{Ankündigungen}
Gemäss dem Zeitplan wird in zwei Wochen die Systemkontextphase abgeschlossen. Dabei erfolgt der erste Release eines GUI-Prototypen.
\subsubsection{besprochene Punkte}
\paragraph{Registrierung}
\begin{itemize}
    \item Lehrer bestimmt, wer im Kurs ist und kann Teilnehmer nachträglich bearbeiten
    \item Umsetzung ist dabei frei
\end{itemize}
\paragraph{Anforderungen}
\begin{itemize}
	
	\item Anforderungen müssen einheitlicher formuliert werden. Alle als Epics oder Userstories.
	\item Anforderungen müssen strukturiert werden, basierend auf dem Strukturplan.
    \item Kategorie für "weitere Anforderungen" definieren. 
    \item In der Software sollen Fragen nicht nur importiert, sondern auch exportiert werden. Format ist dabei frei.
    \item Als Sponsor soll die Hochschule Luzern inklusive dem Studiengang Gebäudetechnik gut sichtbar sein.
    \item Antworten sollen im Übungsmodus direkt auf Korrektheit überprüft werden können. Das Ergebnis ist direkt sichtbar. Sollte dies vom Schüler nicht gewünscht werden, kann eine Übungsprüfung absolviert werden.
    \item Anstatt eines Handbuchs wird ein interaktives Tutorial zur Verfügung gestellt. Ein Handbuch ist nicht nötig.
	\item Spezifizieren wie man Applikation erweitern müsste, falls doch mehr als 120 Nutzern vorhanden sind. 
    \item Applikation muss parallel nutzbar sein (als Anforderung spezifizieren).
    \item Quizes als Kann-Funktion erweitern
 
 \end{itemize}
\paragraph{Stand der Praxis in Dokumentation}
	\begin{itemize}
		\item Beschreiben, wieso alte Software nicht mehr Nutzbar ist.
		\item Was könnte man besser machen?
		\item Gibt es noch andere vergleichbare Applikationen auf dem Markt?
	\end{itemize}
\paragraph{Kapitelaufteilung in Dokumentation}
\begin{itemize}
	\item Projektmanagementplan so belassen, Änderungen notieren und jeweils auf den Anhang verweisen
\end{itemize}
\paragraph{Anderes}
\begin{itemize}
	\item Blockdiagramme: Datenfluss spezifizieren
\end{itemize}
\subsubsection{Tagesordnung der nächsten Sitzung}
\begin{itemize}
	\item Analyse GUI-Prototyp
	\item UML-Diagramme besprechen
	\item Besprechung CI/CD
\end{itemize}

\subsubsection{Unterschriften}

Mit der Unterschrift gilt das Sitzungsprotokoll als bestätigt.\\

Ort, Datum:	\underline{\hspace*{6cm}}	\hspace*{1cm} Visum:	\underline{\hspace*{4cm}}
%\includegraphics[width=1\textwidth]{bilder/unterschriften}
\newpage



\section{Rahmenpläne}\label{Rahmenplaene}
\begin{figure}[H]
    \centering
   \includegraphics[width=1\textwidth]{images/SoDa_Zeitstrahl_v1.png}
    \caption[SoDa Rahmenplan Version 1]{Rahmenplan Version 1,\\ Quelle: Autor}
    \label{img: SoDa Rahmenplan_v1}
\end{figure}
\section{Tokenauflösung mit jwt.io}\label{tokenaufloesung}
\begin{figure}[H]
	\centering
	\includegraphics[scale=0.4]{images/jwtIO.PNG}
	\caption[Token Auflösung mit jwt.io]{Token Auflösung mit jwt.io, Quelle: Autor}
	\label{img: jwtio}
\end{figure} 
\section{Ausschnitt aus der Six Payment Backoffice}\label{sixPayment}
\begin{figure}[H]
	\centering
	\includegraphics[width=1\textwidth]{images/paymentsBackoffice.PNG}
	\caption[Bezahlhistorie Backoffice]{Bezahlhistorie Backoffice, Quelle: Autor}
	\label{img: paymentsBackoffice}
\end{figure} 

\section{Drehbuch Pitching Video}
\subsection{Elevator Pitch}
Suchst auch du eine Möglichkeit, bequem mit deinem Smartphone Zigaretten zu kaufen und sie direkt abzuholen? 

\subsection{Problem}
Es gibt viele verschiedene Möglichkeiten, Tabakwaren zu kaufen- aber keine, die 24h verfügbar ist und auch vollkommen autonom.

\subsection{Lösung}
Die JTI Pick-Station ist die Lösung für dieses Problem. Sie ist 24h verfügbar, bietet einen schnellen, intuitiven Kaufprozess. Und das Beste: ohne Zusatzkosten.  

\subsection{Marktpotenzial}
27,1 Prozent der Schweizer Bevölkerung sind Raucher. 
Pro Tagen werden in Dagmarsellen 53 Millionen Zigaretten hergestellt. 

\subsection{Geschäftsmodell}
Preis pro Paket bleibt gleich
Mehr Kunden für JTI gewinnen
Finanzierung über Tabakverkauf

\subsection{Vergleich mit Wettbewerb}
Einfach und günstig im Aufbau (avec)
Weite Verbreitung von Stations (avec)
Keine Wartezeit, keine Lieferkosten (Coop at home) 
                                                                                                             
\subsection{Team}
Oliver Werlen, Informatik
Arnold Philipp, Elektrotechnik
Lucas Eberli, Maschinenbau

\subsection{Fazit}
enormes Marktpotenzial
einzigartig
innovativ
Kontaktdaten

\begin{figure}[H]
	\centering
	\includegraphics[scale=0.3]{images/richardsonMaturity.png}
	\caption[Richardson Maturity Model]{Richardson Maturity Model,\\ Quelle: \cite{richardsonMaturity}}
	\label{img: richardsonMaturity}
\end{figure}




% !TEX root = PREN2_Dokumentation.tex
\section{Originale Aufgabenstellung}\label{originaleaufgabenstellung}
%\includepdf[pages=-]{pdf/Aufgabenstellung_Fischer_Werlen.pdf}
\section{Zwischenpräsentation}\label{zwischenpraesentation}
\includepdf[pages=-,pagecommand=\thispagestyle{plain}]{pdf/zwischenpraesentation.pdf}
\section{API Documentation mit Swagger}\label{SwaggerDokumentation}
%\includepdf[pages=-,pagecommand=\thispagestyle{plain}]{pdf/apiDocumentation.pdf}
\end{document}
