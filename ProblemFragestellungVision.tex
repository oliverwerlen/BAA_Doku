% !TEX root = PREN2_Dokumentation.tex
\section{Problem und Vision}\label{Problem}
\subsection{Problem}
Der Onlinekauf ist beim Zigarettenkauf ein sehr selten genutzter Absatzweg. Dabei sind vorallem die nachfolgenden Punkte verantwortlich für die seltene Nutzung dieses Angebots. 
\begin{itemize}
	\item Versandkosten/Mindestbestellwert
	\item Dauer bis Ware beim Endkonsumenten
	\item Angebot nur in begrenztem Zeitraum möglich
\end{itemize}

\subsubsection{Versandkosten/Mindestbestellwert}
Bei diversen Onlineshops kommen bei zu geringer Bestellmenge erhebliche Versandkosten hinzu. So kostet der Versand per Paket in der Regel 9 Franken. Bei Kioskolino ist der Versand ab einem Bestellwert von 139.- Fr. Portofrei. [\cite{kioskolino}]
Bei Coop ist die Liefergebühr sogar noch höher. Sie beträgt 17.90 Fr. Die Versandkosten nehmen mit zunehmendem Bestellwert ab. Ab 500.- Fr. ist der Versand kostenlos. [\cite{coop}]
\subsubsection{Dauer bis Ware bei Endkonsumenten}
Bei der Bestellung bei Kioskolino ist die Ware innerhalb von 1-3 Werktagen beim Konsumenten. [\cite{kioskolino}]Für die meisten Kunden dauert dies zu lange. 
Coop verspricht die Lieferung am selben Tag. Dazu können bei der Bestellung verschiedene Zeitfenster ausgewählt werden, die Verfügbarkeit ist dabei von der Region abhängig. Die Ware muss aber frühzeitig bestellt werden, um die Lieferung am gleichen Tag garantieren zu können. Zudem bietet Coop auch die Möglichkeit, die Produkte direkt in der Filiale abzuholen. [\cite{coop}]\\
Das Problem ist aber mit beiden Anbieter identisch. Es muss die Ware sehr früh bestellt werden. Zudem dauert die Lieferung immer zwischen 4 Stunden bis zu 3 Tagen. Eine Lieferung am Sonntag ist dabei nicht möglich, Samstags wird nur am Nachmittag geliefert. 

\subsubsection{Angebot  nur in begrenztem Zeitraum möglich}
Die bestellte Ware wird nur zu bestimmten Zeiten ausgeliefert. So ist eine Lieferung an Sonn-und Feiertagen nicht möglich. 

\section{Vision}
Durch die JTI Pick-Up Station ist es dem Kunden möglich, seine Ware bequem im Onlineshop zu bestellen und anschliessend direkt und ohne Wartezeit an der gewünschten Pick-Up Station abzuholen. \\
Die Artikel werden durch den Kunden an der gewählten Pick-Up Station bereitgestellt. Durch das Vorzeigen der Bestellbestätigung durch den Kunden wird der Artikel freigegeben und steht zur Abholung bereit.\\ 
Ein Mindestbestellwert muss nicht erreicht werden. Zudem werden keine zusätzlichen Gebühren verlangt. \\
Die Applikation soll dabei durch eine einfache und intuitive Bedienung eine optimale Benutzerexperience bieten. Die Bestellung soll schnell und einfach ablaufen. Die Abholung soll anschliessend in kurzer Zeit abgewickelt werden. Durch die Umsetzung als \ac{PWA} ist die Applikation auch ohne aktive Internetverbindung nutzbar. \\
Durch das Verwenden eines bereits etablierten Altersverifikationsanbieters können zudem die rechtlichen Bedingungen erfüllt werden. Der Bezahlvorgang wird ebenfalls durch einen etablierten Anbieter durchgeführt. Dies garantiert eine sichere und zuverlässige Bezahlabwicklung. 

\newpage