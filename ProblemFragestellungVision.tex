% !TEX root = PREN2_Dokumentation.tex
\section{Problem und Vision}\label{Problem}
\subsection{Problem}
Der Onlinekauf von Zigaretten wird von Kunden sehr selten genutzt. Vor allem die nachfolgenden Punkte sind dafür verantwortlich. 
\begin{itemize}
	\item Versandkosten/Mindestbestellwert
	\item Dauer bis Ware beim Endkonsumenten
	\item Angebot nur in begrenztem Zeitraum möglich
\end{itemize}

\subsubsection{Versandkosten/Mindestbestellwert}
Bei diversen Onlineshops kommen bei zu geringer Bestellmenge erhebliche Versandkosten hinzu. So kostet der Versand per Paket in der Regel Fr. 9.00. Bei Kioskolino ist der Versand ab einem Bestellwert von Fr. 139.00 portofrei [\cite{kioskolino}]. 
\\Bei Coop ist die Liefergebühr höher, sie beträgt Fr. 17.90. Die Versandkosten nehmen mit zunehmendem Bestellwert ab. Bei einer Bestellung ab Fr. 500.00 ist der Versand kostenlos [\cite{coop}]. 
\subsubsection{Dauer bis Ware bei Endkonsumenten}
Bei der Bestellung bei Kioskolino wird die Ware innerhalb von drei Werktagen ausgeliefert [\cite{kioskolino}]. 
Coop verspricht die Lieferung am selben Tag, bei der Bestellung können verschiedene Zeitfenster ausgewählt werden. Die Verfügbarkeit ist von der Region abhängig. Die Ware muss aber frühzeitig bestellt werden, um die Lieferung am gleichen Tag garantieren zu können. Zudem bietet Coop auch die Möglichkeit, die Produkte direkt in der Filiale abzuholen [\cite{coop}].\\
Das Problem ist bei beiden Anbietern identisch. Die Lieferung dauert zwischen vier Stunden bis hin zu drei Tagen. 

\subsubsection{Angebot  nur in begrenztem Zeitraum möglich}
Die bestellte Ware wird nur zu bestimmten Zeiten ausgeliefert. Eine Lieferung an Sonn- und Feiertagen nicht möglich. 

\subsection{Vision}
Durch die JTI Pick-Up Station wird es dem Kunden möglich, seine Ware bequem im Onlineshop zu bestellen und ohne Wartezeit an der gewünschten Pick-Up Station abzuholen. \\
Die Artikel werden für den Kunden an der gewählten Pick-Up Station reserviert, durch das Vorzeigen der Bestellbestätigung wird der Artikel freigegeben und kann entnommen werden. \\ 
Ein Mindestbestellwert muss nicht erreicht werden, es fallen keine zusätzlichen Gebühren an.  \\
Die Applikation soll durch eine einfache und intuitive Bedienung eine optimale Benutzerexperience bieten. Die Bestellung soll schnell und einfach ablaufen, die Abholung in kurzer Zeit abgewickelt werden. Durch die Umsetzung als \gls{PWA} ist die Applikation ohne aktive Internetverbindung begrenzt nutzbar, sie unterscheidet sich nur geringfügig von einer nativen App. \\
Die rechtlichen Anforderungen können durch den Einsatz eines Altersverifikationsanbieters erreicht werden. Der Bezahlvorgang wird durch einen etablierten Anbieter durchgeführt. Das garantiert eine sichere und zuverlässige Bezahlabwicklung. 

\newpage