
	\pagenumbering{gobble}
		%\includegraphics[width=0.15\textwidth]{bilder/logoBrainpower.png}\par\vspace{1cm}
		{\scshape\LARGE Hochschule Luzern\par}
		\vspace{1cm}
		{\scshape\Large Bachelorarbeit\par}
		\vspace{1.5cm}
		{\huge\bfseries JTI Pick-Up Station\par}
		\vspace{1.5cm}
		{\Large\itshape Oliver Werlen\par}
		\vfill
		Betreut durch René Meier und Michael Handschuh\par
		Experte Stefan Bernet
		\vfill
		GitLab-Repositories: \par
		\begin{itemize}
			\item \href{https://gitlab.enterpriselab.ch/owerlen/pickup_backend}{Backend} https://gitlab.enterpriselab.ch/owerlen/pickup\char`_backend \par
			\item \href{https://gitlab.enterpriselab.ch/owerlen/pickup_frontend}{Frontend} https://gitlab.enterpriselab.ch/owerlen/pickup\char`_frontend\par
			\item \href{https://gitlab.enterpriselab.ch/owerlen/pickup-piserver}{Pi Station Client} https://gitlab.enterpriselab.ch/owerlen/pickup-piserver\par
		\end{itemize}
		\vfill
		Applikation: 
		\href{https://bdaf21-owerlen.enterpriselab.ch/home}{JTI Pick-Up Station} https://bdaf21-owerlen.enterpriselab.ch/home \par
		\vfill
		{\large \today\par}

	\newpage
	
	\noindent
	\fontsize{12}{14}
	\textbf{Bachelorarbeit an der Hochschule Luzern -- Informatik} \\ \vspace*{0.6cm}
	
	\fontsize{10.5}{12}
	\noindent
	\textbf{Titel:} JTI Pick-Up Station \newline \newline
	\textbf{Studentin/Student:} Oliver Werlen \newline \newline
	\textbf{Studiengang:} BSc Informatik \newline \newline
	\textbf{Jahr:} 2021\newline \newline
	\textbf{Betreuungsperson:} René Meier und Michael Handschuh \newline \newline
	\textbf{Expertin/Experte:} Stefan Bernet \newline \newline
	\textbf{Auftraggeberin/Auftraggeber:} JT International AG (Japan Tobacco International) \newline \newline \newline
	\textbf{Codierung / Klassifizierung der Arbeit:}\\
	$\boxtimes$ A: Einsicht 	(Normalfall) \\
	$\square$ B: R\"ucksprache	(Dauer:  \ \ \ \ \     Jahr / Jahre)\\
	$\square$ C: Sperre	(Dauer:  \ \ \ \ \     Jahr / Jahre)\\
	
	
	%%% you can use \boxtimes for filling a cross inside the square
	%%% e.g., $\boxtimes$ A: Einsicht 	(Normalfall) 
	
	
	\paragraph{\textbf{Eidesstattliche Erkl\"arung}}
	Ich erkl\"are hiermit, dass ich/wir die vorliegende Arbeit selbst\"andig und ohne unerlaubte fremde Hilfe angefertigt haben, alle verwendeten Quellen, Literatur und andere Hilfsmittel angegeben haben, w\"ortlich oder inhaltlich entnommene Stellen als solche kenntlich gemacht haben, das Vertraulichkeitsinteresse des Auftraggebers wahren und die Urheberrechtsbestimmungen der Fachhochschule Zentralschweiz (siehe Merkblatt <<Studentische Arbeiten>> auf MyCampus) respektieren werden. \newline \newline 
	Ort / Datum, Unterschrift	\underline{\hspace*{4cm}} 
	
	
	\newpage
	\noindent
	\textbf{Abgabe der Arbeit auf der Portfolio Datenbank:}\\
	Best\"atigungsvisum Studentin/Student\\
	Ich best\"atige, dass ich die Bachelorarbeit korrekt gem\"ass Merkblatt auf der Portfolio Datenbank abgelegt habe. Die Verantwortlichkeit sowie die Berechtigungen habe ich abgegeben, so dass ich keine \"Anderungen mehr vornehmen kann oder weitere Dateien hochladen kann. \newline \newline 
	Ort / Datum, Unterschrift	\underline{\hspace*{4cm}} \newline \newline
	\textbf{Verdankung}\\
	{	Ein Dankeschön geht an den Betreuer, Michael Handschuh, der während der gesamten Arbeit für Fragen immer zur Verfügung stand. Ein grosser Dank geht an das Team von Japan Tobacco International, Elena Nujic und Ralf Hohermuth. Ihre offene und einfache Art vereinfachte die Kommunikation. \\
	Auch Karl Ebener möchte ich meinen Dank aussprechen. Durch sein aufmerksames Korrekturlesen der gesamten Arbeit konnten viele grammatikalische Fehler korrigiert werden. } \newline \newline \newline
	\noindent
	{\textbf{Ausschliesslich bei Abgabe in gedruckter Form: \\
			Eingangsvisum durch das Sekretariat auszuf\"ullen}} \newline \newline
	Rotkreuz, den	\underline{\hspace*{4cm}} \hspace*{1cm}	Visum:	\underline{\hspace*{4cm}} \vspace*{10cm}
	
	

	\noindent
	{\textbf{Hinweis}}: Die Bachelorarbeit wurde von keinem Dozierenden nachbearbeitet. Ver\"offentlichungen (auch auszugsweise) sind ohne das Einverst\"andnis der Studiengangleitung der Hochschule Luzern -- Informatik nicht erlaubt. \newline \newline
	Copyright \textcopyright\ {2021} Hochschule Luzern -- Informatik \newline \newline
	Alle Rechte vorbehalten. Kein Teil dieser Arbeit darf ohne die schriftliche Genehmigung der Studiengangleitung der Hochschule Luzern -- Informatik in irgendeiner Form reproduziert oder in eine von Maschinen verwendete Sprache übertragen werden.
	
	\newpage
	\pagenumbering{arabic}
	

